\chapter{Propositional Calculus}
\section{Introduction}
\href{https://en.wikipedia.org/wiki/Propositional_calculus}{Propositional calculus}
(also known as \blue{propositional logic}) deals with the connection of \blue{propositions}
(a.k.a. \blue{simple statements}) through 
\blue{logical connectives}.  Here, logical connectives are words like ``\blue{and}'', ``\blue{or}'',
``\blue{not}'', ``\blue{if $\cdots$, then}'', and ``\blue{exactly if}''.  A
\blue{proposition} is a sentence  that 
\begin{itemize}
\item expresses a fact that is either true or false and
\item that does not contain any logical connectives.
\end{itemize}
Examples of propositions are the following:
\begin{enumerate}
\item ``\textsl{The sun is shining}''
\item ``\textsl{It is raining.}''
\item ``\textsl{There is a rainbow in the sky.}''
\end{enumerate}
We also refer to propositions as \blue{atomic} statements because they 
cannot be further decomposed into propositions.  Propositions can be combined by means of logical connectives
into \blue{composite statements}\index{composite statements}.  An example for a
composite statement would be
\\[0.2cm] 
\hspace*{1.3cm}
\textsl{\blue{If} the sun is shining \blue{and} it is raining, \blue{then} there is a rainbow in the sky.} 
\hspace*{\fill} (1)
\\[0.2cm]
This statement is composed of the three atomic propositions
\begin{itemize}
\item ``\textsl{The sun is shining.}'', 
\item ``\textsl{It is raining.}'', and
\item ``\textsl{There is a rainbow in the sky.}''
\end{itemize}
using the logical connectives ``\textsl{and}'' and ``\textsl{if $\cdots$, then}''.
Propositional calculus investigates how the truth value of composite statements is
calculated from the truth values of the propositions.  Furthermore, it investigates
how new statements can be derived from given statements. 

In order to analyze the structure of complex statements we introduce \blue{propositional variables}
\index{propositional variables}.
These propositional variables are just names that denote propositions.
Furthermore, we introduce symbols serving as mathematical operators for the logical connectives
``\blue{not}'', ``\blue{and}'', ``\blue{or}'', ``\blue{if, $\cdots$ then}'', and 
``\blue{if and only if}''.
\begin{enumerate}
\item $\neg a$ \index{$\neg a$}\quad\quad\ is read as \quad \blue{not} $a$ 
      \vspace*{-0.2cm}

\item $a \wedge b$ \index{$a \wedge b$}\,\quad\ is read as \quad $a$ \blue{and} $b$
      \vspace*{-0.2cm}

\item $a \vee b$ \index{$a \vee b$}\,\quad\ is read as \quad $a$ \blue{or} $b$
      \vspace*{-0.2cm}

\item $a \rightarrow b$ \index{$a \rightarrow b$}  \quad is read as \quad \blue{if} $a$, \blue{then} $b$
      \vspace*{-0.2cm}

\item $a \leftrightarrow b$ \index{$a \leftrightarrow b$} \quad is read as \quad  $a$ \blue{if and only if} $b$
\end{enumerate}
\blue{Propositional formulas} are built from propositional variables using the propositional operators shown
above and can have an arbitrary complexity.
Using propositional operators, the statement (1) can be written as follows:
 \\[0.2cm]
\hspace*{1.3cm}
$\texttt{sunny} \wedge \texttt{raining} \rightarrow \texttt{rainBow}$.
\\[0.2cm]
Here, we have used  $\texttt{sunny}$, $\texttt{rainy}$ and $\texttt{rainBow}$ as propositional variables.

Some propositional formulas are always true, no matter how the propositional variables are interpreted,
For example, the propositional formula 
\\[0.2cm]
\hspace*{1.3cm}
$p \vee \neg p$
\\[0.2cm]
is always true, it does not matter whether the proposition denoted by $p$ is true or false.  A propositional
formula that is always true is known as a  \blue{tautology}\index{tautology}.  There are also propositional
formulas that are never true.  For example, the propositional formula
\\[0.2cm]
\hspace*{1.3cm}
$p \wedge \neg p$
\\[0.2cm]
is always false.  A propositional formula is called \blue{satisfiable}\index{satisfiable}, if there is at least
one way to assign truth values to the variables such that the formula is true.  Otherwise the formula is called
\blue{unsatisfiable}\index{unsatisfiable}.  In this lecture we will discuss a number of different algorithms to
check whether a formula is satisfiable.  These algorithms are very important in a number of industrial
applications.  For example, a very important application is the design of digital circuits.
Furthermore, a number of puzzles can be translated into propositional formulas and finding a solution to these
puzzles amounts to checking the satisfiable of these formulas.  For example, we will solve the
\href{https://en.wikipedia.org/wiki/Eight_queens_puzzle}{eight queens puzzle} in this way.

The rest of this chapter is structured as follows:

\begin{enumerate}
\item We list several applications of propositional logic.
\item We define the notion of propositional formulas, i.e.~we define the set of strings that are
      propositional formulas.

      This is known as the \blue{syntax} of propositional formulas.
\item Next, we discuss the \blue{evaluation} of propositional formulas and implement the evaluation in \textsl{Python}.

      This is known as the \blue{semantics} of propositional formulas.
\item Then we formally define the notions \blue{tautology} and \blue{satisfiability} for propositional formulas.   
\item We discuss algebraic manipulations of propositional formulas and introduce the 
      \blue{conjunctive normal form}.

      Some algorithms discussed later require that the propositional formulas have conjunctive normal form.
\item After that we discuss the concept of a \blue{logical derivation}.  The purpose of a logical derivation is to
      derive new formulas from a given set of formulas.
\item Finally, we discuss the algorithm of \blue{Davis and Putnam} for checking the satisfiability of a set of
      propositional formulas.  As an application, we solve the eight queens puzzle using this algorithm.
\end{enumerate}

\section{Applications of Propositional Logic}
Propositional logic is not only the basis of first order logic, but it also has important practical
applications.  As there are many different applications of propositional logic, I will only list those
applications which I have seen myself during the years when I did work in industry.
\begin{enumerate}
\item Analysis and design of electronic circuits.

      Modern digital circuits are comprised of hundreds of millions of logical gates.\footnote{The web page
      \href{https://en.wikipedia.org/wiki/Transistor_count}{\texttt{https://en.wikipedia.org/wiki/Transistor\_count}}
      gives an overview of the complexity of modern processors.}
      A logical gate  is a building block, that represents a logical connective such as ``\blue{and}'',
      ``\blue{or}'', ``\blue{not}'' as an electronic circuit.

      The complexity of modern digital circuits would be unmanageable without
      the use of computer-aided verification methods.  The methods used are applications of propositional logic. 
      A very concrete application is \blue{circuit comparison}.  Here two
      digital circuits are represented as propositional formulas.
      Afterwards it is tried to show the equivalence of these formulas by means of propositional logic.
      Software tools, which are used for the verification of digital
      circuits sometimes cost more than $100\,000\,\symbol{36}$.
      For example, the company Magma offers the \blue{equivalence checker}
      \href{https://www.eetimes.com/document.asp?doc_id=1217672}{Quartz Formal} at a price
      of $150\,000\,
      \symbol{36}$ per license.  Such a license is then valid for three years.


\item Controlling the signals and switches of railroad stations.

      At a large railway station, there are several hundred switches and signals that have to be 
      reset all the time to provide routes for the trains.
      For safety reasons, different routes must not cross each other.  
      The individual routes are described by so-called \blue{closure plans}.
      The correctness of these closure plans can be analyzed via propositional formulas.
\item A number of \blue{puzzles} can be coded as propositional formulas and can then be solved with the
      algorithm of Davis and Putnam.
      For example, we will discuss the 
      \href{https://en.wikipedia.org/wiki/Eight_queens_puzzle}{eight queens puzzle} in this lecture.
      This puzzle asks to place eight queens on a chess board such that no two queens can attack each other.
\end{enumerate}

\section{The Formal Definition of Propositional Formulas}
In this section we first cover the \blue{syntax} of propositional formulas.  After that, we discuss their
\blue{semantics}. The  \blue{syntax}\index{syntax} defines the way in which we represent formulas as strings
and how we can combine formulas into a \blue{proof}.  The  \blue{semantics}\index{semantics} of propositional
logic is concerned with the  \blue{meaning} of propositional formulas.
We will first define the semantics of propositional logic with the help of set theory.
Then, we implement this semantics in \textsl{Python}.

\subsection{The Syntax of Propositional Formulas}
We define propositional formulas as strings.  To this end we assume a set  $\mathcal{P}$ \index{$\mathcal{P}$, set of propositional variables} 
of so called \blue{propositional variables}\index{propositional variables} as given. 
Typically, $\mathcal{P}$ is the set of all lower case Latin characters, which additionally may be indexed.
For example, we will use 
\\[0.2cm]
\hspace*{1.3cm}
$p$, $q$, $r$, $p_1$, $p_2$, $p_3$
\\[0.2cm]
as propositional variables.  Then, propositional formulas are strings that are formed from the alphabet
$$ 
  \mathcal{A} := \mathcal{P} \cup \bigl\{ \verum, \falsum, \neg, \vee, \wedge,
   \rightarrow, \leftrightarrow, (, ) \bigr\}.
$$
We define the set $\mathcal{F}$ of 
\blue{propositional formulas}\index{propositional formulas}
\index{$\mathcal{F}$: set of propositional formulas}
by induction:
\begin{enumerate}
\item $\verum \in \mathcal{F}$ and $\falsum \in \mathcal{F}$.

      Here $\verum$ \index{$\verum$, Verum} \index{Verum} denotes the formula that is always true, while $\falsum$
      \index{$\falsum$, Falsum} \index{Falsum} denotes the formula that is always false.
      The formula $\verum$ is called
      ``\blue{verum}''\footnote{``\href{https://translate.google.com/?sl=la&tl=en&text=falsum&op=translate&hl=en}{Verum}''
        is the Latin word for ``true''.},  while $\falsum$ is called 
      ``\blue{falsum}''\footnote{``\href{https://translate.google.com/?sl=la&tl=en&text=falsum&op=translate&hl=en}{Falsum}''
        is the Latin word for ``false''}.  
\item If $p \in \mathcal{P}$, then  $p \in \mathcal{F}$.

      Every propositional variable is also a propositional formula.
\item If $f \in \mathcal{F}$, then  $\neg f \in \mathcal{F}$.

      The formula  $\neg f$ \index{$\neg f$} (read: \blue{not $f$}) is called the \blue{negation}\index{negation} of $f$.
\item If $f_1, f_2 \in \mathcal{F}$, then we also have
      \begin{tabbing}
        $(f_1 \vee f_2) \in \mathcal{F}$ \index{$\vee$, or} \hspace*{0.5cm} \= (\textsl{read}: \quad \= $f_1$ or $f_2$ \hspace*{2.3cm} \=
         also: \blue{disjunction}\index{disjunction}\qquad\= of $f_1$ and $f_2$), \\
        $(f_1 \wedge f_2) \in \mathcal{F}$ \index{$\wedge$, and} \> (\textsl{read}: \> $f_1$ and $f_2$ \>
         also: \blue{conjunction}\index{conjunction}\> of $f_1$ and $f_2$), \\
        $(f_1 \rightarrow f_2) \in \mathcal{F}$ \index{$\rightarrow$, if $\cdots$, then} \> (\textsl{read}:       \> if $f_1$, then $f_2$ \>
         also: \blue{implication}\index{implication}\> of $f_1$ and $f_2$), \\
        $(f_1 \leftrightarrow f_2) \in \mathcal{F}$ \index{$\leftrightarrow$ if and only iff} \>
        (\textsl{read}:       \> $f_1$ if and only if $f_2$ \>
        also: \blue{biconditional}\index{biconditional}\> of $f_1$ and $f_2$).            
      \end{tabbing}
\end{enumerate}
The set  $\mathcal{F}$ of propositional formulas is the smallest set of those strings formed from the characters
in the alphabet $\mathcal{A}$ that has the closure properties given above.

\exampleEng 
Assume that $\mathcal{P} := \{ p, q, r \}$. Then we have the following:
\begin{enumerate}
\item $p \in \mathcal{F}$,
\item $(p \wedge q) \in \mathcal{F}$,
\item $\Bigl(\bigl(\,(\neg p \rightarrow q) \vee (q \rightarrow \neg p)\,\bigr) \rightarrow r\Bigr) \in \mathcal{F}$.  \qed
\end{enumerate}

\noindent
In order to save parentheses we agree on the following rules:
\begin{enumerate}
\item Outermost parentheses are dropped.  Therefore, we write \\[0.2cm]
      \hspace*{1.3cm} $p \wedge q$ \quad instead of \quad $(p \wedge q)$.
\item The negation operator $\neg$ has a higher precedence than all other operators. 
\item The operators  $\vee$ and $\wedge$ associate to the left.  Therefore, we write 
      \\[0.2cm]
      \hspace*{1.3cm} $p \wedge q \wedge r$ \quad instead of \quad $(p \wedge q) \wedge r$.
\item \underline{\textbf{\red{In this lecture}}} the logical operators 
      $\wedge$ and $\vee$ have the \underline{same} precedence.  This is different from the programming
      language  \textsl{Python}.  In \textsl{Python} the operator ``\texttt{and}'' has a higher precedence than
      the  operator ``\texttt{or}''.
      
      In the programming languages  \texttt{C} and \textsl{Java} the operator ``\texttt{\&\&}'' also has a higher
      precedence than the operator ``\texttt{||}''. 
\item The operator $\rightarrow$ is \underline{ri}g\underline{ht associative}, i.e.~we write \\[0.2cm]
      \hspace*{1.3cm} $p \rightarrow q \rightarrow r$ \quad instead of \quad $p \rightarrow (q \rightarrow r)$.
\item The operators  $\vee$ and $\wedge$ have a higher precedence than the operator $\rightarrow$.  Therefore,
      we write \\[0.2cm]
      \hspace*{1.3cm} $p \wedge q \rightarrow r$ \quad instead of \quad $(p \wedge q) \rightarrow r$.
\item The operator $\rightarrow$ has a higher precedence than the operator $\leftrightarrow$.  Therefore, we write \\[0.2cm]
      \hspace*{1.3cm} $p \rightarrow q \leftrightarrow r$ \quad instead of \quad $(p \rightarrow q) \leftrightarrow
      r$.
\item You should note that the operator $\leftrightarrow$ neither associates to the left nor to the right.
      Therefore, the expression
      \\[0.2cm]
      \hspace*{1.3cm}
      $p \leftrightarrow q \leftrightarrow r$
      \\[0.2cm]
      is \underline{\textbf{\red{ill-defined}}} and has to be parenthesised.  If you encounter this type of expression
      in a book it is usually meant as an abbreviation for the expression
      \\[0.2cm]
      \hspace*{1.3cm}
      $(p \leftrightarrow q) \wedge (q \leftrightarrow r)$.
      \\[0.2cm]
      We will not use this kind of abbreviation.
\end{enumerate}

\remarkEng
Later, we will conduct a series of proofs that prove mathematical statements about formulas.
In these proofs we will make use of propositional connectives.  In order to distinguish these connectives from
the connectives of propositional logic we agree on the following:
\begin{enumerate}
\item Inside a propositional formula, the propositional connective
      ``\blue{not}'' is written as ``$\neg$''.
  
      When we prove a statement about propositional formulas, we use the word ``\texttt{not}'' instead.
\item Inside a propositional formula, the propositional connective
      ``\blue{and}'' is written as ``$\wedge$''.
  
      When we prove a statement about propositional formulas, we use the word ``\texttt{and}'' instead.
\item Inside a propositional formula, the propositional connective
      ``\blue{or}'' is written as ``$\vee$''.
  
      When we prove a statement about propositional formulas, we use the word ``\texttt{or}'' instead.
\item Inside a propositional formula, the propositional connective
      ``\blue{if $\cdots$, then}'' is written as ``$\rightarrow$''.
  
      When we prove a statement about propositional formulas, we use the symbol ``$\Rightarrow$'' instead.
\item Inside a propositional formula, the propositional connective
      ``\blue{if and only if}'' is written as ``$\leftrightarrow$''.
  
      When we prove a statement about propositional formulas, we use the symbol ``$\Leftrightarrow$'' instead.
      \eox
\end{enumerate}

\subsection{Semantics of Propositional Formulas}
In this section we define the \blue{meaning} a.k.a.~the \blue{semantics}\index{semantics} of propositional
formulas.  To this end we assign truth values to propositional formulas.  First, we define the set 
$\mathbb{B}$ \index{$\mathcal{B} = \{ \texttt{True}, \texttt{False} \}$} of \blue{truth values}\index{truth values}:  \\[0.2cm] 
\hspace*{1.3cm} $\mathbb{B} := \{ \texttt{True}, \texttt{False} \}$. \\[0.2cm]
Next, we define the notion of a \blue{propositional valuation}.

\begin{Definition}[Propositional Valuation]
  A \blue{propositional valuation}\index{propositional valuation $\mathcal{I}$}
  \index{$\mathcal{I}$, propositional valuation} is a function \\[0.2cm]
  \hspace*{1.3cm} $\mathcal{I}:\mathcal{P} \rightarrow \mathbb{B}$, \\[0.2cm]
  that maps the propositional variables  $p\in \mathcal{P}$ to truth values $\mathcal{I}(p) \in \mathbb{B}$.
  \eox
\end{Definition}

A propositional valuation $\mathcal{I}$ maps only the propositional variables to truth values.
In order to map propositional formulas to truth values we need to interpret the propositional operators 
``$\neg$'', ``$\wedge$'', ``$\vee$'', ``$\rightarrow$'', and
``$\leftrightarrow$'' as functions on the set $\mathcal{B}$.  To this end we define the functions
$\circneg$, $\circwedge$, $\circvee$, $\circright$, and $\circleftright$.
These functions have the following signatures:
\begin{enumerate}
\item $\circneg: \mathbb{B} \rightarrow \mathbb{B}$ \index{$\circneg$}
\item $\circwedge: \mathbb{B} \times \mathbb{B} \rightarrow \mathbb{B}$ \index{$\circwedge$}
\item $\circvee: \mathbb{B} \times \mathbb{B} \rightarrow \mathbb{B}$ \index{$\circvee$}
\item $\circright: \mathbb{B} \times \mathbb{B} \rightarrow \mathbb{B}$ \index{$\circright$}
\item $\circleftright: \mathbb{B} \times \mathbb{B} \rightarrow \mathbb{B}$ \index{$\circleftright$}
\end{enumerate}
We will use these functions as the valuations of the propositional operators.
It is easiest to define the functions $\circneg$, $\circwedge$, $\circvee$, $\circright$, and $\circleftright$
via the following \blue{truth table} \index{truth table} (Table \ref{tab:aussagen-logik}):   


\begin{table}[!ht]
  \centering
\framebox{
  \begin{tabular}{|l|l||l|l|l|l|l|}
\hline
   $p$            & $q$             & $\circneg\;(p)$ & $\circvee\;(p, q)$ & $\circwedge\;(p, q)$ & $\circright\;(p, q)$ & $\circleftright\;(p, q)$
   \\
\hline
\hline
   \texttt{True}  & \texttt{True}   & \texttt{False}  & \texttt{True}  & \texttt{True}  & \texttt{True}     & \texttt{True}  \\
\hline
   \texttt{True}  & \texttt{False}  & \texttt{False}  & \texttt{True}  & \texttt{False} & \texttt{False}    & \texttt{False}  \\
\hline
   \texttt{False} & \texttt{True}   & \texttt{True}   & \texttt{True}  & \texttt{False} & \texttt{True}     & \texttt{False} \\
\hline
   \texttt{False} & \texttt{False}  & \texttt{True}   & \texttt{False} & \texttt{False} & \texttt{True}     & \texttt{True}  \\
\hline
  \end{tabular}}
  \caption{Interpretation of the propositional operators.}
  \label{tab:aussagen-logik}
\end{table}
Then the truth value of a propositional formula $f$ under a given propositional valuation $\mathcal{I}$ is
defined via induction of $f$.  We will denote the truth value as
$\widehat{\mathcal{I}}(f)$ \index{$\widehat{\mathcal{I}}(f)$}.  We have
\begin{enumerate}
\item $\widehat{\mathcal{I}}(\falsum) := \texttt{False}$.
\item $\widehat{\mathcal{I}}(\verum) := \texttt{True}$.
\item $\widehat{\mathcal{I}}(p) := \mathcal{I}(p)$ for all $p \in \mathcal{P}$.
\item $\widehat{\mathcal{I}}(\neg f) := \circneg\;\bigl(\widehat{\mathcal{I}}(f)\bigr)$ for all $f \in \mathcal{F}$.
\item $\widehat{\mathcal{I}}(f \wedge g) := \circwedge\;\bigl(\widehat{\mathcal{I}}(f), \widehat{\mathcal{I}}(g)\bigr)$ 
      for all $f, g \in \mathcal{F}$.
\item $\widehat{\mathcal{I}}(f \vee g) := \circvee\;\bigl(\widehat{\mathcal{I}}(f), \widehat{\mathcal{I}}(g)\bigr)$ 
      for all $f, g \in \mathcal{F}$.
\item $\widehat{\mathcal{I}}(f \rightarrow g) := \circright\;\bigl(\widehat{\mathcal{I}}(f), \widehat{\mathcal{I}}(g)\bigr)$ 
      for all $f, g \in \mathcal{F}$.
\item $\widehat{\mathcal{I}}(f \leftrightarrow g) := \circleftright\;\bigl(\widehat{\mathcal{I}}(f), \widehat{\mathcal{I}}(g)\bigr)$ 
      for all $f, g \in \mathcal{F}$.
\end{enumerate}
In order to simplify the notation we will not distinguish between the function
\\[0.2cm]
\hspace*{1.3cm}
$\widehat{\mathcal{I}}: \mathcal{F} \rightarrow \mathbb{B}$
\\[0.2cm]
that is defined on all propositional formulas and the function
\\[0.2cm]
\hspace*{1.3cm}
$\mathcal{I}: \mathcal{P} \rightarrow \mathbb{B}$.
\\[0.2cm]
Hence, from here on we will write $\mathcal{I}(f)$ instead of $\widehat{\mathcal{I}}(f)$.

\noindent
\textbf{Example}: We show how to compute the truth value of the formula
$$  (p \rightarrow q) \rightarrow (\neg p \rightarrow q) \rightarrow q $$
for the propositional valuation
\\[0.2cm]
\hspace*{1.3cm}
$\mathcal{I} := \{ p \mapsto \mathtt{True}, q \mapsto \mathtt{False} \}$.
\\[0.2cm]
\hspace*{1.3cm}
$
  \begin{array}[b]{lcl}
   \mathcal{I}\Bigl( (p \rightarrow q) \rightarrow (\neg p \rightarrow q) \rightarrow q  \Bigr) 
   & = &  \circright\Bigl( \mathcal{I}\bigl( (p \rightarrow q) \bigr),\, \mathcal{I}\bigl((\neg p \rightarrow q) \rightarrow q\bigr) \Bigr) \\[0.2cm]
   & = & \circright\Bigl( \circright\bigl( \mathcal{I}(p), \mathcal{I}(q) \bigr),\, \mathcal{I}\bigl((\neg p \rightarrow q) \rightarrow q\bigr) \Bigr) \\[0.2cm]
   & = & \circright\Bigl( \circright\bigl( \texttt{True}, \texttt{False} \bigr),\, \mathcal{I}\bigl((\neg p \rightarrow q) \rightarrow q\bigr) \Bigr) \\[0.2cm]
   & = & \circright\Bigl( \texttt{False}, \, \mathcal{I}\bigl((\neg p \rightarrow q) \rightarrow q\bigr) \Bigr) \\[0.2cm]
   & = & \texttt{True} 
  \end{array}
$ \eox
\\[0.2cm]
Note that we did just evaluate some parts of the formula.  The reason is that as soon as we know that the first
argument of $\circright$ is \texttt{False} the value of the corresponding formula can already be determined.
Nevertheless, this approach is too cumbersome.  Instead, we will evaluate a formula directly via the table
\ref{tab:aussagen-logik} on page \pageref{tab:aussagen-logik}.  For example, to evaluate
$$  (p \rightarrow q) \rightarrow (\neg p \rightarrow q) \rightarrow q $$
for any propositional valuation we can use the table \ref{tab:tautologie} on page
\pageref{tab:tautologie}.  This table contains a column for every subformula of the given formula.
\begin{table}[!ht]
  \centering
\framebox{
  \begin{tabular}{|l|l|l|l|l|l|l|}
\hline
   $p$ & $q$ & $\neg p$ & $p \rightarrow q$ & $\neg p \rightarrow q$ & $(\neg p \rightarrow q) \rightarrow q$ & $ (p \rightarrow q) \rightarrow (\neg p \rightarrow q) \rightarrow q$
   \\
\hline
\hline
   \texttt{True}  & \texttt{True}  & \texttt{False} & \texttt{True}  & \texttt{True}  & \texttt{True}     & \texttt{True}  \\
\hline
   \texttt{True}  & \texttt{False} & \texttt{False} & \texttt{False}  & \texttt{True} & \texttt{False}    & \texttt{True}  \\
\hline
   \texttt{False} & \texttt{True}  & \texttt{True}  & \texttt{True}  & \texttt{True} & \texttt{True}     & \texttt{True} \\
\hline
   \texttt{False} & \texttt{False} & \texttt{True}  & \texttt{True} & \texttt{False} & \texttt{True}     & \texttt{True}  \\
\hline
  \end{tabular}}
  \caption{Berechnung der Wahrheitswerte von $(p \rightarrow q) \rightarrow (\neg p \rightarrow q) \rightarrow q$}
  \label{tab:tautologie}
\end{table}
If we take a look at the last column of this table we observe that this column only contains the value
\texttt{True}.  Therefore, the evaluation of 
\\[0.2cm]
\hspace*{1.3cm}
$(p \rightarrow q) \rightarrow (\neg p \rightarrow q) \rightarrow q $
\\[0.2cm]
yields $\mathtt{True}$ for every propositional valuation $\mathcal{I}$.  
A formula that evaluates as \texttt{True} for every propositional valuation $\mathcal{I}$ is called a
\href{https://en.wikipedia.org/wiki/Tautology_(logic)}{tautology}.

We discuss the construction of this table by means of the second row.
In this row, the variable $p$ takes the value  \texttt{True}, while $q$ has the value \texttt{False}.  Hence we
have
\\[0.2cm]
\hspace*{1.3cm} $\mathcal{I}(p) = \texttt{True}$ and $\mathcal{I}(q) = \texttt{False}$. \\[0.2cm]
Then we have the following:
\begin{enumerate}
\item $\mathcal{I}(\neg p) = \circneg\,(\mathcal{I}(p)) = \circneg\,( \texttt{True}) = \texttt{False}$
\item $\mathcal{I}(p \rightarrow q) = \circright\,(\mathcal{I}(p), \mathcal{I}(q)) = \circright\,(\texttt{True}, \texttt{False}) = \texttt{False}$
\item $\mathcal{I}(\neg p \rightarrow q) = \circright\bigl( \mathcal{I}(\neg p), \mathcal{I}(q)\bigr) = \circright(\texttt{False}, \texttt{False}) = \texttt{True}$
\item $\mathcal{I}\bigl((\neg p \rightarrow q) \rightarrow q\bigr) = 
          \circright\bigl( \mathcal{I}(\neg p \rightarrow q), \mathcal{I}(q) \bigr) = 
          \circright( \texttt{True}, \texttt{False} ) = \texttt{False}$
\item $\mathcal{I}\bigl((p \rightarrow q) \rightarrow  (\neg p \rightarrow q) \rightarrow q\bigr) = 
      \circright\bigl( \mathcal{I}(p \rightarrow q),  \mathcal{I}((\neg p \rightarrow q) \rightarrow q)\bigr) = 
       \circright\,( \texttt{False},  \texttt{False} ) = \texttt{True}$
\end{enumerate}
For complex formulas this manual evaluation is too cumbersome and error prone.
Therefore, we will implement the evaluation of propositional formulas in \textsl{Python}.

\subsection{Implementation} 
In this section we develop a \textsl{Python} program that can evaluate propositional formulas.
Every time when we develop a program to compute something useful we have to decide which data structures are
most appropriate to represent the information that is to be processed by the program.  In this case we want to
process propositional formulas.  Therefore, we have to decide how to represent propositional formulas in
\textsl{Python}.  One obvious possibility would be to use strings.  However, this would be a bad choice as it
would then be difficult to access the parts of a given formula.  It is far more suitable to represent
propositional formulas as \blue{nested tuples}.  A nested tuple is a tuple that contains both strings and
nested tuples.  For example,
\\[0.2cm]
\hspace*{1.3cm}
\texttt{('$\wedge$', ('$\neg$', 'p'), 'q')}
\\[0.2cm]
is a nested tuple that represents the propositional formula $\neg p \wedge q$.

Formally, the representation of propositional formulas is defined by a function 
\\[0.2cm]
\hspace*{1.3cm}
$\textsl{rep}: \mathcal{F} \rightarrow \textsl{Python}$
\\[0.2cm]
that maps a propositional formula $f$ to the corresponding nested tuple \index{nested tuple}
$\textsl{rep}(f)$.  We define $\textsl{rep}(f)$ inductively by induction on $f$.
\begin{enumerate}
\item $\verum$ is represented as the tuple \texttt{('$\top$',)}.
  
      This is possible because \texttt{'$\top$'} is a unicode symbol and \textsl{Python} supports the use of unicode
      symbols in strings.   Alternatively, in \textsl{Python} the string \texttt{'$\top$'} can be written as
      \texttt{'\symbol{92}N\{up tack\}'} since ``\texttt{up tack}'' is the name of the unicode symbol
      ``$\top$'' and any unicode symbol that has the name  $u$ can be written 
      as \texttt{'\symbol{92}N\{$u$\}'} in \textsl{Python}.  Therefore, we have
      \\[0.2cm]
      \hspace*{1.3cm}
      $\textsl{rep}(\verum) := \texttt{('\symbol{92}N\{up tack\}',)}$.
\item $\falsum$  is represented as the tuple \texttt{('$\falsum$',)}.
      
      The unicode symbol \texttt{'$\falsum$'} has the name ``\texttt{down tack}''.
      Therefore, we have
      \\[0.2cm]
      \hspace*{1.3cm}
      $\textsl{rep}(\falsum) := \texttt{('\symbol{92}N\{down tack\}',)}$.
\item Since propositional variables are strings we can represents these variables by themselves:
      \\[0.2cm]
      \hspace*{1.3cm}
      $\textsl{rep}(p) := p$ \quad for all $p \in \mathcal{P}$.
\item If $f$ is a propositional formula, the  negation $\neg f$ is represented as a pair where
      we put the unicode symbol \texttt{'$\neg$'} at the first position, while the representation of $f$ is put
      at the second position.  As the name of the unicode symbol \texttt{'$\neg$'} is
      ``\texttt{not sign}'' we have
      \\[0.2cm]
      \hspace*{1.3cm} 
      $\textsl{rep}(\neg f) := \bigl(\texttt{'$\neg$'}, \textsl{rep}(f)\bigr)$.
\item If $f_1$ and $f_2$ are propositional formulas, we represent $f_1 \wedge f_2$ with the help of the
      unicode symbol \texttt{'$\wedge$'}.  This symbol has the name ``\texttt{logical and}''.  Hence we have
      \\[0.2cm]
      \hspace*{1.3cm} 
      $\textsl{rep}(f \wedge g) := \bigl(\texttt{'$\wedge$'}, \textsl{rep}(f), \textsl{rep}(g)\bigr)$.
\item If $f_1$ and $f_2$ are propositional formulas, we represent $f_1 \vee f_2$ with the help of the unicode symbol
      \texttt{'$\vee$'}. This symbol has the name ``\texttt{logical or}''.  Hence we have
      \\[0.2cm]
      \hspace*{1.3cm} 
      $\textsl{rep}(f \vee g) := \bigl(\texttt{'$\vee$'}, \textsl{rep}(f), \textsl{rep}(g)\bigr)$.
\item If $f_1$ and $f_2$ are propositional formulas, we represent $f_1 \rightarrow f_2$ with the help of the
      unicode symbol \texttt{'$\rightarrow$'}.  This symbol has the name ``\texttt{rightwards arrow}''.
      Hence we have
      \\[0.2cm]
      \hspace*{1.3cm} 
      $\textsl{rep}(f \rightarrow g) := \bigl(\texttt{'$\rightarrow$'}, \textsl{rep}(f), \textsl{rep}(g)\bigr)$.
\item If $f_1$ and $f_2$ are propositional formulas, we represent $f_1 \leftrightarrow f_2$ with the help of
      the unicode symbol \texttt{'$\leftrightarrow$'}.  This symbol has the name ``\texttt{left right arrow}''.
      Hence we have
      \\[0.2cm]
      \hspace*{1.3cm} 
      $\textsl{rep}(f \leftrightarrow g) := \bigl(\texttt{'$\leftrightarrow$'}, \textsl{rep}(f), \textsl{rep}(g)\bigr)$.
\end{enumerate}
When choosing the representation of a formula in \textsl{Python} we have a lot of freedom.
We could as well have represented formulas as objects of different classes.  A good representation should have
the following properties:
\begin{enumerate}
\item It should be intuitive, i.e.~we do not want to use any obscure encoding.
\item It should be adequate.
  \begin{enumerate}[(a)]
  \item It should be easy to recognize whether a formula is a propositional variable, a negation, a
        conjunction, etc.
  \item It should be easy to access the components of a formula.
  \item Given a formula $f$, it should be easy to generate the representation of $f$.
  \end{enumerate}
\item It should be memory efficient.
\end{enumerate}

A \blue{propositional valuation} is a function  
\\[0.2cm]
\hspace*{1.3cm} ${\cal I}: {\cal P} \rightarrow \mathbb{B}$ \\[0.2cm]
mapping the set of propositional variables ${\cal P}$ into the set of truth values
$\mathbb{B} = \bigl\{ \mathtt{True}, \mathtt{False} \bigr\}$.
We represent a propositional valuation $\mathcal{I}$ as the set of all propositional variables that are mapped
to \texttt{True} by $\mathcal{I}$:
\\[0.2cm]
\hspace*{1.3cm}
$\textsl{rep}(\mathcal{I}) := \bigl\{ x \in \mathcal{P} \mid \mathcal{I}(x) = \texttt{True} \bigr\}$.
\\[0.2cm]
This enables us to implement a simple function that evaluates a propositional formula $f$ with a given
propositional valuation $\mathcal{I}$.
The \textsl{Python} function \mytt{evaluate} is shown in Figure \ref{fig:evaluate.py} on page
\pageref{fig:evaluate.py}. 
The function \texttt{evaluate} takes two arguments.
\begin{enumerate}
\item The first argument $F$ is a propositional formula that is represented as a nested tuple.
\item The second argument $I$ is a propositional evaluation.  This evaluation is represented as a set of
      propositional variables. Given a propositional variables $p$, the value of $\mathcal{I}(p)$ is computed
      by the expression  ``$p \;\mathtt{in}\; I$''.
\end{enumerate}

\begin{figure}[!ht]
  \centering
\begin{minted}[ frame         = lines, 
                framesep      = 0.3cm, 
                bgcolor       = sepia,
                numbers       = left,
                numbersep     = -0.2cm,
                xleftmargin   = 0.3cm,
                xrightmargin  = 0.3cm
                ]{python3}
    def evaluate(F, I):
        match F:
            case p if isinstance(p, str): 
                return p in I
            case ('⊤', ):     return True
            case ('⊥', ):     return False
            case ('¬', G):    return not evaluate(G, I)
            case ('∧', G, H): return     evaluate(G, I) and evaluate(H, I)
            case ('∨', G, H): return     evaluate(G, I) or  evaluate(H, I)
            case ('→', G, H): return not evaluate(G, I) or  evaluate(H, I)
            case ('↔', G, H): return     evaluate(G, I) ==  evaluate(H, I)
\end{minted}
\vspace*{-0.3cm}
  \caption{Evaluation of a propositional formula.}
  \label{fig:evaluate.py}
\end{figure} 

\noindent
Next, we discuss the implementation of the function \texttt{evaluate()}.
\begin{enumerate}
\item We make use of the \texttt{match} statement, which is new in \textsl{Python} $3.10$.
      This new control structure is explained in the tutorial ``PEP 636: Structural Pattern Matching''
      available at
      \\[0.2cm]
      \hspace*{1.3cm}
      \href{https://peps.python.org/pep-0636/}{https://peps.python.org/pep-0636/}.
\item Line 3 deals with the case that the argument $F$ is a propositional variable.  We can recognize this by
      the fact that $F$ is a string, which we can check with the predefined function
      \texttt{isinstance}.

      In this case we have to check whether the variable $p$ is an element of the set $I$, because $p$ is
      interpreted as \texttt{True} if and only if $p \in I$.
\item If $F$ is $\verum$, then evaluating $F$ always yields \texttt{True}.
\item If $F$ is $\falsum$, then evaluating $F$ always yields \texttt{False}. 
\item If $F$ has the form $\neg G$, we recursively evaluate $G$ given the evaluation $I$ and negate the result.
\item If $F$ has the form $G \wedge H$, we recursively evaluate
      $G$ and $H$ using $I$.  The results are then combined with the \textsl{Python} operator
      ``\texttt{and}''.
\item If $F$ has the form $G \vee H$, we recursively evaluate
      $G$ and $H$ using $I$.  The results are then combined with the \textsl{Python} operator
      ``\texttt{or}''.
\item If $F$ has the form $G \rightarrow H$, we recursively evaluate
      $G$ and $H$ using $I$.  Then we exploit the fact that the two formulas    
      \\[0.2cm]
      \hspace*{1.3cm}
      $G \rightarrow H$ \quad und \quad $\neg G \vee H$
      \\[0.2cm]
      are equivalent.
\item If $F$ has the form $G \leftrightarrow H$, we recursively evaluate
      $G$ and $H$ using $I$.  Then we exploit the fact that the formula $G \leftrightarrow H$ is true if and
      only if $G$ and $H$ have the same truth value.
\end{enumerate}

\subsection{An Application}
Next, we showcase a playful application of propositional logic.  Inspector Watson is called to investigate a
burglary at a jewelry store.  Three suspects have been detained in the vicinity of the jewelry store.
Their names are Aaron, Bernard, and Cain.  The evaluation of the files reveals the following facts.
\begin{enumerate}
\item \green{At least one of these suspects must have been involved in the crime.}

      If the propositional variable $a$ is interpreted as claiming that Aaron is guilty, while $b$ and $c$
      stand for the guilt of Bernard and Cain, then this statement is captured by the following formula: 
      \\[0.2cm]
      \hspace*{1.3cm} 
      $f_1 := a \vee b \vee c$.
\item \green{If Aaron is guilty, then he has exactly one accomplice.}
      
      To formalize this statement, we decompose it into two statements.
      \begin{enumerate}
      \item If Aaron is guilty, then he has at least one accomplice. \\[0.2cm]
            \hspace*{1.3cm} $f_2 := a \rightarrow b \vee c$ 
      \item If Aaron is guilty, then he has at most one accomplice. \\[0.2cm]
           \hspace*{1.3cm} $f_3 := a \rightarrow \neg (b \wedge c)$
      \end{enumerate}
\item \green{If Bernard is innocent, the Cain is innocent too.} \\[0.2cm]
      \hspace*{1.3cm} $f_4 :=  \neg b \rightarrow \neg c$ 
\item \green{If exactly two of the suspects are guilty, then Cain is one of them.}

      It is not straightforward to translate this statement into a propositional formula.
      One trick we can try is to negate the statement and then try to translate the negation.
      Now the statement given above is wrong if Cain is innocent but both Aaron and Bernard are guilty.
      Therefore we can translate this statement as follows: \\[0.2cm]
      \hspace*{1.3cm} $f_5 := \neg ( \neg c  \wedge a \wedge b )$ 
\item \green{If Cain is innocent, then Aaron is guilty.} 

      This translates into the following formula:\\[0.2cm]
      \hspace*{1.3cm} $f_6 := \neg c \rightarrow a$
\end{enumerate}
We now have a set $F = \{ f_1, f_2, f_3, f_4, f_5, f_6 \}$ of propositional formulas.
The question then is to find all propositional valuations $\mathcal{I}$ that evaluate all formulas from $F$ as
\texttt{True}.  If there is exactly one such propositional valuation, then this valuation gives us the culprits.
As it is too time consuming to try all possible valuations by hand we will write a program that performs the
required computations.
Figure \ref{fig:Usual-Suspects.ipynb} shows the program
\href{https://github.com/karlstroetmann/Logic/blob/master/Python/Chapter-4/Usual-Suspects.ipynb}{\texttt{Usual-Suspects.ipynb}}.
We discuss this program next.

\begin{figure}[!ht]
  \centering
\begin{minted}[ frame         = lines, 
                framesep      = 0.3cm, 
                numbers       = left,
                numbersep     = -0.2cm,
                bgcolor       = sepia,
                xleftmargin   = 0.8cm,
                xrightmargin  = 0.8cm
              ]{python3}
    import propLogParser as plp

    def transform(s):
        "transform the string s into a nested tuple"
        return plp.LogicParser(s).parse()
    
    P = { 'a', 'b', 'c' }
    # Aaron, Bernard, or Cain is guilty.
    f1 = 'a ∨ b ∨ c'
    # If Aaron is guilty, he has exactly one accomplice.
    f2 = 'a → b ∨ c'
    f3 = 'a → ¬(b ∧ c)'
    # If Bernard is innocent, then Cain is innocent, too.
    f4 = '¬b → ¬c'
    # If exactly two of the suspects are guilty, then Cain is one of them.
    f5 = '¬(¬c ∧ a ∧ b)'
    # If Cain is innocent, then Aaron is guilty.
    f6 = '¬c → a'
    Fs = { f1, f2, f3, f4, f5, f6 };
    Fs = { transform(f) for f in Fs }

    def allTrue(Fs, I):
        return all({evaluate(f, I) for f in Fs})

    print({ I for I in power(P) if allTrue(Fs, I) })
\end{minted}
\vspace*{-0.3cm}
  \caption{A program to investigate the burglary.}
  \label{fig:Usual-Suspects.ipynb}
\end{figure}

\begin{enumerate}
\item We input propositional formulas as strings.  However, the function \texttt{evaluate} needs nested tuples
      as input.  Therefore, we first import our parser for propositional formulas.
\item Next, we define the function \texttt{transform}.  This function takes a propositional formula that is
      represented as a string and transforms it into a nested tuple.
\item Line 7 defines the set $P$ of propositional variables.  We use the propositional variable \texttt{a} to
      express that Aaron is guilty, \texttt{b} is short for Bernard is guilty and \texttt{c} is true if and
      only if Cain is guilty. 
\item Next, we define the propositional formulas $f_1$, $\cdots$, $f_6$.
\item \texttt{Fs} is the set of all propositional formulas.
\item In line 20 these formulas are transformed into nested tuples.
\item The function $\texttt{allTrue}(Fs, I)$ takes two inputs.
      \begin{enumerate}[(a)]
      \item \texttt{Fs} is a set of propositional formulas that are represented as nested tuples.
      \item \texttt{I} is a propositional evaluation that is represented as a set of propositional variables.
            Hence \texttt{I} is a subset of \texttt{P}
      \end{enumerate}
      If all propositional formulas $f$ from the set \texttt{Fs} evaluate as \texttt{True} given the evaluation
      \texttt{I}, then \texttt{allTrue} returns the result \texttt{True}, otherwise \texttt{False} is returned.
\item Line 25 computes the set of all propositional variables that render all formulas from \texttt{Fs} true.
      The function \texttt{power} takes a set $M$ and returns the power set of $M$, i.e.~it returns the set $2^M$.
\end{enumerate}
When we run this program we see that there is just a single propositional valuation $I$ such that all formulas
from \texttt{Fs} are rendered \texttt{True} under $I$.  This propositional valuation has the form
\\[0.2cm]
\hspace*{1.3cm}
\texttt{\{'b', 'c'\}.}
\\[0.2cm]
Thus, the given problem is solvable and both Bernard and Cain are guilty, while Aaron is innocent.

\section{Tautologies}
Table \ref{tab:tautologie} on page \pageref{tab:tautologie} shows that the formula
$$  (p \rightarrow q) \rightarrow (\neg p \rightarrow q) \rightarrow q $$
is true for every propositional valuation $\mathcal{I}$.  This property gives rise to a definition.

\begin{Definition}[Tautology]
  If $f$ is a propositional formula and we have  \\[0.2cm]
  \hspace*{1.3cm} $\mathcal{I}(f) = \texttt{True}$ \quad for every propositional valuation $\mathcal{I}$, \\[0.2cm]
  then $f$ is a  \blue{tautology}.\index{tautology}  This is written as \\[0.2cm]
  \hspace*{1.3cm} $\models f$.\index{$\models f$}
  \eox
\end{Definition}

\noindent
If $f$ is a tautology, then we say that $f$ is \blue{universally valid} \index{universally valid} ist.

\noindent
\textbf{Examples}:
\begin{enumerate}
\item $\models p \vee \neg p$
\item $\models p \rightarrow p$
\item $\models p \wedge q \rightarrow p$
\item $\models p \rightarrow p \vee q$
\item $\models (p \rightarrow \falsum) \;\leftrightarrow\; \neg p$
\item $\models p \wedge q \;\leftrightarrow\; q \wedge p$
\end{enumerate}
One way to prove that a formula is universally valid is to construct a table that is analog 
to Table \ref{tab:tautologie} that is shown on page \pageref{tab:tautologie}.
Conceptually, this method is straightforward.  However, if the formula in question contains $n$
propositional variables, then the corresponding table has $2^n$ rows.  Hence for values of $n$ that are greater
than twenty this method is hopelessly inefficient.  Therefore, our goal in the rest of this chapter is to develop
a method that can often deal with hundreds of propositional variables.

The last two examples give rise to a new definition.

\begin{Definition}[Equivalent]
  Two formulas $f$ and $g$ are \blue{equivalent} \index{equivalent} if and only if  \\[0.2cm]
  \hspace*{1.3cm} $\models f \leftrightarrow g$.   
  \eox
\end{Definition}

\noindent
\textbf{Examples}:  We have the following equivalences: \\[0.3cm]
\hspace*{0.3cm} 
$\begin{array}{lll}
\models \neg \falsum \leftrightarrow \verum & \models \neg \verum \leftrightarrow \falsum &  \\[0.2cm]
 \models p \vee   \neg p \leftrightarrow \verum & \models p \wedge \neg p \leftrightarrow \falsum & \mbox{\blue{tertium-non-datur}} \\[0.2cm]
 \models p \vee   \falsum \leftrightarrow p & \models p \wedge \verum  \leftrightarrow p & \mbox{\blue{identity element}}\\[0.2cm]
 \models p \vee   \verum  \leftrightarrow \verum & \models p \wedge \falsum \leftrightarrow \falsum &  \\[0.2cm]
 \models p \wedge p \leftrightarrow p  & \models p \vee p \leftrightarrow p &  \mbox{\blue{idempotent}}\index{idempotent} \\[0.2cm]
 \models p \wedge q \leftrightarrow q \wedge p & \models p \vee   q \leftrightarrow q \vee p & \mbox{\blue{commutative}}\index{commutative} \\[0.2cm]
 \models (p \wedge q) \wedge r \leftrightarrow p \wedge (q \wedge r) & \models (p \vee   q) \vee r \leftrightarrow p \vee   (q \vee r)  &
 \mbox{\blue{Assoziativität}}\index{associative} \\[0.2cm]
 \models \neg \neg p \leftrightarrow p & & \mbox{elimination of $\neg \neg$} \\[0.2cm]
 \models p \wedge (p \vee q)   \leftrightarrow p & \models p \vee   (p \wedge q) \leftrightarrow p &  \mbox{\blue{absorption}}\index{absorption} \\[0.2cm]
 \models p \wedge (q \vee r)   \leftrightarrow (p \wedge q) \vee   (p \wedge r) & 
 \models p \vee   (q \wedge r) \leftrightarrow (p \vee q)   \wedge (p \vee   r) & \mbox{\blue{distributive}}\index{distributive} \\[0.2cm]
 \models \neg (p \wedge q) \leftrightarrow  \neg p \vee   \neg q &  \models \neg (p \vee   q) \leftrightarrow  \neg p \wedge \neg q &
 \mbox{\blue{DeMorgan'sche Regeln}}\index{DeMorgan rules}  \\[0.2cm]
 \models (p \rightarrow q) \leftrightarrow \neg p \vee q & &  \mbox{elimination of $\rightarrow$} \\[0.2cm]
 \models (p \leftrightarrow q) \leftrightarrow (\neg p \vee q) \wedge (\neg q \vee p) & & \mbox{elimination of $\leftrightarrow$}
\end{array}$ \\[0.3cm]
We can prove these equivalences using a table.  We demonstrate this method for the first of DeMorgan's rules.

\begin{table}[!ht]
  \centering
\framebox{
  \begin{tabular}{|l|l|l|l|l|l|l|}
\hline
   $p$            & $q$            &  $\neg p$      &  $\neg q$    & $p \wedge q$   & $\neg (p \wedge q)$ & $\neg p \vee \neg q$ \\
\hline
\hline
   \texttt{True}  & \texttt{True}  & \texttt{False} & \texttt{False}  & \texttt{True}  & \texttt{False}  & \texttt{False}  \\
\hline
   \texttt{True}  & \texttt{False} & \texttt{False} & \texttt{True}  & \texttt{False} & \texttt{True}    & \texttt{True}  \\
\hline
   \texttt{False} & \texttt{True}  & \texttt{True}  & \texttt{False}  & \texttt{False} & \texttt{True}     & \texttt{True} \\
\hline
   \texttt{False} & \texttt{False} & \texttt{True}  & \texttt{True} & \texttt{False} & \texttt{True}     & \texttt{True}  \\
\hline
  \end{tabular}}
  \caption{Nachweis der ersten DeMorgan'schen Regel}
  \label{tab:deMorgan}
\end{table}
We see that the last two columns in Table \ref{tab:deMorgan} on page \pageref{tab:deMorgan} have the same
entries.  Hence the formulas corresponding to these entries are equivalent.

\subsection{\textsl{Python} Implementation}
Constructing a truth table is far to tedious to do it a by hand if the formula in question has more than a few
variables.  Hence we develop a \textsl{Python} program that is able to decide whether a given propositional
formula $f$ is a tautology.  The idea is that the program evaluates $f$ for all possible propositional
interpretations.  Hence we have to compute the set of all propositional interpretations for a given set of
propositional variables.  We have already seen that the propositional interpretations are in a 1-to-1
correspondence with the subsets of the set $\mathcal{P}$ of all propositional variables because we can
represent a propositional interpretation $\mathcal{I}$ as the set of all propositional variables that evaluate
as \texttt{True}:
\\[0.2cm]
\hspace*{1.3cm}
$\bigl\{ q \in \mathcal{P} \mid \mathcal{I}(q) = \mathtt{True} \bigr\}$.
\\[0.2cm]
If we have a propositional formula $f$ and want to check whether $f$ is a tautology, we first have to determine
the set of propositional variables occurring in $f$.
To this end we define a function
\\[0.2cm]
\hspace*{1.3cm}
$\texttt{collectVars}: \mathcal{F} \rightarrow 2^{\mathcal{P}}$
\\[0.2cm]
such that $\mathtt{collectVars}(f)$ is the set of propositional variables occurring in $f$.  This function can
be defined recursively.
\begin{enumerate}
\item $\mathtt{collectVars}(p) = \{ p \}$ \quad for all propositional variables $p$.
\item $\mathtt{collectVars}(\verum) = \{\}$.
\item $\mathtt{collectVars}(\falsum) = \{\}$.
\item $\mathtt{collectVars}(\neg f) := \mathtt{collectVars}(f)$.
\item $\mathtt{collectVars}(f \wedge g) := \mathtt{collectVars}(f) \cup \mathtt{collectVars}(g)$.
\item $\mathtt{collectVars}(f \vee g) := \mathtt{collectVars}(f) \cup \mathtt{collectVars}(g)$.
\item $\mathtt{collectVars}(f \rightarrow g) := \mathtt{collectVars}(f) \cup \mathtt{collectVars}(g)$.
\item $\mathtt{collectVars}(f \leftrightarrow g) := \mathtt{collectVars}(f) \cup \mathtt{collectVars}(g)$.
\end{enumerate}
Figure \ref{fig:tautology.py-collectVars} on page \pageref{fig:tautology.py-collectVars} shows how to implement
this definition.  Note that we have been able to combine the last four cases.

\begin{figure}[!ht]
  \centering
\begin{minted}[ frame         = lines, 
                framesep      = 0.3cm, 
                numbers       = left,
                numbersep     = -0.2cm,
                bgcolor       = sepia,
                xleftmargin   = 0.0cm,
                xrightmargin  = 0.0cm
              ]{python3}
    def collectVars(f):
        "Collect all propositional variables occurring in the formula f."
        match f:
            case p if isinstance(p, str): return { p }
            case ('⊤', ):   return set()
            case ('⊥', ):   return set()
            case ('¬', g):  return collectVars(g)
            case (_, g, h): return collectVars(g) | collectVars(h) 
\end{minted}
\vspace*{-0.3cm}
  \caption{Überprüfung der Allgemeingültigkeit einer aussagenlogischen Formel}
  \label{fig:tautology.py-collectVars}
\end{figure}

Now we are able to implement the function 
\\[0.2cm]
\hspace*{1.3cm}
$\mathtt{tautology}: \mathcal{F} \rightarrow \mathbb{B}$
\\[0.2cm]
that takes a formula $f$ and returns \texttt{True} if and only if $\models f$ holds.
Figure \ref{fig:tautology.py} on page \pageref{fig:tautology.py}
shows the implementation.
\begin{enumerate}
\item First, we compute the set $P$ of all propositional variables occurring in $f$.
\item Next, the function $\mathtt{allSubsets}$ computes a list containing all subsets of the set $P$.
      Every propositional interpretation  $\mathcal{I}$ is contained in this list.
\item Then we try to find a propositional interpretation $\mathcal{I}$ such that $\mathcal{I}(f)$
      \texttt{False}.  In this case $\mathcal{I}$ is returned. 
\item Otherwise $f$ is a tautology and we return \texttt{True}.
\end{enumerate}

\begin{figure}[!ht]
  \centering
\begin{minted}[ frame         = lines, 
                framesep      = 0.3cm, 
                numbers       = left,
                numbersep     = -0.2cm,
                bgcolor       = sepia,
                xleftmargin   = 0.0cm,
                xrightmargin  = 0.0cm
              ]{python3}
    def tautology(f):
        "Check, whether the formula f is a tautology."
        P = collectVars(f)
        for I in power.allSubsets(P):
            if not evaluate(f, I):
                return I
        return True
\end{minted}
\vspace*{-0.3cm}
  \caption{Checking that $f$ is a tautology.}
  \label{fig:tautology.py}
\end{figure}

\section{Conjunctive Normal Form}
The following section discusses algebraic manipulations of propositional formulas.  Concretely, we will define
the notion of a \emph{conjunctive normal form} and show how a propositional formula can be turned into
conjunctive normal form.  The Davis-Putnam algorithm that is discussed later requires us to put the given
formulas into conjunctive normal form.

\begin{Definition}[Literal]
  A propositional formula $f$ is a \blue{Literal}\index{literal} if and only if we have one of the following cases: 
  \begin{enumerate}
  \item $f = \verum$ or $f = \falsum$.
  \item $f = p$, where $p$ is a propositional variable.

        In this case $f$ is a \blue{positive} literal.\index{positive literal}
  \item $f = \neg p$, where $p$ is a propositional variable. 

        In this case $f$ is a \blue{negative} literal.\index{negatives literal}
  \end{enumerate}
  The set of all literals is denoted as  $\blue{\mathcal{L}}$.\index{$\mathcal{L}$, set of all literals}  \eox
\end{Definition}

If  $l$ is a literal, then the \blue{complement}\index{complement} $l$ of is denoted as $\komplement{l}$.
It is defined by a case distinction.
\begin{enumerate}
\item $\komplement{\verum} = \falsum$ \quad and \quad $\komplement{\falsum} = \verum$. 
\item $\komplement{p} := \neg p$, \quad if $p \in \mathcal{P}$.
\item $\komplement{\neg p} := p$, \quad if $p \in \mathcal{P}$.
\end{enumerate}
The complement $\komplement{l}$ of a literal $l$ is equivalent to the negation of $l$, we have
\\[0.2cm]
\hspace*{1.3cm}
$\models \komplement{l} \leftrightarrow \neg l$.
\\[0.2cm]
However, the complement of a literal $l$ is also a literal, while the negation of a literal is in general not a
literal. For example, if $p$ is a propositional variable, then the complement of $\neg p$ is $p$, while the
negation of $\neg p$ is $\neg \neg p$ and this is not a literal.

\begin{Definition}[Clause]
  A propositional formula $K$ is a \blue{clause}\index{clause} when it has the form \\[0.2cm]
  \hspace*{1.3cm} $K = l_1 \vee \cdots \vee l_r$ \\[0.2cm]
  where $l_i$ is a literal for all $i=1,\cdots,r$.  Hence a clause is a disjunction of literals.
  The set of all clauses is denoted as  $\blue{\mathcal{K}}$.\index{$\mathcal{K}$ set of all clauses}
  \eox
\end{Definition}

Often, a clause is seen as a \blue{set} of its literals.  Interpreting a clause as a set of its literals
abstracts from both the order and the number of the occurrences of its literals.
This is possible because the operator ``$\vee$'' is associative, commutative, and idempotent.  
Hence the clause  $l_1 \vee \cdots \vee l_r$ will be written as the set
\\[0.2cm]
\hspace*{1.3cm} $\{ l_1, \cdots, l_r \}$.
\\[0.2cm]
This notation is called the \blue{set notation} for clauses.\index{set notation}  
The following example shows how the set notation is beneficial.  We take the clauses
\\[0.2cm]
\hspace*{1.3cm}
$p \vee (q \vee \neg r) \vee p$ \quad and \quad $\neg r \vee q \vee (\neg r \vee p)$. 
\\[0.2cm]
Although these clauses are equivalent, they are not syntactically identical.  However, if we transform these
clauses into set notation, we get
\\[0.2cm]
\hspace*{1.3cm}
$\{p, q, \neg r \}$ \quad and \quad $\{ \neg r, q, p \}$. 
\\[0.2cm]
In a set every element occurs at most once.  Furthermore, the order of the elements does not matter.
Hence the sets given above are the same!

Let us now transfer the propositional equivalence
\\[0.2cm]
\hspace*{1.3cm}
$l_1 \vee \cdots \vee l_r \vee \falsum \;\Leftrightarrow\; l_1 \vee \cdots \vee l_r$
\\[0.2cm]
into set notation.  We get
\\[0.2cm]
\hspace*{1.3cm}
$\{ l_1, \cdots, l_r, \falsum \} \;\Leftrightarrow\; \{ l_1, \cdots, l_r \}$.
\\[0.2cm]
This shows, that the element $\falsum$ can be discarded from a clause.  If we write this equivalence for $r=0$,
we get
\\[0.2cm]
\hspace*{1.3cm}
$\{\falsum \} \Leftrightarrow \{\}$.
\\[0.2cm]
Hence the empty set of literals is to be interpreted as $\falsum$.

\begin{Definition}
  A clause $K$ is \blue{trivial},\index{trivial} if and only if one of the following cases occurs:
  \begin{enumerate}
  \item $\verum \in K$.
  \item There is a variable  $p \in \mathcal{P}$, such that we have both $p \in K$ as well as $\neg p \in K$.

        In this case $p$ and $\neg p$ are called  \blue{complementary literals}.\index{complementary literals}
        \eox
\end{enumerate}
\end{Definition}

\begin{Satz} \label{satz:trivial}
  Eine Klausel $K$ ist genau dann eine Tautologie, wenn sie trivial ist.
\end{Satz}
\textbf{Beweis}:  Wir nehmen zunächst an, dass die Klausel $K$ trivial ist.
Falls nun $\verum \el K$ ist, dann gilt wegen der Gültigkeit der Äquivalenz 
$f \vee \verum \leftrightarrow \verum$
offenbar $K \leftrightarrow \verum$.   Ist $p$ eine Aussage-Variable, so dass
sowohl $p \el K$ als auch $\neg p \el K$ gilt, dann folgt aufgrund der Äquivalenz $p \vee
\neg p \leftrightarrow \verum$ sofort $K \leftrightarrow \verum$.

Wir nehmen nun an, dass die Klausel $K$ eine Tautologie ist.  Wir führen den Beweis
indirekt und nehmen an, dass $K$ nicht trivial ist.  Damit gilt  $\verum \notin K$ und
$K$ kann auch keine komplementären Literale enthalten.  Damit hat $K$ dann die Form
\\[0.2cm]
\hspace*{1.3cm} 
$K = \{ \neg p_1, \cdots, \neg p_m, q_1, \cdots, q_n \}$ \quad mit $p_i
\not= q_j$ für alle $i \in \{ 1,\cdots,m\}$ und $j \in \{1, \cdots, n\}$.
\\[0.2cm]
Dann könnten wir eine Interpretation $\mathcal{I}$ wie folgt definieren:
\begin{enumerate}
\item $\mathcal{I}(p_i) = \texttt{True}$ für alle $i = 1, \cdots, m$ und
\item $\mathcal{I}(q_j) = \texttt{False}$ für alle $j = 1, \cdots, n$,
\end{enumerate}
Mit dieser Interpretation würde offenbar $\mathcal{I}(K) = \texttt{False}$ gelten und damit könnte $K$ keine
Tautologie sein.  Also ist die Annahme, dass $K$ nicht trivial ist, falsch.
\hspace*{\fill}  $\Box$

\begin{Definition}[Konjunktive Normalform]  
  Eine Formel $F$ ist in \blue{konjunktiver Normalform}\index{konjunktive Normalform} (kurz KNF)\index{KNF}
  genau dann, wenn $F$ eine Konjunktion von Klauseln ist, wenn also gilt \\[0.2cm]
  \hspace*{1.3cm} $F = K_1 \wedge \cdots \wedge K_n$, \\[0.2cm]
  wobei die $K_i$ für alle $i=1,\cdots,n$ Klauseln sind. \eox
\end{Definition}

\noindent
Aus der Definition der KNF folgt sofort:
\begin{Korollar} \label{korollar:knf}
  Ist $F = K_1 \wedge \cdots \wedge K_n$ in konjunktiver Normalform, so gilt\\[0.2cm]
  \hspace*{1.3cm} $\models F$ \quad genau dann, wenn \quad $\models K_i$ \quad für alle $i=1,\cdots,n$. \qed
\end{Korollar}

Damit können wir für eine Formel $F = K_1 \wedge \cdots \wedge K_n$ in konjunktiver
Normalform leicht entscheiden, ob $F$ eine Tautologie ist, denn $F$ ist genau dann eine
Tautologie, wenn alle Klauseln $K_i$ trivial sind.
\vspace*{0.2cm}

Da für die Konjunktion analog zur Disjunktion das Assoziativ-, Kommutativ- und Idempotenz-Gesetz
gilt, ist es zweckmäßig, auch für Formeln in konjunktiver Normalform wie folgt eine
\blue{Mengen-Schreibweise}\index{Mengen-Schreibweise für Formeln in KNF} einzuführen:  Ist die Formel
\\[0.2cm]
\hspace*{1.3cm}
$F = K_1 \wedge \cdots \wedge K_n$
\\[0.2cm]
in konjunktiver Normalform, so repräsentieren wir diese
Formel  durch die Menge ihrer Klauseln und schreiben \\[0.2cm]
\hspace*{1.3cm} 
$F = \{ K_1, \cdots, K_n \}$. 
\\[0.2cm]
Hierbei werden auch die Klauseln ihrerseits in Mengen-Schreibweise angegeben.
Wir geben ein Beispiel:  Sind $p$, $q$ und $r$ Aussage-Variablen, so ist die Formel
\\[0.2cm]
\hspace*{1.3cm}
$(p \vee q \vee \neg r) \wedge (q \vee \neg r \vee p \vee q)\wedge (\neg r \vee p \vee \neg q)$
\\[0.2cm]
in konjunktiver Normalform.  In Mengen-Schreibweise wird daraus
\\[0.2cm]
\hspace*{1.3cm}
$\bigl\{ \{p, q, \neg r \},\, \{ p, \neg q, \neg r \} \bigr\}$.
\\[0.2cm]
Wir stellen nun ein Verfahren vor, mit dem sich jede Formel $F$ in KNF transformieren lässt.  Nach
dem oben Gesagten können wir dann leicht entscheiden, ob $F$ eine Tautologie ist.
\begin{enumerate}
\item Eliminiere alle Vorkommen des Junktors ``$\leftrightarrow$'' mit Hilfe der Äquivalenz \\[0.2cm]
      \hspace*{1.3cm} 
      $(F \leftrightarrow G) \leftrightarrow (F \rightarrow G) \wedge (G \rightarrow F)$.
\item Eliminiere alle Vorkommen des Junktors ``$\rightarrow$'' mit Hilfe der Äquivalenz \\[0.2cm]
      \hspace*{1.3cm} 
      $(F \rightarrow G) \leftrightarrow \neg F \vee G$.
\item Schiebe die Negationszeichen soweit es geht nach innen.  Verwende dazu die folgenden Äquivalenzen:
      \begin{enumerate}
      \item $\neg \falsum \leftrightarrow \verum$
      \item $\neg \verum \leftrightarrow \falsum$
      \item $\neg \neg F \leftrightarrow F$
      \item $\neg (F \wedge G) \leftrightarrow  \neg F \vee \neg G$ 
      \item $\neg (F \vee   G) \leftrightarrow  \neg F \wedge \neg G$ 
      \end{enumerate}
      In dem Ergebnis, das wir nach diesem Schritt erhalten, stehen die Negationszeichen
      nur noch unmittelbar vor den aussagenlogischen Variablen.  Formeln mit dieser
      Eigenschaft bezeichnen wir auch als Formeln in \blue{Negations-Normalform}.\index{Negations-Normalform}
\item Stehen in der Formel jetzt ``$\vee$''-Junktoren über ``$\wedge$''-Junktoren, so können wir durch
      \blue{Ausmultiplizieren},\index{Ausmultiplizieren} sprich Verwendung des Distributiv-Gesetzes \\[0.2cm]
      \hspace*{1.3cm} 
      $
      \begin{array}{cl}
                      & (F_1 \wedge \cdots \wedge F_m) \vee (G_1 \wedge \cdots \wedge G_n) \\[0.2cm]
      \leftrightarrow & (F_1 \vee G_1) \wedge \cdots \wedge (F_1 \vee G_n) \;\wedge \;\cdots\; \wedge\;
                        (F_m \vee G_1) \wedge \cdots \wedge (F_m \vee G_n)
      \end{array}
      $
      \\[0.2cm]
      den Junktor ``$\vee$'' nach innen schieben.
\item In einem letzten Schritt überführen wir die Formel nun in Mengen-Schreibweise, indem
      wir zunächst die Disjunktionen aller Literale als Mengen zusammenfassen und anschließend
      alle so entstandenen Klauseln wieder in einer Menge zusammen fassen.
\end{enumerate}
Hier sollten wir noch bemerken, dass die Formel beim Ausmultiplizieren stark anwachsen kann.
Das liegt daran, dass die Formel $F$ auf der rechten Seite der Äquivalenz 
$F \vee (G \wedge H) \leftrightarrow (F \vee G) \wedge (F \vee H)$ zweimal auftritt, während sie
links nur einmal vorkommt. 

Wir demonstrieren das Verfahren am Beispiel der Formel\\[0.2cm]
\hspace*{1.3cm} $(p \rightarrow q) \rightarrow (\neg p \rightarrow \neg q)$.
\begin{enumerate}
\item Da die Formel den Junktor ``$\leftrightarrow$'' nicht enthält,
      ist im ersten Schritt nichts zu tun.
\item Die Elimination des Junktors ``$\rightarrow$'' liefert \\[0.2cm]
      \hspace*{1.3cm} $\neg (\neg p \vee q) \vee (\neg \neg p \vee \neg q)$.
\item Die Umrechnung auf Negations-Normalform ergibt \\[0.2cm]
      \hspace*{1.3cm} $(p \wedge \neg q) \vee (p \vee \neg q)$.
\item Durch ``Ausmultiplizieren'' erhalten wir \\[0.2cm]
      \hspace*{1.3cm} $\bigl(p \vee (p \vee \neg q)\bigr) \wedge \bigl(\neg q \vee (p \vee \neg q)\bigr)$.
\item Die Überführung in die Mengen-Schreibweise ergibt zunächst als Klauseln die beiden Mengen \\[0.2cm]
      \hspace*{1.3cm} $\{p, p, \neg q\}$ \quad und \quad $\{\neg q,  p,  \neg q\}$. \\[0.2cm]
      Da die Reihenfolge der Elemente einer Menge aber unwichtig ist und außerdem eine Menge
      jedes Element nur einmal enthält, stellen wir fest, dass diese beiden Klauseln gleich sind.
      Fassen wir jetzt die Klauseln noch in einer Menge zusammen, so erhalten wir \\[0.2cm]
      \hspace*{1.3cm} $\bigl\{ \{p, \neg q\} \bigr\}$. \\[0.2cm]
      Beachten Sie, dass sich die Formel durch die Überführung in 
      Mengen-Schreibweise noch einmal deutlich vereinfacht hat.
\end{enumerate}
Damit ist die Formel in KNF überführt.

\subsection{Berechnung der konjunktiven Normalform in \textsl{Python}}
Wir geben nun eine Reihe von Funktionen an, mit deren Hilfe sich eine gegebene
Formel $f$ in konjunktive Normalform überführen lässt.  Diese Funktionen sind Teil des Jupyter Notebooks
\href{https://github.com/karlstroetmann/Logic/blob/master/Python/CNF.ipynb}{CNF.ipynb}.
Wir beginnen mit der
Funktion 
\\[0.2cm]
\hspace*{1.3cm}
$\texttt{elimBiconditional}: \mathcal{F} \rightarrow \mathcal{F}$
\\[0.2cm]
welche die Aufgabe hat, eine vorgegebene aussagenlogische Formel $f$ in eine äquivalente Formel
umzuformen, die den Junktor ``$\leftrightarrow$'' nicht mehr enthält.  Die Funktion
$\texttt{elimBiconditional}(f)$ wird durch Induktion über den Aufbau der aussagenlogischen Formel $f$ definiert.
Dazu stellen wir zunächst rekursive Gleichungen auf,
die das Verhalten der Funktion $\texttt{elimBiconditional}$ beschreiben:
\begin{enumerate}
\item Wenn $f$ eine
      Aussage-Variable $p$ ist, so ist nichts zu tun:
      \\[0.2cm]
      \hspace*{1.3cm}
      $\texttt{elimBiconditional}(p) = p$ \quad für alle $p \in \mathcal{P}$.
\item Die Fälle, in denen $f$ gleich dem Verum oder dem Falsum ist, sind ebenfalls trivial:
      \\[0.2cm]
      \hspace*{1.3cm}
      $\texttt{elimBiconditional}(\verum) = \verum$ \quad und \quad
      $\texttt{elimBiconditional}(\falsum) = \falsum$.
\item Hat $f$ die Form $f = \neg g$, so eliminieren wir den Junktor
      ``$\leftrightarrow$'' rekursiv aus der Formel $g$ und negieren die resultierende Formel: \\[0.2cm]
      \hspace*{1.3cm} 
      $\texttt{elimBiconditional}(\neg g) = \neg \texttt{elimBiconditional}(g)$.
\item In den Fällen $f = g_1 \wedge g_2$,  $f = g_1 \vee g_2$ und $f = g_1 \rightarrow g_2$ eliminieren wir
      rekursiv den Junktor ``$\leftrightarrow$'' aus den Formeln $g_1$ und $g_2$ und setzen dann die Formel
      wieder zusammen: 
      \begin{enumerate}
      \item $\texttt{elimBiconditional}(g_1 \wedge g_2) = \texttt{elimBiconditional}(g_1) \wedge \texttt{elimBiconditional}(g_2)$.
      \item $\texttt{elimBiconditional}(g_1 \vee g_2) = \texttt{elimBiconditional}(g_1) \vee \texttt{elimBiconditional}(g_2)$.
      \item $\texttt{elimBiconditional}(g_1 \rightarrow g_2) = \texttt{elimBiconditional}(g_1) \rightarrow \texttt{elimBiconditional}(g_2)$.
      \end{enumerate}
\item Hat $f$ die Form $f = g_1 \leftrightarrow g_2$, so benutzen wir die
      Äquivalenz \\[0.2cm]
      \hspace*{1.3cm} 
      $(g_1 \leftrightarrow g_2) \leftrightarrow \bigl( (g_1 \rightarrow g_2) \wedge (g_2 \rightarrow g_1)\bigr)$.
      \\[0.2cm]
      Das führt auf die Gleichung:
      \\[0.2cm]
      \hspace*{1.3cm} 
      $\texttt{elimBiconditional}(g_1 \leftrightarrow g_2) = \texttt{elimBiconditional}\bigl( (g_1 \rightarrow g_2) \wedge (g_2 \rightarrow g_1)\bigr)$. 
      \\[0.2cm]
      Der Aufruf der Funktion \texttt{elimBiconditional} auf der rechten Seite der Gleichung ist notwendig,
      denn der Junktor ``$\leftrightarrow$'' kann ja noch in $g_1$ und $g_2$ auftreten.
\end{enumerate}


\begin{figure}[!ht]
  \centering
\begin{minted}[ frame         = lines, 
                framesep      = 0.3cm, 
                numbers       = left,
                numbersep     = -0.2cm,
                bgcolor       = sepia,
                xleftmargin   = 0.0cm,
                xrightmargin  = 0.0cm
              ]{python3}
    def elimBiconditional(f):
        "Eliminate the logical operator '↔' from the formula f."
        if isinstance(f, str):   # This case covers variables.
            return f
        if f[0] == '↔':
            g, h = f[1:]
            ge   = elimBiconditional(g)
            he   = elimBiconditional(h)
            return ('∧', ('→', ge, he), ('→', he, ge))
        if f[0] == '⊤' or f[0] == '⊥':
            return f
        if f[0] == '¬':
            g  = f[1]
            ge = elimBiconditional(g)
            return ('¬', ge)
        else:
            op, g, h = f
            ge       = elimBiconditional(g)
            he       = elimBiconditional(h)
            return (op, ge, he)
\end{minted}
\vspace*{-0.3cm}
  \caption{Elimination von $\leftrightarrow$}
  \label{fig:elimBiconditional}
\end{figure} 
Abbildung
\ref{fig:elimBiconditional} auf Seite \pageref{fig:elimBiconditional} zeigt die Implementierung der
Funktion 
\href{https://github.com/karlstroetmann/Logic/blob/master/Python/CNF.ipynb}{\texttt{elimBiconditional}}.
\begin{enumerate}
\item In Zeile 3 prüft der Funktions-Aufruf $\texttt{isinstance}(f, \mathtt{str})$, ob $f$ ein String ist.  In diesem Fall
      muss $f$ eine aussagenlogische Variable sein, denn alle anderen aussagenlogischen Formeln werden als
      geschachtelte Listen dargestellt.  Daher wird $f$ in diesem Fall unverändert zurück gegeben.
\item In Zeile 5 überprüfen wir den Fall, dass $f$ die Form $g \leftrightarrow h$ hat.
      In diesem Fall eliminieren wir den Junktor ``$\leftrightarrow$'' aus $g$ und $h$ durch einen rekursiven
      Aufruf der Funktion \texttt{elimBiconditional}.  Anschließend benutzen wir die Äquivalenz
      \\[0.2cm]
      \hspace*{1.3cm}
      $(g \leftrightarrow h) \leftrightarrow (g \rightarrow h) \wedge (h \rightarrow g)$.
\item In Zeile 10 behandeln wir die Fälle, dass $f$ gleich dem Verum oder dem Falsum ist.
      Hier ist zu beachten, dass diese Formeln ebenfalls als geschachtelte Tupel dargestellt werden,
      Verum wird beispielsweise als das Tuple \texttt{('⊤',)} dargestellt, während Falsum von uns in
      \textsl{Python} durch das Tuple \texttt{('⊥',)} repräsentiert wird.  In diesem Fall wird $f$
      unverändert zurück gegeben.
\item In Zeile 14 betrachten wir den Fall, dass $f$ eine Negation ist.  Dann hat $f$ die Form
      \\[0.2cm]
      \hspace*{1.3cm}
      \texttt{('¬', $g$)}  
      \\[0.2cm]
      und wir müssen den Junktor ``$\leftrightarrow$'' rekursiv aus $g$ entfernen.
\item In den jetzt noch verbleibenden Fällen hat $f$ die Form
      \\[0.2cm]
      \hspace*{1.3cm}
      \texttt{('$o$', $g$, $h$)}  \quad mit $o \in \{\rightarrow, \wedge, \vee\}$.
      \\[0.2cm]
      In diesen Fällen muss der Junktor ``$\leftrightarrow$'' rekursiv aus den Teilformeln $g$ und $h$ entfernt
      werden. 
\end{enumerate}
\FloatBarrier

Als nächstes betrachten wir die Funktion zur Elimination des Junktors ``$\rightarrow$''. 
Abbildung
\ref{fig:eliminate-folgt} auf Seite \pageref{fig:eliminate-folgt} zeigt die
Implementierung der Funktion
\href{https://github.com/karlstroetmann/Logic/blob/master/Python/CNF.ipynb}{\texttt{elimFolgt}}.
Die der Implementierung zu Grunde liegende Idee ist dieselbe wie bei der Elimination des
Junktors ``$\leftrightarrow$''.  Der einzige Unterschied besteht darin, dass wir jetzt die
Äquivalenz \\[0.2cm]
\hspace*{1.3cm}
$(g \rightarrow h) \leftrightarrow (\neg g \vee h)$ \\[0.2cm]
benutzen.  Außerdem können wir bei der Implementierung dieser Funktion voraussetzen, dass der Junktor ``$\leftrightarrow$''
bereits aus der aussagenlogischen Formel \texttt{F}, die als Argument übergeben wird, eliminiert worden ist.
Dadurch entfällt bei der Implementierung ein Fall. 

\begin{figure}[!ht]
  \centering
\begin{minted}[ frame         = lines, 
                  framesep      = 0.3cm, 
                  numbers       = left,
                  numbersep     = -0.2cm,
                  bgcolor       = sepia,
                  xleftmargin   = 0.0cm,
                  xrightmargin  = 0.0cm
                ]{python3}
    def elimConditional(f):
        "Eliminate the logical operator '→' from f."
        if isinstance(f, str): 
            return f
        if f[0] == '⊤' or f[0] == '⊥':
            return f
        if f[0] == '→':
            g, h = f[1:]
            ge   = elimConditional(g)
            he   = elimConditional(h)
            return ('∨', ('¬', ge), he)
        if f[0] == '¬':
            g  = f[1]
            ge = elimConditional(g)
            return ('¬', ge)
        else:
            op, g, h = f
            ge       = elimConditional(g)
            he       = elimConditional(h)
            return (op, ge, he)
\end{minted}
\vspace*{-0.3cm}
  \caption{Elimination von $\rightarrow$}
  \label{fig:eliminate-folgt}
\end{figure}
 
Als nächstes zeigen wir die Funktionen zur Berechnung der Negations-Normalform.
Abbildung
\ref{fig:nnf} auf Seite \pageref{fig:nnf} zeigt die Implementierung der Funktionen
\href{https://github.com/karlstroetmann/Logic/blob/master/Python/CNF.ipynb}{\texttt{nnf}} und
\href{https://github.com/karlstroetmann/Logic/blob/master/Python/CNF.ipynb}{\texttt{neg}},
die sich wechselseitig aufrufen.  Dabei berechnet \texttt{nnf($f$)} die Negations-Normalform von $f$, während  
\texttt{neg($f$)} die Negations-Normalform von $\neg f$ berechnet,  es gilt also
\\[0.2cm]
\hspace*{1.3cm}
$\texttt{neg}(\texttt{F}) = \texttt{nnf}(\neg \texttt{f})$.
\\[0.2cm]
 Die eigentliche Arbeit wird dabei in der
Funktion \texttt{neg} erledigt, denn dort werden die beiden DeMorgan'schen Gesetze 
\\[0.2cm]
\hspace*{1.3cm}
$\neg (f \wedge g) \leftrightarrow (\neg f \vee \neg g)$ \quad und \quad 
$\neg (f \vee g) \leftrightarrow (\neg f \wedge \neg g)$ 
\\[0.2cm]
angewendet.  Wir beschreiben die Umformung in Negations-Normalform durch 
die folgenden Glei\-chungen: \index{\texttt{nnf}}
\begin{enumerate}
\item $\texttt{nnf}(p) = p$ \quad für alle $p \in \mathcal{P}$,
\item $\texttt{nnf}(\verum) = \verum$,
\item $\texttt{nnf}(\falsum) = \falsum$,
\item $\texttt{nnf}(\neg \texttt{f}) = \texttt{neg}(\texttt{f})$,
\item $\texttt{nnf}(f_1 \wedge f_2) = \texttt{nnf}(f_1) \wedge \texttt{nnf}(f_2)$,
\item $\texttt{nnf}(f_1 \vee f_2) = \texttt{nnf}(f_1) \vee \texttt{nnf}(f_2)$.
\end{enumerate}

\begin{figure}[!ht]
  \centering
\begin{minted}[ frame         = lines, 
                  framesep      = 0.3cm, 
                  xleftmargin   = 0.0cm,
                  xrightmargin  = 0.0cm,
                  bgcolor       = sepia,
                  numbers       = left,
                  numbersep     = -0.2cm,
                ]{python3}
    def nnf(f):
        "Compute the negation normal form of f."
        if isinstance(f, str): 
            return f
        if f[0] == '⊤' or f[0] == '⊥':
            return f
        if f[0] == '¬':
            g = f[1]
            return neg(g)
        if f[0] == '∧':
            g, h = f[1:]
            return ('∧', nnf(g), nnf(h))
        if f[0] == '∨':
            g, h = f[1:]
            return ('∨', nnf(g), nnf(h))

\end{minted}
\vspace*{-0.3cm}
  \caption{Berechnung der Negations-Normalform: Die Funktion \texttt{nnf}.}
  \label{fig:nnf}
\end{figure}

Die Hilfsprozedur \texttt{neg}, die die Negations-Normalform von $\neg f$ berechnet,
spezifizieren wir ebenfalls durch rekursive Gleichungen: \index{\texttt{neg}}
\begin{enumerate}
\item $\texttt{neg}(p) = \texttt{nnf}(\neg p) = \neg p$ für alle Aussage-Variablen $p$.
\item $\texttt{neg}(\verum) = \texttt{nnf}(\neg\verum) = \texttt{nnf}(\falsum) = \falsum$, 
\item $\texttt{neg}(\falsum) = \texttt{nnf}(\neg\falsum) = \texttt{nnf}(\verum) = \verum$,
\item $\texttt{neg}(\neg f) = \texttt{nnf}(\neg \neg f) = \texttt{nnf}(f)$.
\item $\begin{array}[t]{cl}
         & \texttt{neg}\bigl(f_1 \wedge f_2 \bigr) \\[0.1cm]
       = & \texttt{nnf}\bigl(\neg(f_1 \wedge f_2)\bigr) \\[0.1cm]
       = & \texttt{nnf}\bigl(\neg f_1 \vee \neg f_2\bigr) \\[0.1cm]
       = & \texttt{nnf}\bigl(\neg f_1\bigr) \vee \texttt{nnf}\bigl(\neg f_2\bigr) \\[0.1cm]
       = & \texttt{neg}(f_1) \vee \texttt{neg}(f_2).
       \end{array}
      $

      Also haben wir:
      \\[0.2cm]
      \hspace*{1.3cm}
      $\texttt{neg}\bigl(f_1 \wedge f_2 \bigr) = \texttt{neg}(f_1) \vee \texttt{neg}(f_2)$.
\item $\begin{array}[t]{cl}
         & \texttt{neg}\bigl(f_1 \vee f_2 \bigr)        \\[0.1cm]
       = & \texttt{nnf}\bigl(\neg(f_1 \vee f_2) \bigr)  \\[0.1cm]
       = & \texttt{nnf}\bigl(\neg f_1 \wedge \neg f_2 \bigr)  \\[0.1cm]
       = & \texttt{nnf}\bigl(\neg f_1\bigr) \wedge \texttt{nnf}\bigl(\neg f_2 \bigr)  \\[0.1cm]
       = & \texttt{neg}(f_1) \wedge \texttt{neg}(f_2). 
       \end{array}
      $

      Also haben wir: 
      \\[0.2cm]
      \hspace*{1.3cm}
      $\texttt{neg}\bigl(f_1 \vee f_2 \bigr) = \texttt{neg}(f_1) \wedge \texttt{neg}(f_2)$.
\end{enumerate}
Die in den Abbildungen \ref{fig:nnf} und \ref{fig:neg} auf Seite \pageref{fig:nnf} und Seite \pageref{fig:neg} gezeigten Funktionen setzen die oben diskutierten
Gleichungen unmittelbar um. 

\begin{figure}[!ht]
  \centering
\begin{minted}[ frame         = lines, 
                  framesep      = 0.3cm, 
                  xleftmargin   = 0.0cm,
                  xrightmargin  = 0.0cm,
                  bgcolor       = sepia,
                  numbers       = left,
                  numbersep     = -0.2cm,
                ]{python3}
    def neg(f):
        "Compute the negation normal form of ¬f."
        if isinstance(f, str): 
            return ('¬', f)
        if f[0] == '⊤':
            return ('⊥',)
        if f[0] == '⊥':
            return ('⊤',)
        if f[0] == '¬':
            g = f[1]
            return nnf(g)
        if f[0] == '∧':
            g, h = f[1:]
            return ('∨', neg(g), neg(h))
        if f[0] == '∨':
            g, h = f[1:]
            return ('∧', neg(g), neg(h))
\end{minted}
\vspace*{-0.3cm}
  \caption{Berechnung der Negations-Normalform: Die Funktion \texttt{neg}.}
  \label{fig:neg}
\end{figure}


Als letztes stellen wir die Funktionen vor, mit denen die Formeln, die bereits in
Negations-Normalform sind, ausmultipliziert und dadurch in konjunktive
Normalform gebracht werden.  Gleichzeitig werden  die zu normalisierenden Formeln dabei
in die Mengen-Schreibweise transformiert, d.h.~die Formeln werden als Mengen von Mengen 
von Literalen dargestellt.  Dabei interpretieren wir eine Menge von Literalen als
Disjunktion der Literale und eine Menge von Klauseln interpretieren wir als Konjunktion
der Klauseln.  Mathematisch ist unser Ziel also, eine Funktion
\\[0.2cm]
\hspace*{1.3cm}
$\texttt{cnf}: \texttt{NNF} \rightarrow \texttt{KNF}$
\\[0.2cm]
zu definieren, so dass $\texttt{cnf}(f)$ für eine Formel $f$, die in Negations-Normalform vorliegt, eine Menge von Klauseln als
Ergebnis zurück gibt, deren Konjunktion zu $f$ äquivalent ist.  Die Definition von $\texttt{cnf}(f)$ erfolgt
rekursiv.
\begin{enumerate}
\item Falls $f$ eine aussagenlogische Variable ist, geben wir als Ergebnis eine Menge zurück, die genau eine
      Klausel enthält.  Diese Klausel ist selbst wieder eine Menge von Literalen, die als einziges Literal die
      aussagenlogische Variable $f$ enthält:
      \\[0.2cm]
      \hspace*{1.3cm}
      $\texttt{cnf}(f) := \bigl\{\{f\}\bigr\}$ \quad falls $f \in \mathcal{P}$.
\item Wir hatten früher gesehen, dass die leere Menge von \emph{Klauseln} als $\verum$ interpretiert werden kann.
      Daher gilt:
      \\[0.2cm]
      \hspace*{1.3cm}
      $\texttt{cnf}(\verum) := \bigl\{\bigr\}$.      
\item Wir hatten ebenfalls gesehen, dass die leere Menge von \emph{Literalen} als $\falsum$ interpretiert werden kann.
      Daher gilt:
      \\[0.2cm]
      \hspace*{1.3cm}
      $\texttt{cnf}(\falsum) := \bigl\{\{\}\bigr\}$.      
\item Falls $f$ eine Negation ist, dann muss gelten
      \\[0.2cm]
      \hspace*{1.3cm}
      $f = \neg p$ \quad mit $p \in \mathcal{P}$,
      \\[0.2cm]
      denn $f$ ist ja in Negations-Normalform und in einer solchen Formel kann der Negations-Operator nur auf
      eine aussagenlogische Variable angewendet werden.  Daher ist $f$ ein Literal und wir geben als Ergebnis
      eine Menge zurück, die genau eine Klausel enthält.  Diese Klausel ist selbst wieder eine Menge von
      Literalen, die als einziges Literal die Formel $f$ enthält:
      \\[0.2cm]
      \hspace*{1.3cm}
      $\texttt{cnf}(\neg p) := \bigl\{\{\neg p\}\bigr\}$ \quad falls $p \in \mathcal{P}$.
\item Falls $f$ eine Konjunktion ist und also $f = g \wedge h$ gilt,  dann können wir die 
      zunächst die Formeln $g$ und $h$ in KNF transformieren.  Dabei erhalten wir dann Mengen von Klauseln 
      $\texttt{cnf}(g)$ und $\texttt{cnf}(h)$.  Da wir eine Menge von Klauseln als Konjunktion der in der Menge
      enthaltenen Klauseln interpretieren, reicht es aus,  
      die Vereinigung der Mengen $\texttt{cnf}(f)$ und $\texttt{cnf}(g)$ zu bilden,
      wir haben also \\[0.2cm]
      \hspace*{1.3cm} $\texttt{cnf}(g \wedge h) = \texttt{cnf}(g) \cup  \texttt{cnf}(h)$.
\item Falls $f = g \vee h$ ist, transformieren wir zunächst $g$ und $h$ in KNF.
      Dabei erhalten wir \\[0.2cm]
      \hspace*{1.3cm} 
      $\texttt{cnf}(g) = \{ g_1, \cdots, g_m \}$ \quad und \quad
      $\texttt{cnf}(h) = \{ h_1, \cdots, h_n \}$. \\[0.2cm]
      Dabei sind die $g_i$ und die $h_j$ Klauseln.  Um nun die KNF von $g \vee h$ zu
      bilden, rechnen wir wie folgt: 
      $$
      \begin{array}[c]{ll}
        & g \vee h  \\[0.2cm]
      \Leftrightarrow & (k_1 \wedge \cdots \wedge k_m) \vee (l_1 \wedge \cdots \wedge l_n) \\[0.2cm]
      \Leftrightarrow & (k_1 \vee l_1) \quad \wedge \quad \cdots \quad \wedge \quad (k_m \vee l_1) \quad \wedge \\ 
                      & \qquad \vdots     \hspace*{4cm} \vdots                \\
                      & (k_1 \vee l_n) \quad \wedge \quad \cdots \quad \wedge \quad (k_m \vee l_n) \\[0.2cm] 
      \Leftrightarrow & \bigl\{ k_i \vee l_j : i \in \{ 1, \cdots, m\}, j \in \{ 1, \cdots, n \} \bigr\} \\ 
      \end{array}
      $$
      Berücksichtigen wir noch, dass Klauseln in der Mengen-Schreibweise als Mengen von
      Literalen aufgefasst werden, die implizit disjunktiv verknüpft werden, so können wir
      für $k_i \vee l_j$ auch $k_i \cup l_j$ schreiben.  Wir sehen, dass jede Klausel aus $\texttt{cnf}(g)$
      mit jeder Klausel aus $\texttt{cnf}(g)$ vereinigt wird.  Insgesamt erhalten wir damit 
      \\[0.2cm]
      \hspace*{1.3cm} 
      $\texttt{cnf}(g \vee h) = \bigl\{ k \cup l \mid k \in \texttt{cnf}(g) \;\wedge\; l \in \texttt{cnf}(h) \bigr\}$.
\end{enumerate}
Abbildung \ref{fig:cnf} auf Seite \pageref{fig:cnf} zeigt die Implementierung der Funktion
\href{https://github.com/karlstroetmann/Logic/blob/master/Python/CNF.ipynb}{\texttt{cnf}}.
(Der Name \texttt{cnf} ist die Abkürzung von \blue{\underline{c}onjunctive \underline{n}ormal \underline{f}orm}.)

\begin{figure}[!ht]
  \centering
\begin{minted}[ frame         = lines, 
                  framesep      = 0.3cm, 
                  numbers       = left,
                  numbersep     = -0.2cm,
                  bgcolor       = sepia,
                  xleftmargin   = 0.0cm,
                  xrightmargin  = 0.0cm
                ]{python3}
    def cnf(f):
        if isinstance(f, str): 
            return { frozenset({f}) }
        if f[0] == '⊤':
            return set()
        if f[0] == '⊥':
            return { frozenset() }
        if f[0] == '¬':
            return { frozenset({f}) }
        if f[0] == '∧':
            g, h = f[1:]
            return cnf(g) | cnf(h)
        if f[0] == '∨':
            g, h = f[1:]
            return { k1 | k2 for k1 in cnf(g) for k2 in cnf(h) }
\end{minted}
\vspace*{-0.3cm}
  \caption{Berechnung der konjunktiven Normalform. \index{\texttt{cnf}}}
  \label{fig:cnf}
\end{figure}

Zum Abschluss zeigen wir in Abbildung \ref{fig:normalize} auf Seite \pageref{fig:normalize}
wie die einzelnen Funktionen zusammenspielen.
\begin{enumerate}
\item Die Funktion \texttt{normalize} eliminiert zunächst die Junktoren ``$\leftrightarrow$''
      mit Hilfe der Funktion \texttt{elimBiconditional}.
\item Anschließend wird der Junktor  ``$\rightarrow$'' mit Hilfe der Funktion \texttt{elimConditional}
      ersetzt.      
\item Der Aufruf von \texttt{nnf} bringt die Formel in Negations-Normalform.
\item Die Negations-Normalform wird nun mit Hilfe der Funktion \texttt{cnf} in konjunktive Normalform gebracht,
      wobei gleichzeitig die Formel in Mengen-Schreibweise überführt wird.  
\item Schließlich entfernt die Funktion \texttt{simplify} alle Klauseln aus der Menge \texttt{N4}, die trivial
      sind.
\item Die Funktion \texttt{isTrivial} überprüft, ob eine Klausel $C$, die in Mengen-Schreibweise vorliegt,
      sowohl eine Variable $p$ als auch die Negation $\neg p$ dieser Variablen enthält, denn dann ist diese
      Klausel zu $\verum$ äquivalent und kann weggelassen werden.
\end{enumerate}
Das vollständige Programm zur Berechnung der konjunktiven Normalform finden Sie als die Datei
\href{https://github.com/karlstroetmann/Logic/blob/master/Python/CNF.ipynb}{\texttt{CNF.ipynb}} 
unter GitHub.

\begin{figure}[!ht]
  \centering
\begin{minted}[ frame         = lines, 
                framesep      = 0.3cm, 
                numbers       = left,
                numbersep     = -0.2cm,
                bgcolor       = sepia,
                xleftmargin   = 0.0cm,
                xrightmargin  = 0.0cm
              ]{python3}
    def normalize (f):
        n1 = elimBiconditional(f)
        n2 = elimConditional(n1)
        n3 = nnf(n2)
        n4 = cnf(n3)
        return simplify(n4)
    
    def simplify(Clauses):
        return { C for C in Clauses if not isTrivial(C) }
    
    def isTrivial(Clause):
        return any(('¬', p) in Clause for p in Clause)
\end{minted} 
\vspace*{-0.3cm}
  \caption{Normalisierung einer Formel}
  \label{fig:normalize}
\end{figure}

\exercise
Berechnen Sie die konjunktiven Normalformen der folgenden aussagenlogischen Formeln und geben Sie Ihr Ergebnis
in Mengenschreibweise an.  Überprüfen Sie Ihr Ergebnis mit Hilfe des Jupyter-Notebooks 
\href{https://github.com/karlstroetmann/Logic/blob/master/Python/Chapter-4/CNF.ipynb}{CNF.ipynb}.
\begin{enumerate}[(a)]
\item $p \vee q \rightarrow r$,
\item $p \vee q \leftrightarrow r$,
\item $(p \rightarrow q) \leftrightarrow (\neg p \rightarrow \neg q)$,
\item $(p \rightarrow q) \leftrightarrow (\neg q \rightarrow \neg p)$,
\item $\neg r \wedge (q \vee p \rightarrow r) \rightarrow \neg q \wedge \neg p$.
\end{enumerate}


\section{Der Herleitungs-Begriff}
Ist $\{f_1,\cdots,f_n\}$ eine Menge von Formeln, und $g$ eine weitere Formel, so
können wir uns fragen, ob  die  Formel $g$ aus $f_1$, $\cdots$, $f_n$ \blue{folgt}, ob
also 
\\[0.2cm]
\hspace*{1.3cm}
$\ds \models f_1 \wedge \cdots \wedge f_n \rightarrow g$
\\[0.2cm]
gilt.
Es gibt verschiedene Möglichkeiten, diese Frage zu beantworten.  Ein Verfahren kennen wir
schon: Zunächst überführen wir die Formel  $f_1 \wedge \cdots \wedge f_n \rightarrow g$ in
konjunktive Normalform.  Wir erhalten dann eine Menge
$\{k_1,\cdots,k_m\}$ von Klauseln, deren Konjunktion zu der  Formel
\\[0.2cm]
\hspace*{1.3cm}
$f_1 \wedge \cdots \wedge f_n \rightarrow g$
\\[0.2cm] 
äquivalent ist.  Diese Formel ist nun genau dann eine Tautologie, wenn
jede der Klauseln $k_1$, $\cdots$, $k_m$ trivial ist.  

Das oben dargestellte Verfahren ist aber sehr aufwendig.  Wir zeigen dies anhand eines
Beispiels und wenden das Verfahren
an, um zu entscheiden, ob $p \rightarrow r$ aus den beiden Formeln $p \rightarrow q$ und
$q \rightarrow r$ folgt.   Wir bilden also die konjunktive Normalform der Formel 
\\[0.2cm]
\hspace*{1.3cm}
$h := (p \rightarrow q) \wedge (q \rightarrow r) \rightarrow p \rightarrow r$
\\[0.2cm]
und erhalten nach mühsamer Rechnung
\\[0.2cm]
\hspace*{1.3cm}
$\ds (p \vee \neg p \vee r \vee \neg r) \wedge (\neg q \vee \neg p \vee r \vee \neg r) \wedge
     (\neg q \vee \neg p \vee q \vee r) \wedge (p \vee \neg p \vee q \vee r).
$
\\[0.2cm]
Zwar können wir jetzt sehen, dass die Formel $h$ eine Tautologie ist, aber angesichts der
Tatsache, dass wir mit bloßem Auge sehen, dass  $p \rightarrow r$ aus den Formeln $p \rightarrow q$ und
$q \rightarrow r$ folgt, ist die Rechnung  doch  sehr aufwendig.

Wir stellen daher nun eine weiteres Verfahren vor, mit dessen Hilfe wir entscheiden
können, ob eine Formel aus einer gegebenen Menge von Formeln folgt.  Die Idee bei diesem Verfahren
ist es, die zu beweisende Formel mit Hilfe von \blue{Schluss-Regeln} aus vorgegebenen Formeln 
\blue{herzuleiten}.  Das Konzept einer Schluss-Regel wird in der nun folgenden Definition
festgelegt. 
\begin{Definition}[Schluss-Regel]
    Eine aussagenlogische \blue{Schluss-Regel} ist eine Paar der Form  $\bigl\langle \langle f_1, f_2 \rangle, k \bigr\rangle$.
    Dabei ist  $\langle f_1, f_2 \rangle$ ein Paar von aussagenlogischen Formeln und $k$ ist eine
    einzelne aussagenlogische Formel.  
    Die beiden Formeln $f_1$ und $f_2$ bezeichnen wir als
    \blue{Prämissen}, die Formel $k$ heißt die \blue{Konklusion} der Schluss-Regel.
    Ist das Paar $\bigl\langle \langle f_1, f_2 \rangle, k \bigr\rangle$ eine Schluss-Regel, so
    schreiben wir dies als: 
    \\[0.3cm]
    \hspace*{1.3cm}      
    $\schluss{f_1 \qquad f_2}{k}$.
    \\[0.3cm]
    Wir lesen diese Schluss-Regel wie folgt: 
    ``\textsl{Aus $f_1$ und $f_2$ kann auf\, $k$ geschlossen werden.}''
    \eox
\end{Definition}
\vspace*{0.3cm}

\noindent
\textbf{Beispiele} für Schluss-Regeln: 
\\[0.2cm]
\hspace*{1.3cm}            
\begin{tabular}[t]{|c|c|c|}
\hline
\rule{0pt}{15pt} \href{https://en.wikipedia.org/wiki/Modus_ponens}{Modus Ponens} & \href{https://en.wikipedia.org/wiki/Modus_tollens}{Modus Tollens} & \blue{Unfug} \\[0.3cm]
\hline
$
\rule[-15pt]{0pt}{40pt}\schluss{f \quad\quad f \rightarrow g}{g}$ &
$\schluss{\neg g \quad\quad f \rightarrow g}{\neg f}$ &
$\schluss{\neg f \quad\quad f \rightarrow g}{\neg g}$ \\[0.3cm]
\hline
\end{tabular}
\\[0.3cm]

\noindent
Die Definition der Schluss-Regel schränkt zunächst die Formeln, die als Prämissen
bzw.~Konklusion verwendet werden können, nicht weiter ein.  Es ist aber sicher nicht
sinnvoll, beliebige Schluss-Regeln zuzulassen.  Wollen wir Schluss-Regeln in Beweisen
verwenden, so sollten die Schluss-Regeln in dem in der folgenden Definition erklärten
Sinne \blue{korrekt} sein.

\begin{Definition}[Korrekte Schluss-Regel]
  Eine Schluss-Regel der Form \\[0.2cm]
  \hspace*{1.3cm} $\schluss{f_1 \qquad f_2}{k}$ \\[0.2cm]
  ist genau dann \blue{korrekt}, wenn 
  $\models f_1 \wedge f_2 \rightarrow k$ gilt. \eox
\end{Definition}
Mit dieser Definition sehen wir, dass 
die oben als ``\blue{Modus Ponens}'' und ``\blue{Modus  Tollens}'' bezeichneten
Schluss-Regeln korrekt sind, während die als  ``\blue{Unfug}'' bezeichnete
Schluss-Regel nicht korrekt ist.

Im Folgenden gehen wir davon aus, dass alle Formeln Klauseln sind.  Einerseits ist dies
keine echte Einschränkung, denn wir können ja jede Formel in eine äquivalente Menge von
Klauseln umformen.  Andererseits haben die Formeln bei vielen in der Praxis auftretenden aussagenlogischen
Problemen ohnehin die Gestalt von Klauseln.  Daher stellen wir jetzt eine Schluss-Regel vor, in der
sowohl die Prämissen als auch die Konklusion Klauseln sind.
     
\begin{Definition}[Schnitt-Regel]\index{Schnitt-Regel}
    Ist $p$ eine aussagenlogische Variable und sind $k_1$ und $k_2$ Mengen von Literalen,
    die wir als Klauseln interpretieren, so bezeichnen wir die folgende Schluss-Regel
    als die \blue{Schnitt-Regel}: 
    \\[0.2cm]
    \hspace*{1.3cm}
    $\ds \schluss{ k_1 \cup \{p\} \quad \{\neg p\} \cup k_2 }{k_1 \cup k_2}$. 
    \eox
\end{Definition}

\noindent
Die Schnitt-Regel ist sehr allgemein.  Setzen wir in der obigen Definition für $k_1 = \{\}$ und  $k_2 = \{q\}$ 
ein, so erhalten wir die folgende Regel als Spezialfall: \\[0.2cm]
\hspace*{1.3cm} $\schluss{\{\} \cup \{p\} \quad\quad \{\neg p\} \cup \{ q \} }{ \{\} \cup \{q\} }$ \\[0.2cm]
Interpretieren wir nun die Mengen von Literalen als Disjunktionen, so haben wir: \\[0.2cm]
\hspace*{1.3cm}  $\schluss{p \quad\quad \neg p \vee q }{ q }$ \\[0.2cm]
Wenn wir jetzt noch berücksichtigen, dass die Formel $\neg p \vee q$ äquivalent zu der
Formel $p \rightarrow q$ ist, dann ist das nichts anderes als \blue{Modus Ponens}.  
Die Regel \blue{Modus Tollens} ist ebenfalls ein Spezialfall der Schnitt-Regel.  Wir
erhalten diese Regel, wenn wir in der Schnitt-Regel $k_1 = \{ \neg q \}$ und $k_2 = \{\}$ setzen.

\begin{Satz}
  Die Schnitt-Regel ist korrekt.
\end{Satz}
\textbf{Beweis}:  Wir müssen zeigen, dass
\\[0.2cm]
\hspace*{1.3cm}
$\models (k_1 \vee p) \wedge (\neg p \vee k_2) \rightarrow k_1 \vee k_2$
\\[0.2cm]
gilt.  Dazu überführen wir die obige Formel in konjunktive Normalform:
$$
\begin{array}{ll}
  & (k_1 \vee p) \wedge (\neg p \vee k_2) \rightarrow k_1 \vee k_2  \\[0.2cm]
\Leftrightarrow  & 
    \neg \bigl( (k_1 \vee p) \wedge (\neg p \vee k_2) \bigr) \vee k_1 \vee k_2 \\[0.2cm]
\Leftrightarrow  & 
    \neg (k_1 \vee p) \vee \neg (\neg p \vee k_2) \vee k_1 \vee k_2 \\[0.2cm]
\Leftrightarrow  & 
     (\neg k_1 \wedge \neg p) \vee  (p \wedge \neg k_2) \vee k_1 \vee k_2 \\[0.2cm]
\Leftrightarrow  & 
     (\neg k_1 \vee p \vee k_1 \vee k_2)  \wedge 
     (\neg k_1 \vee \neg k_2 \vee k_1 \vee k_2)  \wedge 
     (\neg p \vee p \vee k_1 \vee k_2)  \wedge 
     (\neg p \vee \neg k_2 \vee k_1 \vee k_2) 
      \\[0.2cm]
\Leftrightarrow  & 
     \verum  \wedge 
     \verum  \wedge 
     \verum  \wedge 
     \verum 
      \\[0.2cm]
\Leftrightarrow  & 
     \verum    \hspace*{13.5cm} _\Box
      \\
\end{array}
$$



\begin{Definition}[Herleitungs-Begriff, $\vdash$]
    Es sei $M$ eine Menge von Klauseln  und $f$ sei eine einzelne Klausel.  
    Die Formeln aus $M$ bezeichnen wir als unsere \blue{Axiome}.  Unser Ziel ist es, mit den Axiomen aus $M$
    die Formel $f$ \blue{herzuleiten}.  Dazu definieren wir induktiv die Relation \\[0.2cm]
    \hspace*{1.3cm}
    $\blue{M \vdash f}$. \\[0.2cm]
    Wir lesen ``$M \vdash f$'' als ``$M$ \blue{leitet} $f$ \blue{her}''.  Die induktive Definition ist
    wie folgt:
    \begin{enumerate}
    \item Aus einer Menge $M$ von Annahmen kann jede der Annahmen hergeleitet werden: \\[0.2cm]
          \hspace*{1.3cm} 
          Falls $f \el M$ ist, dann gilt  $M \vdash f$.
    \item Sind $k_1 \cup \{p\}$ und $\{ \neg p \} \cup k_2$ Klauseln, die aus $M$
          hergeleitet werden können, so kann mit der Schnitt-Regel auch die Klausel $k_1 \cup k_2$ aus $M$
          hergeleitet werden: \\[0.2cm]
          \hspace*{1.3cm} 
          Falls sowohl $M \vdash k_1 \cup \{p\}$ als auch $M \vdash \{ \neg p \} \cup k_2$
          gilt, dann gilt auch $M \vdash k_1 \cup k_2$.
    \eox
    \end{enumerate}
\end{Definition}



\noindent
\textbf{Beispiel}:  Um den Beweis-Begriff zu veranschaulichen geben wir ein Beispiel und
zeigen 
\[ \bigl\{\; \{\neg p, q\},\; \{ \neg q, \neg p \},\; \{ \neg q, p \},\; \{ q, p \}\; \bigr\} \vdash \falsum.
\]
Gleichzeitig zeigen wir anhand des Beispiels, wie wir Beweise zu Papier bringen:
\begin{enumerate}
\item Aus $\{\neg p, q \}$ und $\{ \neg q, \neg p \}$ folgt mit der Schnitt-Regel   
      $\{ \neg p, \neg p \}$.   Wegen $\{ \neg p, \neg p \} = \{ \neg p \}$
      schreiben wir dies als 
      \[ \{\neg p, q \}, \{ \neg q, \neg p \} \;\vdash\; \{ \neg p \}. \]
      \remark
      Dieses Beispiel zeigt, dass die Klausel $k_1 \cup k_2$ durchaus auch weniger
      Elemente enthalten kann als die Summe $\mathtt{card}(k_1) + \mathtt{card}(k_2)$.  Dieser
      Fall tritt genau dann ein, wenn es Literale gibt, die sowohl in $k_1$ als auch in
      $k_2$ vorkommen.
\item $\{\neg q, \neg p \},\; \{ p, \neg q \} \;\vdash\; \{ \neg q \}$. 
\item $\{ p, q \},\; \{ \neg q \} \;\vdash\; \{ p \}$. 
\item $\{ \neg p \},\; \{ p \} \;\vdash\; \{\}$. 
\end{enumerate}
Als weiteres Beipiel zeigen wir nun, dass $p \rightarrow r$ aus $p \rightarrow q$ und $q \rightarrow r$ 
folgt.  Dazu überführen wir zunächst alle Formeln in Klauseln: 
\[ \texttt{cnf}(p \rightarrow q) = \bigl\{ \{ \neg p, q \} \bigr\}, \quad
   \texttt{cnf}(q \rightarrow r) = \bigl\{ \{ \neg q, r \} \bigr\}, \quad 
   \texttt{cnf}(p \rightarrow r) = \bigl\{ \{ \neg p, r \} \bigr\}.
\]
Wir haben also $M = \bigl\{\, \{ \neg p, q \},\; \{ \neg q, r \}\,\bigr\}$ und müssen zeigen, dass
\[ M \vdash  \{ \neg p, r \} \]
gilt.  Der Beweis besteht aus einer einzigen Anwendung der Schnitt-Regel: 
\\[0.2cm]
\hspace*{1.3cm}
$ \{ \neg p, q \},\; \{ \neg q, r \} \;\vdash\; \{ \neg p, r \}$.  \eox

\subsection{Eigenschaften des Herleitungs-Begriffs}
Die Relation $\vdash$ hat zwei wichtige Eigenschaften:

\begin{Satz}[\blue{Korrektheit}]
  Ist $\{k_1, \cdots, k_n \}$ eine Menge von Klauseln und $k$ eine einzelne Klausel,
  so haben wir:
  \\[0.2cm]
  \hspace*{3.3cm} Wenn $\{k_1, \cdots, k_n \} \vdash k$ gilt, dann gilt auch $\models k_1 \wedge \cdots \wedge
  k_n \rightarrow k$.
  \\[0.2cm]
  Mit anderen Worten: Wenn wir eine Klausel $k$ mit Hilfe der Annahmen $k_1$, $\cdots$, $k_n$ beweisen können,
  dann folgt die Klausel $k$ logisch aus diesen Annahmen.
\end{Satz}

\noindent
\textbf{Beweis}:  Der Beweis des Korrektheits-Satzes verläuft durch eine Induktion nach der Definition der Relation $\vdash$. 
\begin{enumerate}
\item Fall: Es gilt $\{ k_1, \cdots, k_n \} \vdash k$, weil $k \in \{ k_1, \cdots, k_n \}$ ist.  
      Dann gibt es also ein $i \in \{1,\cdots,n\}$, so dass $k = k_i$ ist.  In diesem Fall
      müssen wir
      \\[0.2cm]
      \hspace*{1.3cm}
      $\ds\models k_1 \wedge \cdots \wedge k_i \wedge \cdots \wedge k_n \rightarrow k_i$
      \\[0.2cm]
      zeigen, was offensichtlich ist.
\item Fall: Es gilt $\{ k_1, \cdots, k_n \} \vdash k$, weil es eine aussagenlogische
      Variable $p$ und Klauseln $g$ und $h$ gibt, so dass 
      \\[0.2cm]
      \hspace*{1.3cm}
      $\{ k_1, \cdots, k_n \} \vdash g \cup \{ p \} \quad \mathrm{und} \quad
         \{ k_1, \cdots, k_n \} \vdash h \cup \{ \neg p \}
      $
      \\[0.2cm]
      gilt und daraus haben wir mit der Schnitt-Regel auf
      \\[0.2cm]
      \hspace*{1.3cm}
      $\{ k_1, \cdots, k_n \} \vdash g \cup h$
      \\[0.2cm]
      geschlossen, wobei $k = g \cup h$ gilt.  Wir müssen nun zeigen, dass 
      \\[0.2cm]
      \hspace*{1.3cm}
      $\models k_1 \wedge \cdots \wedge k_n \rightarrow g \vee h$
      \\[0.2cm]
      gilt.  Es sei also $\mathcal{I}$ eine aussagenlogische Interpretation, so dass
      \\[0.2cm]
      \hspace*{1.3cm}
      $\mathcal{I}(k_1 \wedge \cdots \wedge k_n) = \texttt{True}$  
      \\[0.2cm]
      ist. Dann müssen wir zeigen, dass 
      \\[0.2cm]
      \hspace*{1.3cm}
      $\mathcal{I}(g) = \texttt{True}$ \quad oder \quad $\mathcal{I}(h) = \texttt{True}$
      \\[0.2cm]
      ist.  Nach Induktions-Voraussetzung wissen wir
      \\[0.2cm]
      \hspace*{1.3cm}
      $\models k_1 \wedge \cdots \wedge k_n \rightarrow g \vee p \quad \mathrm{und} \quad 
         \models k_1 \wedge \cdots \wedge k_n \rightarrow h \vee \neg p
      $.
      \\[0.2cm]
      Wegen $\mathcal{I}(k_1 \wedge \cdots \wedge k_n) = \texttt{True}$ folgt dann
      \\[0.2cm]
      \hspace*{1.3cm}
      $\mathcal{I}(g \vee p) = \texttt{True}$ \quad und \quad $\mathcal{I}(h \vee \neg p) = \texttt{True}$.
      \\[0.2cm]
      Nun gibt es zwei Fälle:
      \begin{enumerate}
      \item Fall: $\mathcal{I}(p) = \texttt{True}$.

            Dann ist $\mathcal{I}(\neg p) = \texttt{False}$ und daher folgt aus der Tatsache, dass 
            $\mathcal{I}(h \vee \neg p) = \texttt{True}$ ist, dass
            \\[0.2cm]
            \hspace*{1.3cm}
            $\mathcal{I}(h) = \texttt{True}$
            \\[0.2cm]
            sein muss.  Daraus folgt aber sofort
            \\[0.2cm]
            \hspace*{1.3cm}
            $\mathcal{I}(g \vee h) = \texttt{True}$.  $\green{\surd}$
      \item Fall: $\mathcal{I}(p) = \texttt{False}$.

            Nun folgt aus $\mathcal{I}(g \vee p) = \texttt{True}$, dass
            \\[0.2cm]
            \hspace*{1.3cm}
            $\mathcal{I}(g) = \texttt{True}$
            \\[0.2cm]
            gelten muss.  Also gilt auch in diesem Fall
            \\[0.2cm]
            \hspace*{1.3cm}
            $\mathcal{I}(g \vee h) = \texttt{True}$.  $\green{\surd}$ 
            \qed
      \end{enumerate}
\end{enumerate}

\noindent
Die Umkehrung dieses Satzes gilt nur in abgeschwächter Form und zwar dann, wenn $k$
die leere Klausel ist, die ja dem Falsum entspricht.  Wir sagen daher, dass die Schnitt-Regel
\blue{widerlegungs-vollständig} ist.

\remark
Es gibt alternative Definitionen des Herleitungs-Begriffs, die nicht nur \blue{widerlegungs-vollständig} sondern
tatsächlich \blue{vollständig} sind, d.h.~immer wenn $\models f_1 \wedge \cdots \wedge f_n \rightarrow g$ gilt,
dann folgt auch
\\[0.2cm]
\hspace*{1.3cm}
$\{ f_1, \cdots, f_n \} \vdash g$.
\\[0.2cm]  
Diese Herleitungs-Begriffe sind allerdings wesentlich komplexer und daher umständlicher zu implementieren.  Wir
werden später sehen, dass die Widerlegungs-Vollständigkeit für unsere Zwecke ausreichend ist.
\eox

\subsection{Beweis der Widerlegungs-Vollständigkeit}
Um den Satz von der Widerlegungs-Vollständigkeit der Aussagenlogik kompakt formulieren zu können, benötigen wir
den Begriff der \blue{Erfüllbarkeit}, den wir jetzt formal einführen. 

\begin{Definition}[Erfüllbarkeit] \index{erfüllbar}
  Es sei $M$ eine Menge von aussagenlogischen Formeln.
  Falls es eine aussagen\-logische Interpretation $\I$ gibt, die alle Formeln aus $M$ erfüllt, für die also
  \\[0.2cm]
  \hspace*{1.3cm}
  $\mathcal{I}(f) = \texttt{True}$ \quad für alle $f \in M$
  \\[0.2cm]
  gilt, so nennen wir $M$ \blue{erfüllbar}.  Weiter sagen wir, dass $M$ \blue{unerfüllbar}\index{unerfüllbar}
  ist und schreiben 
  \\[0.2cm]
  \hspace*{1.3cm}
  $\blue{M \models \falsum}$, \index{$M \models \falsum$}
  \\[0.2cm]
  wenn es keine aussagenlogische Interpretation $\I$ gibt, die gleichzeitig alle Formel aus $M$ erfüllt.
  Be\-zeichnen wir die Menge der aussagenlogischen Interpretationen mit
  \textsc{\blue{Ali}}, so schreibt sich das formal als
  \\[0.2cm]
  \hspace*{1.3cm}
  $\blue{M \models \falsum}$ \quad g.d.w. \quad 
  $\forall \I \in \textsc{Ali}: \exists C \in M: \I(C) = \texttt{False}$.
  \\[0.2cm]
  Falls eine Menge $M$ von aussagenlogischen Formeln erfüllbar ist, schreiben wir auch
  \\[0.2cm]
  \hspace*{1.3cm}
  $\blue{M \not\models \falsum}$.
  \eox
\end{Definition}

\remark 
Ist $M = \{ f_1, \cdots, f_n \}$ eine Menge von aussagenlogischen Formeln, so können Sie sich leicht überlegen, dass
$M$ genau dann unerfüllbar ist, wenn
\\[0.2cm]
\hspace*{1.3cm}
$\models f_1 \wedge \cdots \wedge f_n \rightarrow \falsum$
\\[0.2cm]
gilt. \eox

\begin{Definition}[Saturierte Klausel-Mengen] \hspace*{\fill} \linebreak
  Eine Menge $M$ von Klauseln ist \blue{saturiert} \index{saturiert} wenn jede Klausel, die sich aus zwei
  Klauseln $C_1,C_2 \in M$ mit Hilfe der Schnitt-Regel ableiten lässt, bereits selbst wieder ein Element der
  Menge $M$ ist.
\end{Definition}

\remark
Ist $M$ eine endlichen Menge von Klauseln, so können wir $M$ zu einer saturierten Menge von Klauseln
$\overline{M}$ erweitern, indem wir $\overline{M}$ als die Menge $M$ initialisieren und dann solange die
Schnitt-Regel auf Klauseln aus $\overline{M}$ anwenden und zu der Menge $\overline{M}$ hinzufügen, wie dies möglich
ist. Da es zu einer gegebenen endlichen Menge von aussagenlogischen Variablen nur eine beschränkte Menge von
Klauseln gibt, die wir aus diesen Variablen bilden können, muss dieses Verfahren abbrechen und die Menge der
Formeln, die wir dann erhalten, ist saturiert. \eox

\begin{Satz}[\blue{Widerlegungs-Vollständigkeit}] \label{widerlegungs-vollstaendig}
  Ist  $M$ eine Menge von Klauseln,
  so haben wir:
  \\[0.2cm]
  \hspace*{1.3cm} 
  Wenn $M \models \falsum$ ist, dann gilt auch  $M \vdash \{\}$.
\end{Satz}

\noindent
\textbf{Beweis}:  Wir führen den Beweis durch
Kontraposition.  Wir nehmen an, dass $M$ eine endliche Menge von Klauseln ist,
\begin{enumerate}[(a)]
\item die einerseits zwar unerfüllbar ist, es gilt also
      \\[0.2cm]
      \hspace*{1.3cm}
      $M \models \falsum$,
\item aus der sich aber andererseits die leere Klausel nicht herleiten lässt, was wir als 
      \\[0.2cm]
      \hspace*{1.3cm}
      $M \not\,\vdash \{\}$
      \\[0.2cm]
      shreiben.
\end{enumerate}
Wir werden zeigen, dass es dann eine aussagenlogische Belegung $\mathcal{I}$ gibt, unter der alle Klauseln aus
$M$ wahr werden, was der Unerfüllbarkeit von $M$ widerspricht.

Nach der obigen Bemerkung können wir die Menge $M$ saturieren und erhalten dann die satu\-rierte Menge
$\overline{M}$.  Es sei  
\\[0.2cm]
\hspace*{1.3cm}
$\{ p_1, \cdots, p_N \}$
\\[0.2cm]
die Menge aller aussagenlogischen Variablen, die in Klauseln aus $M$ auftreten.  Im folgenden be\-zeichnen wir
die Menge der aussagenlogischen Variablen, die in einer Klausel $C$ auftreten, mit $\mathtt{var}(C)$.
Für alle $k=0,1,\cdots,N$
definieren wir nun eine aussagenlogische Belegung $\mathcal{I}_k$ durch Induktion nach $k$.  Die
aussagenlogischen Belegungen haben für alle $k=0,1,\cdots,N$ die Eigenschaft, dass für jede Klausel $C \in \overline{M}$,
die nur die Variablen $p_1,\cdots,p_k$ enthält,
$\mathcal{I}_k(C) = \mathtt{True}$ gilt.  Als Formel schreibt sich dies wie folgt:
\\[0.2cm]
\hspace*{1.3cm}
$\forall C \in \overline{M}:\Bigl(\mathtt{var}(C) \subseteq \{p_1, \cdots, p_k\} \Rightarrow \mathcal{I}_k(C) = \mathtt{True}\Bigl)$.
\\[0.2cm]
Außerdem definiert die aussagenlogische Belegung $\mathcal{I}_k$ nur Werte für die Variablen
$p_1,\cdots,p_k$.
\begin{enumerate}
\item[I.A.:] $k=0$.
 
  Wir definieren $\mathcal{I}_0$ als die leere aussagenlogische Belegung, die keiner
  Variablen einen Wert zuweist.  Um $(*)$ nachzuweisen müssen wir zeigen, dass für jede Klausel $C \in \overline{M}$, die
  keine aussagenlogische Variable enthält, $\mathcal{I}_0(C) = \mathtt{True}$ gilt.  Die einzige Klausel, die
  keine Variable enthält, ist die leere Klausel.  Da wir vorausgesetzt haben, dass  $M \not\,\vdash \{\}$
  gilt, kann $\overline{M}$ die leere Klausel nicht enthalten.  Also ist nichts zu zeigen.
\item[I.S.:] $k \mapsto k+1$.

  Nach IV ist $\mathcal{I}_k$ bereits definiert.  Wir setzen zunächst
  \\[0.2cm]
  \hspace*{1.3cm}
  $\mathcal{I}_{k+1}(p_i) := \mathcal{I}_k(p_i)$ \quad für alle $i=1,\cdots,k$
  \\[0.2cm]
  und müssen nun noch $\mathcal{I}_{k+1}(p_{k+1})$ definieren.  Dies geschieht über eine Fallunterscheidung.
  \begin{enumerate}
  \item Es gibt eine Klausel
    \\[0.2cm]
    \hspace*{1.3cm}
    $C \cup \{p_{k+1}\} \in \overline{M}$,
    \\[0.2cm]
    so dass $C$ höchstens die Variablen $p_1,\cdots,p_k$ enthält und außerdem $\mathcal{I}_k(C) = \mathtt{False}$
    ist.  Dann setzen wir
    \\[0.2cm]
    \hspace*{1.3cm}
    $\mathcal{I}_{k+1}(p_{k+1}) := \mathtt{True}$,
    \\[0.2cm]    
    denn sonst würde ja insgesamt $\mathcal{I}_{k+1}\bigl(C \cup \{p_{k+1}\}\bigr) = \mathtt{False}$ gelten. 

    Wir müssen nun zeigen, dass für jede Klausel $D \in \overline{M}$, die nur die Variablen
    $p_1,\cdots,p_k,p_{k+1}$ die Aussage $\mathcal{I}_{k+1}(D) = \mathtt{True}$ gilt.  Hier gibt es drei
    Möglichkeiten:
    \begin{enumerate}[1.]
    \item Fall: $\mathtt{var}(D) \subseteq \{p_1,\cdots,p_k\}$

          Dann gilt die Behauptung nach IV, denn auf diesen Variablen stimmen die Belegungen
          $\mathcal{I}_{k+1}$ und $\mathcal{I}_k$ überein.
    \item Fall: $p_{k+1} \in D$.
          
          Da wir $\mathcal{I}_{k+1}(p_{k+1})$ als $\texttt{True}$ definiert haben, gilt $\mathcal{I}_{k+1}(D) = \mathtt{True}$.
    \item Fall: $(\neg p_{k+1}) \in D$.

          Dann hat $D$ also die Form
          \\[0.2cm]
          \hspace*{1.3cm}
          $D = E \cup \{ \neg p_{k+1} \}$.
          \\[0.2cm]
          An dieser Stelle benötigen wir nun die Tatsache, dass $\overline{M}$ saturiert ist.  Wir wenden die
          Schnitt-Regel auf die Klauseln $C \cup \{ p_{k+1} \}$ und $E \cup \{ \neg p_{k+1} \}$ an:
          \\[0.2cm]
          \hspace*{1.3cm}
          $C \cup \{ p_{k+1} \},\quad E \cup \{ \neg p_{k+1} \} \quad\vdash\quad C \cup E$
          \\[0.2cm]
          Da $\overline{M}$ saturiert ist, gilt $C\cup E \in \overline{M}$.  $C \cup E$ enthält nur die
          Variablen $p_1,\cdots, p_k$.  Nach IV gilt also
          \\[0.2cm]
          \hspace*{1.3cm}
          $\mathcal{I}_{k+1}(C \cup E) = \mathcal{I}_{k}(C \cup E) = \mathtt{True}$.
          \\[0.2cm]
          Da $\mathcal{I}_k(C) = \mathtt{False}$ ist, muss $\mathcal{I}_k(E) = \mathtt{True}$ gelten.  Damit
          haben wir
          \\[0.2cm]
          \hspace*{1.3cm}
          $
          \begin{array}[t]{lcl}
            \mathcal{I}_{k+1}(D) & = & \mathcal{I}_{k+1}\bigl(E \cup \{ \neg p_{k+1} \bigr\}) \\
                                & = & \mathcal{I}_k(E) \;\circvee\;\; \mathcal{I}_{k+1}(\neg p_{k+1}) \\
                                & = & \mathtt{True} \;\circvee\;\; \mathtt{False} = \mathtt{True}
          \end{array}
          $
          \\[0.2cm]
          und das war zu zeigen.
    \end{enumerate}
  \item Es gibt keine Klausel
    \\[0.2cm]
    \hspace*{1.3cm}
    $C \cup \{p_{k+1}\} \in \overline{M}$,
    \\[0.2cm]
    so dass $C$ höchstens die Variablen $p_1,\cdots,p_k$ enthält und außerdem $\mathcal{I}_k(C) = \mathtt{False}$
    ist.  Dann setzen wir
    \\[0.2cm]
    \hspace*{1.3cm}
    $\mathcal{I}_{k+1}(p_{k+1}) := \mathtt{False}$.
    \\[0.2cm]    
    In diesem Fall gibt es für Klauseln $C \in \overline{M}$ drei Fälle:
    \begin{enumerate}[1.]
    \item $C$ enthält die Variable $p_{k+1}$ nicht.
      
          In diesem Fall gilt nach IV bereits $\mathcal{I}_{k+1} = \mathtt{True}$.        
    \item $C = D \cup \{ p_{k+1} \}$.

          In diesem Fall muss nach der Fallunterscheidung von Fall (b) $\mathcal{I}_k(D) = \mathtt{True}$
          gelten und daraus folgt sofort, dass auch $\mathcal{I}_{k+1}(C) = \mathtt{True}$ ist.
    \item $C = D \cup \{ \neg p_{k+1} \}$.

          In diesem Fall gilt nach Definition von $\mathcal{I}$, dass $\mathcal{I}_{k+1}(p_{k+1}) = \mathtt{False}$
          ist und daraus folgt sofort, dass $\mathcal{I}_{k+1}(\neg p_{k+1}) = \mathtt{True}$ ist, was
          $\mathcal{I}_{k+1}(C) = \mathtt{True}$ zur Folge hat.
    \end{enumerate}
    Durch die Induktion haben wir damit gezeigt, dass die aussagenlogische Belegungen
    $\mathcal{I}_{N}$ alle Klauseln $C \in \overline{M}$ wahr macht, denn in $\overline{M}$ kommen ja nur die
    aussagenlogischen Variablen $p_1$, $\cdots$, $p_N$ vor.
    Da $M \subseteq \overline{M}$, macht $\mathcal{I}_{N}$ damit auch alle Klauseln aus $M$ war, was zum
    Widerspruch mit der Voraussetzung $M \models \falsum$ führt. \qed
  \end{enumerate}
\end{enumerate}

\subsection{Konstruktive Interpretation des Beweises der Widerlegungs-Vollständigkeit}
In diesem Abschnitt implementieren wir ein Programm, mit dessen Hilfe sich für eine Klausel-Menge $M$, aus der
sich die leere Klausel-Menge nicht herleiten lässt, eine aussagenlogische Belegung $\mathcal{I}$ berechnen
lässt, die alle Klauseln aus $M$ war macht.  Dieses Programm ist in den 
Abbildungen \ref{fig:Completeness.ipynb-1}, \ref{fig:Completeness.ipynb-2} und
\ref{fig:Completeness.ipynb-3} auf den folgenden Seiten gezeigt.  Sie finden dieses Programm unter der Adresse
\\[0.2cm]
\hspace*{0.8cm}
\href{https://github.com/karlstroetmann/Logic/blob/master/Python/Chapter-4/Completeness.ipynb}{\texttt{github.com/karlstroetmann/Logic/blob/master/Python/Chapter-4/Completeness.ipynb}}
\\[0.2cm]
im Netz.

\begin{figure}[!ht]
\centering
\begin{minted}[ frame         = lines, 
                framesep      = 0.3cm, 
                firstnumber   = 1,
                numbers       = left,
                numbersep     = -0.2cm,
                bgcolor       = sepia,
                xleftmargin   = 0.0cm,
                xrightmargin  = 0.0cm,
              ]{python3}
    def complement(l):
        "Compute the complement of the literal l."
        match l:
            case p if isinstance(p, str):  return ('¬', p)
            case ('¬', p):                 return p

     def extractVariable(l):
        "Extract the variable of the literal l."
        match l:
            case p if isinstance(p, str): return p           
            case ('¬', p):                return p
    
    def collectVariables(M):
        "Return the set of all variables occurring in M."
        return { extractVariable(l) for C in M 
                                    for l in C
               }        
        
    def cutRule(C1, C2):
        '''
        Return the set of all clauses that can be deduced with the cut rule 
        from the clauses c1 and c2.
        '''
        return { C1 - {l} | C2 - {complement(l)} for l in C1
                                                 if  complement(l) in C2
               }
\end{minted}
\vspace*{-0.3cm}
\caption{Hilfsprozeduren, die in Abbildung \ref{fig:Completeness.ipynb-2} genutzt werden}
\label{fig:Completeness.ipynb-1}
\end{figure}

Die Grundidee
bei diesem Programm besteht darin, dass wir versuchen, aus einer gegebenen Menge $M$ von Klauseln
\underline{alle} Klauseln herzuleiten, die mit der Schnitt-Regel aus $M$ herleitbar sind.  Wenn wir dabei auch
die leere Klausel herleiten, dann ist $M$ aufgrund der Korrektheit der Schnitt-Regel offenbar
unerfüllbar.  Falls es uns aber nicht gelingt, die leere Klausel aus $M$ abzuleiten, dann konstruieren wir
aus der Menge aller Klauseln, die wir aus $M$ hergeleitet haben, eine aussagenlogische Interpretation
$\I$, die alle Klauseln aus $M$ erfüllt, womit $M$ erfüllbar wäre.
Wir diskutieren zunächst die Hilfsprozeduren, die in Abbildung \ref{fig:Completeness.ipynb-1} gezeigt
sind. 
\begin{enumerate}
\item Die Funktion \texttt{complement} erhält als Argument ein Literal $l$ und berechnet das
      \blue{Komplement} $\komplement{l}$ dieses Literals.
      Falls das Literal $l$ eine aussagenlogische Variable $p$ ist, was wir daran erkennen, dass $l$ ein String
      ist, so haben wir $\komplement{p} = \neg p$.
      Falls $l$ die Form $\neg p$ mit einer aussagenlogischen Variablen $p$ hat, so gilt $\komplement{\neg p} =
      p$.
\item Die Funktion \texttt{extractVariable} extrahiert die aussagenlogische Variable, die in einem Literal $l$
      enthalten ist.  Die Implementierung verläuft analog zu der Implementierung der Funktion \linebreak
      \texttt{complement} über eine Fallunterscheidung, bei der wir berücksichtigen, dass $l$ entweder die Form
      $p$ oder die Form $\neg p$ hat, wobei $p$ die zu extrahierende aussagenlogische Variable ist.
\item Die Funktion \texttt{collectVars} erhält als Argument eine Menge $M$ von Klauseln, wobei die
      einzelnen Klauseln  \mbox{$C \!\in\! M$} als Mengen von Literalen dargestellt werden.  Die Aufgabe der
      Funktion \texttt{collectVars} ist es, die Menge aller aussagenlogischen Variablen zu
      berechnen, die in einer der Klauseln $C$ aus $M$ vorkommen.  Bei der Implementierung iterieren
      wir zunächst über die Klauseln $C$ der Menge $M$ und dann für jede Klausel $C$ über die in $C$
      vorkommenden Literale $l$, wobei die Literale mit Hilfe der Funktion \texttt{extractVariable} in
      aussagenlogische Variablen umgewandelt werden.
\item Die Funktion \texttt{cutRule} erhält als Argumente zwei Klauseln $C_1$ und $C_2$ und berechnet
      die Menge aller Klauseln, die mit Hilfe einer Anwendung der Schnitt-Regel aus $C_1$ und $C_2$ gefolgert werden
      können.  Beispielsweise können wir aus den beiden Klauseln
      \\[0.2cm]
      \hspace*{1.3cm}
      $\{ p, q \}$ \quad und \quad $\{ \neg p, \neg q \}$ 
      \\[0.2cm]
      mit der Schnitt-Regel sowohl die Klausel
      \\[0.2cm]
      \hspace*{1.3cm}
      $\{q, \neg q\}$ \quad als auch die Klausel \quad $\{p, \neg p \}$
      \\[0.2cm]
      herleiten.
\end{enumerate}

\begin{figure}[!ht]
\centering
\begin{minted}[ frame         = lines, 
                framesep      = 0.3cm, 
                firstnumber   = last,
                numbers       = left,
                numbersep     = -0.2cm,
                bgcolor       = sepia,
                xleftmargin   = 0.0cm,
                xrightmargin  = 0.0cm,
              ]{python3}
    def saturate(Clauses):
        while True:
            Derived = { C for C1 in Clauses
                          for C2 in Clauses
                          for C in cutRule(C1, C2)
                      }
            if frozenset() in Derived:
                return { frozenset() }  # This is the set notation of ⊥.
            Derived -= Clauses
            if Derived == set():        # no new clauses found
                return Clauses
            Clauses |= Derived
\end{minted}
\vspace*{-0.3cm}
\caption{Die Funktion \texttt{saturate}}
\label{fig:Completeness.ipynb-2}
\end{figure}
    
Abbildung \ref{fig:Completeness.ipynb-2} zeigt die Funktion \texttt{saturate}.  Diese Funktion erhält
als Eingabe eine Menge \texttt{Clauses} von aus\-sagenlogischen Klauseln, die als Mengen von Literalen
dargestellt werden.  Aufgabe der Funktion ist es, alle Klauseln herzuleiten, die mit Hilfe der
Schnitt-Regel auf direktem oder indirekten Wege aus der Menge \texttt{Clauses} hergeleitet werden
können.  Genauer gesagt ist die Menge $S$ der Klauseln, die von der Funktion \texttt{saturate}
zurück gegeben wird, unter Anwendung der Schnitt-Regel \blue{saturiert}, es gilt also:

\begin{enumerate}
\item Falls $S$ die leere Klausel $\{\}$ enthält, dann ist $S$ saturiert.
\item Andernfalls muss \texttt{Clauses} eine Teilmenge von $S$ sein und es muss zusätzlich Folgendes
      gelten: Falls für ein Literal $l$ sowohl die Klausel $C_1 \cup \{ l \}$ als auch die Klausel $C_2 \cup
      \bigl\{ \komplement{l} \bigr\}$ Klausel in $S$
      enthalten ist, dann ist auch die Klausel $C_1 \cup C_2$ ein Element der Klausel\-menge $S$:
      \\[0.2cm]
      \hspace*{1.3cm}
      $C_1 \cup \{ l \} \in S \;\wedge\; C_2 \cup \bigl\{ \komplement{l} \bigr\} \in S \;\Rightarrow\; C_1 \cup C_2 \in S$ 
\end{enumerate}

Wir erläutern nun die Implementierung der Funktion \texttt{saturate}.
\begin{enumerate}
\item Die \texttt{while}-Schleife, die in Zeile 30 beginnt, hat die Aufgabe, die Schnitt-Regel
      so lange wie möglich anzuwenden, um mit Hilfe der Schnitt-Regel neue Klauseln aus den gegebenen
      Klauseln herzuleiten.  Da die Bedingung dieser Schleife den Wert \texttt{True} hat, kann diese
      Schleife nur durch die Ausführung einer der beiden \texttt{return}-Befehle in Zeile 34 bzw.~Zeile 37
      abgebrochen werden. 
\item In Zeile 29 wird die Menge \texttt{Derived} als die Menge der Klauseln definiert, die mit Hilfe der
      Schnitt-Regel aus zwei der Klauseln in der Menge \texttt{Clauses} gefolgert werden können.
\item Falls die Menge \texttt{Derived} die leere Klausel enthält, dann ist die Menge
      \texttt{Clauses} widersprüchlich und die Funktion \texttt{saturate} gibt als Ergebnis die
      Menge $\bigl\{ \{\} \bigr\}$ zurück, wobei die innere Menge als \texttt{frozenset} dargestellt werden
      muss.  Beachten Sie, dass die Menge $\bigl\{ \{\} \bigr\}$ dem Falsum entspricht.
\item Andernfalls ziehen wir in Zeile 35 von der Menge \texttt{Derived} zunächst die Klauseln
      ab, die schon in der Menge \texttt{Clauses} vorhanden waren, denn es geht uns darum
      festzustellen, ob wir im letzten Schritt tatsächlich 
      \underline{neue} Klauseln gefunden haben, oder ob alle Klauseln, die wir im letzten Schritt in Zeile 31
      hergeleitet haben, schon vorher bekannt waren.
\item Falls wir nun in Zeile 36 feststellen, dass wir keine neuen Klauseln hergeleitet haben,
      dann ist die Menge \texttt{Clauses} \blue{saturiert} und wir geben diese Menge in Zeile 39 zurück.
\item Andernfalls fügen wir in Zeile 38 die Klauseln, die wir neu gefunden haben, zu der Menge
      \texttt{Clauses} hinzu und setzen die \texttt{while}-Schleife fort.
\end{enumerate}
An dieser Stelle müssen wir uns überlegen, dass die \texttt{while}-Schleife tatsächlich irgendwann
abbricht.  Das hat zwei Gründe:  
\begin{enumerate}
\item In jeder Iteration der Schleife wird die Anzahl der Elemente der Menge \texttt{Clauses}
      mindestens um Eins erhöht, denn wir wissen ja, dass die Menge \texttt{Derived}, die wir in Zeile 38 zur
      Menge \texttt{Clauses} hinzufügen, einerseits nicht leer ist und andererseits auch nur solche
      Klauseln enthält, die nicht bereits in \texttt{Clauses} auftreten.
\item Die Menge \texttt{Clauses}, mit der wir ursprünglich starten, enthält eine bestimmte Anzahl $n$
      von aussagenlogischen Variablen.  Bei der Anwendung der Schnitt-Regel werden aber keine neuen
      Variablen erzeugt.  Daher bleibt die Anzahl der aussagenlogischen Variablen, die in
      \texttt{Clauses} auftreten, immer gleich.  Damit ist natürlich auch die Anzahl der Literale,
      die in \texttt{Clauses} auftreten, beschränkt: Wenn es nur $n$ aussagenlogische Variablen gibt,
      dann kann es auch höchstens $2 \cdot n$ verschiedene Literale geben.  Jede Klausel aus \texttt{Clauses} ist
      aber eine Teilmenge der Menge aller Literale.  Da eine Menge mit $k$ Elementen insgesamt $2^k$
      Teilmengen hat, gibt es höchstens $2^{2 \cdot n}$ verschiedene Klauseln, die in
      \texttt{Clauses} auftreten können.  
\end{enumerate}
Aus den beiden oben angegebenen Gründen können wir schließen, dass die \texttt{while}-Schleife in
Zeile 28 spätestens nach $2^{2 \cdot n}$ Iterationen abgebrochen wird.


\begin{figure}[!ht]
\centering
\begin{minted}[ frame         = lines, 
                framesep      = 0.3cm, 
                firstnumber   = last,
                numbers       = left,
                numbersep     = -0.2cm,
                bgcolor       = sepia,
                xleftmargin   = 0.0cm,
                xrightmargin  = 0.0cm,
              ]{python3}
    def findValuation(Clauses):
        "Given a set of Clauses, find an interpretation satisfying all clauses."
        Variables = collectVariables(Clauses)
        Clauses   = saturate(Clauses)
        if frozenset() in Clauses:  # The set Clauses is inconsistent.
            return False
        Literals = set()
        for p in Variables:
            if any(C for C in Clauses 
                     if  p in C and C - {p} <= { complement(l) for l in Literals }
                  ):
                Literals |= { p }
            else:
                Literals |= { ('¬', p) }
        return Literals
\end{minted}
\vspace*{-0.3cm}
\caption{Die Funktion \texttt{findValuation}.\index{\texttt{findValuation}}}
\label{fig:Completeness.ipynb-3}
\end{figure}

\FloatBarrier

Als nächstes diskutieren wir die Implementierung der Funktion \texttt{findValuation}, die in
Abbildung \ref{fig:Completeness.ipynb-3} gezeigt ist.  Diese Funktion erhält als Eingabe eine Menge
\texttt{Clauses} von Klauseln.  Falls diese Menge widersprüchlich ist, soll die Funktion
das Ergebnis \texttt{False} zurück geben.  Andernfalls soll eine aussagenlogische Belegung $\I$ berechnet werden,
unter der alle Klauseln aus der Menge \texttt{Clauses} erfüllt sind.
Im Detail arbeitet die Funktion \texttt{findValuation} wie folgt.
\begin{enumerate}
\item Zunächst berechnen wir in Zeile 41 die Menge aller aussagenlogischen Variablen, die in der Menge
      \texttt{Clauses} auftreten.  Wir benötigen diese Menge, denn in der
      aussagenlogischen Interpretation, die wir als Ergebnis zurück geben wollen, müssen 
      wir diese Variablen auf die Menge $\{ \texttt{True}, \texttt{False} \}$ abbilden.       
\item In Zeile 42 saturieren wir die Menge \texttt{Clauses} und berechnen alle Klauseln, die aus der
      ursprünglich gegebenen Menge von Klauseln mit Hilfe der Schnitt-Regel hergeleitet werden
      können.  Hier können zwei Fälle auftreten:
      \begin{enumerate}
      \item Falls die leere Klausel hergeleitet werden kann, dann folgt aus der Korrektheit der Schnitt-Regel,
            dass die ursprünglich gegebene Menge von Klauseln widersprüchlich ist und wir geben als Ergebnis an
            Stelle einer Belegung den Wert \texttt{False} zurück, denn eine widersprüchliche Menge von Klauseln
            ist sicher nicht erfüllbar.
      \item Andernfalls berechnen wir nun eine aussagenlogische Belegung, unter der alle Klauseln aus
            der Menge \texttt{Clauses} wahr werden.  Zu diesem Zweck berechnen wir zunächst eine Menge von
            Literalen, die wir in der Variablen \texttt{Literals} abspeichern.  Die Idee ist dabei, dass wir
            die aussagenlogische Variable \texttt{p} genau dann in die Menge \texttt{Literals} aufnehmen, wenn
            die gesuchte Belegung $\I$ die aussagenlogische
            Variable \texttt{p} zu \texttt{True} auswertet.  Andernfalls nehmen wir an Stelle von \texttt{p} das
            Literal $\neg \texttt{p}$ in der Menge \texttt{Literals} auf.  Als Ergebnis geben wir daher in
            Zeile 53 die Menge \texttt{Literals} zurück.  Die gesuchte aussagenlogische Belegung
            $\mathcal{I}$ kann dann gemäß der Formel  
            \\[0.2cm]
            \hspace*{1.3cm}
            $\mathcal{I}(\texttt{p}) = \left\{
             \begin{array}{ll}
               \texttt{True}  & \mbox{falls $\hspace*{0.25cm}\texttt{p} \in \texttt{Literals}$} \\
               \texttt{False} & \mbox{falls $\neg\texttt{p} \in \texttt{Literals}$}
             \end{array}\right.
            $
            \\[0.2cm]
            berechnet werden.
      \end{enumerate}
\item Die Berechnung der Menge \texttt{Literals} erfolgt nun über eine \texttt{for}-Schleife.
      Dabei ist der Gedanke, dass wir für eine aussagenlogische Variable \texttt{p} genau dann das Literal
      $\texttt{p}$ zu der Menge \texttt{Literals} hinzufügen, wenn die Belegung $\I$ die Variable \texttt{p}
      auf \texttt{True} abbilden \underline{muss}, um die Klauseln zu erfüllen.  Andernfalls fügen wir
      stattdessen das Literal $\neg\texttt{p}$ zu dieser Menge hinzu.

      Die Bedingung dafür, dass wir das Literal $p$ hinzufügen müssen ist wie folgt:
      Angenommen, wir haben bereits Werte für die Variablen
      $p_1$, $\cdots$, $p_n$ in der Menge \texttt{Literals}  gefunden.
      Die Werte dieser Variablen seien durch die Literale $l_1$, $\cdots$, $l_n$ in der Menge \texttt{Literals}
      wie folgt festgelegt: Wenn $l_i = p_i$ ist, dann gilt $\I(p_i) = \texttt{True}$ 
      und falls $l_i = \neg p_i$ gilt, so haben wir $\I(p_i) = \texttt{False}$.
      Nehmen wir nun weiter an, dass eine Klausel $C$ in der Menge \texttt{Clauses} existiert, so dass
      \\[0.2cm]
      \hspace*{1.3cm}
      $C \backslash \{p\} \subseteq \{ \komplement{l_1}, \cdots, \komplement{l_n} \}$ \quad und \quad $\texttt{p} \in C$
      \\[0.2cm]
      gilt.  Wenn $\I(C) = \texttt{True}$ gelten soll, dann muss $\I(\texttt{p}) = \texttt{True}$ gelten, denn
      nach Konstruktion von $\I$ gilt 
      \\[0.2cm]
      \hspace*{1.3cm}
      $\I\bigl(\komplement{l_i}) = \texttt{False}$ \quad für alle $i \in \{1,\cdots,n\}$
      \\[0.2cm]
      und damit ist \texttt{p} das einzige Literal in der Klausel $C$, das wir mit Hilfe der Belegung $\I$
      überhaupt noch wahr machen können.  In diesem Fall fügen wir also das Literal
      $\texttt{p}$ in die Menge \texttt{Literals} ein.  Andernfalls wird das Literal $\neg p$ zu der Menge
      \texttt{Literals} hinzugefügt.
\end{enumerate}
Die Definition der Funktion \texttt{findValuation} setzt die induktive Definition der Belegungen
$\mathcal{I}_k$ um, die wir im Beweis der Widerlegungs-Vollständigkeit angegeben haben.

% Der entscheidende Punkt ist nun der Nachweis, dass die Funktion \texttt{findValuation} in dem Falle,
% dass in Zeile 46 nicht der Wert \texttt{False} zurück gegeben wird, eine aussagenlogische Belegung $\I$ berechnet, bei der
% alle Klauseln aus der Menge \texttt{Clauses} den Wert \texttt{True} erhalten.  Um diesen Nachweis zu
% erbringen, nummerieren wie die aussagenlogischen Variablen, die in der Menge \texttt{Clauses}
% auftreten, in derselben Reihenfolge durch, in der diese Variablen in der \texttt{for}-Schleife in Zeile
% 48 betrachtet werden.  Wir bezeichnen diese Variablen als
% \\[0.2cm]
% \hspace*{1.3cm}
%  $p_1, p_2, p_3, \cdots, p_k$
% \\[0.2cm]
% und zeigen durch Induktion nach $n$, dass nach $n$ Durchläufen der Schleife für jede Klausel $D \in \texttt{Clauses}$, 
% in der nur die Variablen $p_1$, $\cdots$, $p_n$ vorkommen, 
% \\[0.2cm]
% \hspace*{1.3cm}
% $\I(D) = \texttt{True}$
% \\[0.2cm]
% gilt.
% \begin{enumerate}
% \item[I.A.:] $n = 1$.

%              In diesem Fall muss entweder
%              \\[0.2cm]
%              \hspace*{1.3cm}
%              $D = \{ p \}$ \quad oder \quad $D = \{ \neg p \}$
%              \\[0.2cm]
%              gelten.  An dieser Stelle brauchen wir eine Fallunterscheidung.
%              \begin{enumerate}
%              \item $D = \{ p \}$.

%                    Daraus folgt aber sofort
%                    \\[0.2cm]
%                    \hspace*{1.3cm}
%                    $D \backslash \{p\} = \{\} \subseteq \{\komplement{l} \mid l \in \texttt{Literals} \}$.
%                    \\[0.2cm]
%                    Also ist die Bedingung in Zeile 50 erfüllt und wir haben $\texttt{p} \in
%                    \texttt{Literals}$.
%                    Damit gilt $\I(p) = \texttt{True}$ nach Definition von $\I$.
%              \item $D = \{ \neg p \}$.

%                    Würde es jetzt eine Klausel $E = \{ p\} \in \texttt{Clauses}$ geben, so könnten
%                    wir aus den beiden Klauseln $D$ und $E$ sofort die leere Klausel $\{\}$
%                    herleiten und die Funktion \texttt{findValuation} würde in Zeile 46 den Wert
%                    \texttt{False} zurück geben.  Da wir aber vorausgesetzt haben, dass dies nicht
%                    passiert, kann es keine solche Klausel $E$ geben.  Damit ist die Bedingung in
%                    Zeile 49 falsch und folglich gilt $\neg p \in \texttt{Literals}$.
%                    Nach Definition von $\I$ folgt dann $\I(\neg p) = \texttt{True}$.
%              \end{enumerate}
%              Damit haben wir in jedem Fall $\I(D) = \texttt{True}$.   
% \item[I.S.:] $n \mapsto n+1$.

%              Wir setzen nun voraus, dass die Behauptung vor dem $(n\!+\!1)$-ten Durchlauf der
%              \texttt{for}-Schleife gilt und haben zu zeigen, dass die Behauptung dann auch nach
%              diesem Durchlauf erfüllt ist.  Sei dazu $D$ eine Klausel, in der nur die Variablen
%              $p_1$, $\cdots$, $p_n$, $p_{n+1}$ vorkommen.  Die Klausel ist dann eine Teilmenge einer
%              Menge der Form
%              \\[0.2cm]
%              \hspace*{1.3cm}
%              $\{ l_1, \cdots, l_n, l_{n+1} \}$, \quad wobei $l_i \in \{ p_i, \neg p_i \}$ für alle
%              $i \in \{1,\cdots, n+1\}$ gilt.
%              \\[0.2cm]
%              Nun gibt es mehrere Möglichkeiten, die wir getrennt untersuchen.
%              \begin{enumerate}
%              \item Es gibt ein $i \in \{1,\cdots,n\}$, so dass $l_i \in D$ und  $\I(l_i) =
%                \texttt{True}$ ist.  

%                    Da eine Klausel als Disjunktion ihrer Literale aufgefasst wird, gilt dann auch
%                    $\I(D) = \texttt{True}$ unabhängig davon, ob $\I(p_{n+1})$ den Wert \texttt{True} oder
%                    \texttt{False} hat.
%              \item Für alle $i \in \{1,\cdots,n\}$ mit $l_i \in D$ gilt $\I(l_i) = \texttt{False}$ und es gilt $l_{n+1} = p_{n+1}$.
                   
%                    Dann gilt für die Klausel $D$ gerade die Bedingung
%                    \\[0.2cm]
%                    \hspace*{1.3cm}
%                    $C \backslash \{ p_{n+1} \} \subseteq \bigl\{ \komplement{l_1}, \cdots, \komplement{l_n} \bigr\}$
%                    \quad und \quad $p_{n+1} \in C$
%                    \\[0.2cm]
%                    und daher wird in Zeile 47 der Funktion \texttt{findValuation} das Literal $p_{n+1}$ 
%                    zu der Menge \texttt{Literals} hinzugefügt.  Nach Definition der Belegung $\I$, 
%                    die von der Funktion \texttt{findValuation} zurück gegeben wird, heißt dies
%                    gerade, dass 
%                    \\[0.2cm]
%                    \hspace*{1.3cm}
%                    $\I(p_{n+1}) = \texttt{True}$
%                    \\[0.2cm]
%                    ist und dann gilt natürlich auch $\I(D) = \texttt{True}$.
%              \item Für alle $i \in \{1,\cdots,n\}$ mit $l_i \in D$ gilt $\I(l_i) = \texttt{False}$ und es gilt $l_{n+1} = \neg p_{n+1}$.

%                    An dieser Stelle ist eine weitere Fall-Unterscheidung notwendig.
%                    \begin{enumerate}
%                    \item Es gibt eine weitere Klausel $C$ in der Menge \texttt{Clauses}, so dass
%                          \\[0.2cm]
%                          \hspace*{1.3cm}
%                          $C \backslash \{ p_{n+1} \} \subseteq \bigl\{ \komplement{l_1}, \cdots,
%                          \komplement{l_n} \bigr\}$ \quad und \quad $p_{n+1} \in C$
%                          \\[0.2cm]
%                          gilt.  Hier sieht es zunächst so aus, als ob wir ein Problem hätten, denn
%                          in diesem Fall würde um die Klausel $C$ wahr zu machen das Literal $p_{n+1}$ zur Menge
%                          \texttt{Literals} hinzugefügt und damit wäre zunächst $\I(p_{n+1}) = \texttt{True}$ 
%                          und damit $\I(\neg p_{n+1}) = \texttt{False}$, woraus insgesamt 
%                          $\I(D) = \texttt{False}$ folgern würde.  In diesem Fall würden sich
%                          die Klauseln $C$ und $D$  in der Form
%                          \\[0.2cm]
%                          \hspace*{1.3cm}
%                          $C = C' \cup \{p_{n+1}\}$, \quad $D = D' \cup \{ \neg p_{n+1} \}$
%                          \\[0.2cm]
%                          schreiben lassen, wobei 
%                          \\[0.2cm]
%                          \hspace*{1.3cm}
%                          $C' \subseteq \Bigl\{ \komplement{l} \mid l \in \texttt{Literals} \Bigr\}$  \quad und \quad
%                          $D' \subseteq \Bigl\{ \komplement{l} \mid l \in \texttt{Literals} \Bigr\}$
%                          \\[0.2cm]
%                          gelten würde.  Daraus würde sowohl
%                          \\[0.2cm]
%                          \hspace*{1.3cm}
%                          $\I(C') = \texttt{False}$ \quad als auch \quad $\I(D') = \texttt{False}$
%                          \\[0.2cm]
%                          folgen und das würde auch
%                          \\[0.2cm]
%                          \hspace*{1.3cm}
%                          $\I(C' \cup D') = \texttt{False}$ \hspace*{\fill} $(*)$
%                          \\[0.2cm]
%                          implizieren.
%                          Die entscheidende Beobachtung ist nun, dass die Klausel $C' \cup D'$ mit
%                          Hilfe der Schnitt-Regel aus den beiden Klauseln
%                          \\[0.2cm]
%                          \hspace*{1.3cm}
%                          $C = C' \cup \{p_{n+1}\}$, \quad $D = D' \cup \{ \neg p_{n+1} \}$, 
%                          \\[0.2cm]
%                          gefolgert werden kann.  Das heißt dann aber, dass die Klausel $C' \cup D'$ ein
%                          Element der Menge \texttt{Clauses} sein muss, denn die Menge
%                          \texttt{Clauses} ist ja saturiert!  Da die Klausel $C' \cup D'$ außerdem
%                          nur die aussagenlogischen Variablen $p_1, \cdots, p_n$ enthält, gilt nach
%                          Induktions-Voraussetzung 
%                          \\[0.2cm]
%                          \hspace*{1.3cm}
%                          $\I(C' \cup D') = \texttt{True}$.
%                          \\[0.2cm]
%                          Dies steht aber im Widerspruch zu $(*)$.  Dieser Widerspruch zeigt, dass 
%                          es keine Klausel $C \in \texttt{Clauses}$ mit 
%                          \\[0.2cm]
%                          \hspace*{1.3cm}
%                          $C \subseteq \bigl\{ \komplement{l} \mid l \in \texttt{Literals} \bigr\} \cup
%                          \{p_{n+1}\}$ 
%                          \quad und \quad $p_{n+1} \in C$
%                          \\[0.2cm]
%                          geben kann und damit tritt der hier untersuchte Fall gar nicht auf. 
%                    \item Es gibt \underline{keine} Klausel $C$ in der Menge \texttt{Clauses}, so dass
%                          \\[0.2cm]
%                          \hspace*{1.3cm}
%                          $C \subseteq \bigl\{ \komplement{l} \mid l \in \texttt{Literals} \bigr\} \cup \{p_{n+1}\}$ 
%                          \quad und \quad $p_{n+1} \in C$
%                          \\[0.2cm]
%                          gilt.  In diesem Fall wird das Literal $\neg p_{n+1}$ zur Menge \texttt{Literals}
%                          hinzugefügt und damit gilt zunächst $\I(p_{n+1}) = \texttt{False}$ und folglich
%                          $\I(\neg p_{n+1}) = \texttt{True}$, woraus schließlich $\I(D) = \texttt{True}$ folgt.
%                    \end{enumerate}
%                    Wir sehen, dass der erste Fall der vorherigen Fall-Unterscheidung nicht
%                    auftritt und dass im zweiten Fall $\I(D) = \texttt{True}$ gilt, womit wir insgesamt 
%                    $\I(D) = \texttt{True}$ gezeigt haben.  Damit ist der Induktions-Schritt
%                    abgeschlossen.
%              \end{enumerate}
%              Da jede Klausel $C \in \texttt{Clauses}$ nur eine endliche Anzahl von Variablen
%              enthält, haben wir insgesamt gezeigt, dass für alle diese Klauseln 
%              $\I(C) = \texttt{True}$ gilt. \qed
% \end{enumerate}

\section{Das Verfahren von Davis und Putnam}
In der Praxis stellt sich oft die Aufgabe, für eine gegebene Menge von Klauseln $K$ eine aussagenlogische
Belegung $\I$ zu berechnen, so dass 
\\[0.2cm]
\hspace*{1.3cm} $\texttt{evaluate}(C,\I) = \texttt{True}$ \quad für alle $C\in K$ \\[0.2cm]
gilt.  In diesem Fall sagen wir auch, dass die Belegung $\I$ eine \blue{Lösung} der
Klausel-Menge $K$ ist.  Im letz\-ten Abschnitt haben wir bereits die Funktion \texttt{findValuation}
kennengelernt, mit der wir eine solche Belegung berechnen könnten.
Bedauerlicherweise ist diese Funktion für eine praktische Anwendung nicht effizient genug, denn das Saturieren
einer Klausel-Menge ist im Allgemeinen sehr aufwendig.
Wir werden daher in diesem Abschnitt ein Verfahren vorstellen, mit dem die Berechnung einer Lösung
einer aussagenlogischen Klausel-Menge in vielen praktisch relevanten Fällen auch dann möglich ist, wenn die
Anzahle der Variablen groß ist.  Dieses Verfahren geht auf Davis und Putnam
\cite{davis:1960, davis:1962} zurück.  Verfeinerungen dieses Verfahrens \cite{moskewicz:2001} werden beispielsweise
eingesetzt, um die Korrektheit digitaler elektronischer Schaltungen nachzuweisen.  

Um das Verfahren zu motivieren, überlegen wir zunächst, bei welcher Form der Klausel-Menge $K$
unmittelbar klar ist, ob es eine Belegung gibt, die $K$ löst und wie diese Belegung
aussieht.  Betrachten wir dazu ein Beispiel: \\[0.2cm]
\hspace*{1.3cm} 
$K_1 = \bigl\{\; \{p\},\; \{\neg q\},\; \{r\},\; \{\neg s\}, \; \{\neg t\} \;\bigr\}$ 
\\[0.2cm]
Die Klausel-Menge $K_1$ entspricht der aussagenlogischen Formel
\\[0.2cm]
\hspace*{1.3cm}
$p \wedge \neg q \wedge r \wedge \neg s \wedge \neg t$.
\\[0.2cm]
Daher ist $K_1$ lösbar und die Belegung  \\[0.2cm]
\hspace*{1.3cm} 
$\I = \bigl\{\; \pair(p, \texttt{True}),\; \pair(q, \texttt{False}),\;\pair(r, \texttt{True}),\; \pair(s, \texttt{False}),\; \pair(t, \texttt{False})\;\}$
\\[0.2cm]
ist eine Lösung.  Betrachten wir eine weiteres Beispiel: \\[0.2cm]
\hspace*{1.3cm} 
$K_2 = \bigl\{\; \{\}, \{p\},\; \{\neg q\},\; \{r\}\; \bigr\}$ 
\\[0.2cm]
Diese Klausel-Menge entspricht der Formel
\\[0.2cm]
\hspace*{1.3cm}
$\falsum \wedge p \wedge \neg q \wedge r$.
\\[0.2cm]
Offensichtlich ist $K_2$ unlösbar.  Als letztes Beispiel betrachten wir 
\\[0.2cm]
\hspace*{1.3cm} $K_3 = \bigl\{ \{p\}, \{\neg q\}, \{\neg p\} \bigr\}$.
\\[0.2cm]
Diese Klausel-Menge kodiert die Formel
\\[0.2cm]
\hspace*{1.3cm}
$p \wedge \neg q \wedge \neg p $
\\[0.2cm]
und ist offenbar  ebenfalls unlösbar, denn eine Lösung $\I$ müsste die aussagenlogische Variable $p$ gleichzeitig
wahr und falsch machen.  Wir nehmen die an den letzten drei Beispielen gemachten Beobachtungen zum Anlass für zwei Definitionen.

\begin{Definition}[Unit-Klausel]
  Eine Klausel $C$ ist eine \blue{Unit-Klausel}\index{Unit-Klausel}, wenn $C$ nur aus einem Literal besteht.
  Es gilt dann entweder
  \\[0.2cm]
  \hspace*{1.3cm}
  $C = \{p\}$ \quad oder \quad $C = \{\neg p\}$ 
  \\[0.2cm]
  für eine Aussage-Variable $p$. \eox
\end{Definition}

\begin{Definition}[Triviale Klausel-Mengen]
  Eine Klausel-Menge $K$ ist genau dann eine \blue{triviale Klausel-Menge}\index{triviale Klausel-Menge}, wenn
  einer der beiden folgenden Fälle vorliegt:
  \begin{enumerate}
  \item $K$ enthält die leere Klausel, es gilt also $\{\} \el K$.

        In diesem Fall ist $K$ offensichtlich unlösbar.
  \item $K$ enthält nur Unit-Klauseln mit \underline{\red{verschiedenen}} Aussage-Variablen.
        Bezeichnen wir die Menge der aussagenlogischen Variablen mit $\mathcal{P}$,
        so schreibt sich diese Bedingung als 
        \begin{enumerate}[(a)]
        \item $\texttt{card}(C) = 1$ \quad für alle $C \in K$ \quad und
        \item Es gibt kein $p\in\mathcal{P}$, so dass gilt: 
              \\[0.2cm]
              \hspace*{1.3cm}
              $\{p\} \in K$ \quad und \quad $\{\neg p\} \in K\bigr)$.
        \end{enumerate}
        In diesem Fall können wir die aussagenlogische Belegung $\I$ wie folgt definieren:
        \\[0.2cm]
        \hspace*{1.3cm}
        $\I(p) = \left\{
                   \begin{array}{lcr}
                     \texttt{True}  & \mbox{falls} & \{p\}      \in K, \\[0.1cm]
                     \texttt{False} & \mbox{falls} & \{\neg p\} \in K.
                   \end{array}
                   \right.
        $
        \\[0.2cm]
        Damit ist $\mathcal{I}$ dann eine \blue{Lösung}\index{Lösung} der Klausel-Menge $K$. \eox
  \end{enumerate}
\end{Definition}

Wie können wir nun eine gegebene Klausel-Menge in eine triviale Klausel-Menge umwandeln?
Es gibt drei Möglichkeiten, Klauselmengen zu vereinfachen.  Die erste der beiden Möglichkeiten kennen wir
schon, die anderen beiden Möglichkeiten werden wir später näher erläutern.
\begin{enumerate}
\item \blue{Schnitt-Regel},
\item \blue{Subsumption} und
\item \blue{Fallunterscheidung}.
\end{enumerate}
Wir betrachten diese Möglickeiten jetzt der Reihe nach.

\subsection{Vereinfachung mit der Schnitt-Regel}
Eine typische Anwendung der Schnitt-Regel hat die Form: \\[0.2cm]
\hspace*{1.3cm} $\schluss{ C_1 \cup \{p\} \quad \{\neg p\} \cup C_2}{C_1 \cup C_2}$
\\[0.2cm]
Die hierbei erzeugte Klausel $C_1 \cup C_2$ wird in der Regel mehr Literale enthalten
als die Prämissen $C_1 \cup \{p\}$ und $\bigl\{\neg p\} \cup C_2$.  Enthält die
Klausel $C_1 \cup \{p\}$ insgesamt $m+1$ Literale und enthält die Klausel
$\bigl\{\neg p\} \cup C_2$ insgesamt $n+1$ Literale, so kann die Konklusion $C_1 \cup C_2$ 
bis zu $m + n$ Literale enthalten.  Natürlich können es auch weniger Literale 
sein, und zwar dann, wenn es Literale gibt, die sowohl in $C_1$ als auch in $C_2$
auftreten.  Oft ist $m + n$ aber sowohl größer als $m + 1$ als auch größer als $n + 1$.  Die
Klauseln wachsen nur dann sicher nicht, wenn  $n = 0$ oder $m = 0$ ist.
Dieser Fall liegt vor, wenn einer der beiden Klauseln nur aus einem Literal besteht
und folglich eine \blue{Unit-Klausel} ist.  Da es unser Ziel ist, die Klausel-Mengen
zu vereinfachen, lassen wir nur solche Anwendungen der Schnitt-Regel zu, bei denen
eine der Klausel eine Unit-Klausel ist.  Solche Schnitte bezeichnen wir als
\blue{Unit-Schnitte}.\index{Unit-Schnitt}  Um alle mit einer gegebenen Unit-Klausel $\{l\}$ möglichen Schnitte
durchführen zu können, definieren wir eine Funktion
\\[0.2cm]
\hspace*{1.3cm}
$\texttt{unitCut}: 2^\mathcal{K} \times \mathcal{L} \rightarrow 2^\mathcal{K}$\index{\texttt{unitCut}}
\\[0.2cm]
so, dass für eine Klausel-Menge $K$ und ein Literal $l$ die Funktion
$\texttt{unitCut}(K,l)$ die Klausel-Menge $K$ soweit wie möglich mit Unit-Schnitten mit der Klausel
$\{l\}$ vereinfacht:
\\[0.2cm]
\hspace*{1.3cm}
$\texttt{unitCut}(K,l) = \Bigl\{ C \backslash \bigl\{ \komplement{l} \bigr\} \;\Big|\; C \in K \Bigr\}$.
\\[0.2cm]
Beachten Sie, dass die Menge $\texttt{unitCut}(K,l)$ genauso viele Klauseln enthält wie die Menge
$K$.  Allerdings sind diejenigen Klauseln aus der Menge $K$, die das Literal $\komplement{l}$
enthalten, verkleinert worden.   Alle anderen Klauseln aus $K$ bleiben unverändert.

Eine Klauselmenge $K$ werden wir nur dann mit Hilfe des Ausdrucks $\texttt{unitCut}(K, l)$ vereinfachen, wenn 
die Unit-Klausel $\{l\}$ ein Element der Menge $K$ ist.

\subsection{Vereinfachung durch Subsumption}
Das Prinzip der Subsumption demonstrieren wir zunächst an einem Beispiel.
Wir betrachten \\[0.2cm]
\hspace*{1.3cm} $K = \bigl\{ \{p, q, \neg r\}, \{p\} \bigr\} \cup M$. \\[0.2cm]
Offenbar impliziert die Klausel $\{p\}$ die Klausel $\{p, q, \neg r\}$, denn immer wenn
$\{p\}$ erfüllt ist, ist automatisch auch $\{q, p, \neg r\}$ erfüllt.  Das liegt daran, dass 
\\[0.2cm]
\hspace*{1.3cm} $\models p \rightarrow q \vee p \vee \neg r$
\\[0.2cm]
gilt.  Allgemein sagen wir, dass eine Klausel $C$
 von einer Unit-Klausel $U$ \blue{subsumiert}\index{subsumiert} wird, wenn
\\[0.2cm]
\hspace*{1.3cm} $U \subseteq C$ \\[0.2cm]
gilt.  Ist $K$ eine Klausel-Menge mit $C \in K$ und $U \in K$ und wird
$C$ durch $U$ subsumiert, so können wir die Menge $K$ durch Unit-Subsumption zu der Menge $K - \{ C \}$
verkleinern, wir können also die Klausel $C$ aus $K$ löschen.  Dazu definieren wir eine Funktion
\\[0.2cm]
\hspace*{1.3cm}
$\texttt{subsume}: 2^\mathcal{K} \times \mathcal{L} \rightarrow 2^\mathcal{K}$,\index{\texttt{subsume}}
\\[0.2cm]
die eine gegebene Klauselmenge $K$, welche die Unit-Klausel $\{l\}$ enthält, mittels Subsumption 
dadurch vereinfacht, dass alle durch $\{l\}$ subsumierten Klauseln aus $K$ gelöscht werden.
Die Unit-Klausel $\{l\}$ selbst behalten wir natürlich.  Daher definieren wir:
\\[0.2cm]
\hspace*{1.3cm}
$\texttt{subsume}(K, l) := 
\bigl(K \backslash \bigl\{ C \in K \mid l \in C \bigr\}\bigr) \cup \bigl\{\{l\}\bigr\} = 
\bigl\{ C \in K \mid l \not\in C \bigr\} \cup \bigl\{\{l\}\bigr\}$.
\\[0.2cm]
In der obigen Definition muss die Unit-Klausel $\{l\}$ in das Ergebnis eingefügt werden, weil die Menge
\mbox{$\bigl\{ C \in K \mid l \not\in C \bigr\}$} die Unit-Klausel $\{l\}$ nicht enthält.  Die beiden Klausel-Mengen
$\mathtt{subsume}(K,l)$ und $K$ sind genau dann äquivalent, wenn $\{l\} \in K$ gilt.  
Eine Klauselmenge $K$ werden wir daher nur dann mit Hilfe des Ausdrucks $\texttt{subsume}(K, l)$ vereinfachen, wenn 
die Unit-Klausel $\{l\}$ in der Menge $K$ ent\-halten ist.

\subsection{Vereinfachung durch Fallunterscheidung}
Ein Kalkül, der nur mit Unit-Schnitten und Subsumption arbeitet, ist nicht 
widerlegungs-vollständig.  Wir brauchen 
daher eine weitere Verfahren, Klausel-Mengen zu vereinfachen.
Eine solches Vereinfachen ist die \blue{Fallunterscheidung}.\index{Fallunterscheidung}  Dieses Prinzip basiert
auf dem folgenden Satz.

\begin{Satz}
  Ist $K$ eine Menge von Klauseln und ist $p$ eine aussagenlogische Variable, 
  so ist $K$ genau dann erfüllbar, wenn $K \cup \bigl\{\{p\}\bigr\}$ oder 
  $K \cup \bigl\{\{\neg p\}\bigr\}$ erfüllbar ist.  
\end{Satz}

\noindent
\textbf{Beweis}:
\begin{enumerate}
\item[``$\Rightarrow$'':] 
  Ist $K$ erfüllbar durch eine Belegung $\I$, so gibt es für  $\I(p)$ zwei Möglichkeiten, denn 
  $\mathcal{I}(p)$ ist entweder wahr oder falsch.  Falls $\I(p) = \texttt{True}$ ist, ist damit auch die Menge $K \cup
  \bigl\{\{p\}\bigr\}$ erfüllbar,
  andernfalls ist $K \cup \bigl\{\{\neg p\}\bigr\}$
  erfüllbar. 
\item[``$\Leftarrow$'':] 
  Da $K$ sowohl eine Teilmenge von $K \cup \bigl\{\{p\}\bigr\}$ als auch von 
  $K \cup \bigl\{\{\neg p\}\bigr\}$ ist, ist klar, dass $K$ erfüllbar
  ist, wenn eine dieser Mengen erfüllbar sind.  
\qed
\end{enumerate}

Wir können nun eine Menge $K$ von Klauseln dadurch vereinfachen, dass wir eine
aussagenlogische Variable $p$ wählen, die in $K$ vorkommt.
Anschließend bilden wir die Mengen \\[0.2cm]
\hspace*{1.3cm} $K_1 := K \cup \bigl\{\{p\}\bigr\}$ \quad und \quad $K_2 := K \cup
\bigl\{\{\neg p\}\bigr\}$
\\[0.2cm]
und untersuchen rekursiv ob $K_1$ erfüllbar ist.  Falls wir eine Lösung für $K_1$ finden,
ist dies auch eine Lösung für die ursprüngliche Klausel-Menge $K$ und wir haben unser Ziel
erreicht.
Andernfalls untersuchen wir rekursiv ob $K_2$ erfüllbar ist.
Falls wir eine Lösung finden, ist dies auch eine Lösung von $K$.  Wenn wir weder
für $K_1$ noch für $K_2$ eine Lösung finden, dann kann auch $K$ keine Lösung haben,
denn jede Lösung $\mathcal{I}$ von $K$ muss die Variable $p$ entweder wahr oder falsch machen.
Die rekursive Untersuchung von $K_1$ bzw.~$K_2$ ist leichter als die Untersuchung von $K$,
weil wir ja in $K_1$ und $K_2$ mit den Unit-Klausel $\{p\}$ bzw.~$\{\neg p\}$
sowohl Unit-Subsumptionen als auch Unit-Schnitte durchführen können und dadurch diese Mengen vereinfacht
werden. 


\subsection{Der Algorithmus}
Wir können jetzt den Algorithmus von Davis und Putnam \index{Davis-Putnam Algorithmus} skizzieren.
Gegeben sei eine Menge $K$ von Klauseln.  Gesucht ist dann eine Lösung von $K$.  Wir
suchen  also eine Belegung $\I$, so dass gilt: \\[0.2cm]
\hspace*{1.3cm} $\I(C) = \texttt{True}$ \quad für alle $C \in K$.\\[0.2cm]
Das Verfahren von Davis und Putnam besteht nun aus den folgenden Schritten.
\begin{enumerate}
\item Führe alle Unit-Schnitte und Unit-Subsumptionen aus, die mit Klauseln aus $K$ möglich sind.
\item Falls $K$ jetzt trivial ist, endet das Verfahren.
     \begin{enumerate}
     \item Wenn $\{\} \in K$ ist, dann ist $K$ unlösbar.
     \item Andernfalls ist die aussagenlogische Interpretation 
           \\[0.2cm]
           \hspace*{1.3cm}
           $\mathcal{I} := \bigl\{ p \mapsto \mathtt{True}  \mid p \in \mathcal{P} \wedge \{p\}\in K\bigr\} \cup
                           \bigl\{ p \mapsto \mathtt{False} \mid p \in \mathcal{P} \wedge \{\neg p\}\in K\bigr\}  $
           \\[0.2cm]
           eine Lösung von $K$.     
     \end{enumerate}
\item Falls $K$ nicht trivial ist, wählen wir eine aussagenlogische Variable $p$, die in $K$ auftritt.
      \begin{enumerate}
      \item Jetzt versuchen  wir rekursiv,  die Klausel-Menge \\[0.2cm]
            \hspace*{1.3cm}  $K \cup \bigl\{\{p\}\bigr\}$ \\[0.2cm]
            zu lösen. Falls diese gelingt, haben wir eine Lösung von $K$.
      \item Falls $K \cup \bigl\{\{p\}\bigr\}$ nicht lösbar ist, versuchen wir stattdessen, die Klausel-Menge
            \\[0.2cm] 
            \hspace*{1.3cm} $K \cup \bigl\{\{\neg p\}\bigr\}$
            \\[0.2cm]
            zu lösen.  Wenn auch dies fehlschlägt, ist $K$ unlösbar.  Andernfalls
            haben wir eine Lösung von $K$.
      \end{enumerate}
\end{enumerate}
Für die Implementierung ist es zweckmäßig, die beiden oben definierten Funktionen $\texttt{unitCut}()$ und
$\texttt{subsume}()$ zu einer Funktion zusammen zu fassen.  Wir definieren daher die Funktion
\\[0.2cm]
\hspace*{1.3cm}
$\texttt{reduce}: 2^\mathcal{K} \times \mathcal{L} \rightarrow 2^\mathcal{K}$\index{\texttt{reduce}}
\\[0.2cm]
wie folgt: 
\\[0.2cm]
\hspace*{1.3cm}
$\texttt{reduce}(K,l)  = 
 \Bigl\{\, C \backslash \bigl\{\komplement{l}\bigr\} \;|\; C \in K \wedge \komplement{l} \in C \,\Bigr\} 
       \,\cup\, \Bigl\{\, C \in K \mid \komplement{l} \not\in C \wedge l \not\in C \Bigr\} \cup \bigl\{\{l\}\bigr\}.
$
\\[0.2cm]
Die Menge enthält also einerseits die Ergebnisse von Schnitten mit
der Unit-Klausel $\{l\}$ und andererseits nur die Klauseln $C$,
die mit $l$ nichts zu tun haben, weil weder $l \in C$ noch $\komplement{l} \in C$
gilt.  Außerdem fügen wir noch die Unit-Klausel $\{l\}$ hinzu.
Dadurch erreichen wir, dass die beiden Mengen $K$ und $\texttt{reduce}(K,l)$
logisch äquivalent sind, falls $\{l\} \in K$ gilt.
Wir werden daher $K$ nur dann durch $\mathtt{reduce}(K, l)$ ersetzen, wenn $\{l\} \in K$ ist.

\subsection{Ein Beispiel}
Zur Veranschaulichung demonstrieren wir das Verfahren von Davis und Putnam an einem Beispiel.
Die Menge $K$ sei wie folgt definiert: \\[0.2cm]
\hspace*{-0.3cm}
 $K := \Big\{ \{p, q, s\},\; \{\neg p, r, \neg t\},\;  \{r, s\},\; \{\neg r, q, \neg p\}, 
               \{\neg s, p\},\; \{\neg p, \neg q, s, \neg r\},\; \{p, \neg q, s\},\; \{\neg r, \neg s\},\;
             \{\neg p, \neg s\} 
        \Big\}$. 
\\[0.2cm]
Wir zeigen nun mit dem Verfahren von Davis und Putnam, dass $K$ nicht lösbar ist.  Da die
Menge $K$ keine Unit-Klauseln enthält, ist im ersten Schritt nichts zu tun.  Da $K$ nicht
trivial ist, sind wir noch nicht fertig.  Also gehen wir jetzt zu Schritt 3 und wählen
eine aussagenlogische Variable, die in $K$ auftritt.  An dieser Stelle ist es sinnvoll
eine Variable zu wählen, die in möglichst vielen Klauseln von $K$ auftritt.  Wir wählen
daher die aussagenlogische Variable $p$.
\begin{enumerate}
\item Zunächst bilden wir die Menge \\[0.2cm]
      \hspace*{1.3cm} $K_0 := K \cup \bigl\{ \{p\} \bigr\}$       \\[0.2cm]
      und versuchen, diese Menge zu lösen.  Dazu bilden wir \\[0.2cm]
      \hspace*{0.3cm} 
      $K_1 := \texttt{reduce}\bigl(K_0,p\bigr) = 
          \Big\{ \{r, \neg t\},\; \{r, s\},\; \{\neg r, q\},\; \{\neg q, s, \neg r\},\; \{\neg r, \neg s\},\; \{ \neg s\},\;\{p\}\, \Big\}$.
      \\[0.2cm]
      Die Klausel-Menge $K_1$ enthält die Unit-Klausel $\{\neg s\}$,
      so dass wir als nächstes mit dieser Klausel reduzieren können: \\[0.2cm]
      \hspace*{1.3cm} 
      $K_2 := \texttt{reduce}\bigl(K_1,\neg s\bigr) = 
              \Big\{ \{r, \neg t\},\; \{r\},\; \{\neg r, q\},\; \{\neg q, \neg r\},\; \{ \neg s\},\; \{p\} \Big\}$.
      \\[0.2cm]
      Hier haben wir nun die neue Unit-Klausel $\{r\}$, mit der wir weiter reduzieren:
      \\[0.2cm]
      \hspace*{1.3cm} 
      $K_3 := \textsl{reduce}\bigl(K_2, r\bigr) = 
              \Big\{ \{r\},\; \{q\},\; \{\neg q\},\; \{ \neg s\},\; \{p\} \Big\}$
      \\[0.2cm]
      Da $K_3$ die Unit-Klausel $\{q\}$ enthält, reduzieren wir jetzt mit $q$: \\[0.2cm]
      \hspace*{1.3cm} 
      $K_4 := \textsl{reduce}\bigl(K_2, q\bigr) = 
              \Big\{ \{r\},\; \{q\},\; \{\},\; \{ \neg s\},\; \{p\} \Big\}$.
      \\[0.2cm]
      Die Klausel-Menge $K_4$ enthält die leere Klausel und ist damit unlösbar.
     
\item Also bilden wir jetzt die Menge \\[0.2cm]
      \hspace*{1.3cm} $K_5 := K \cup \bigl\{ \{\neg p\} \bigr\}$ \\[0.2cm]
      und versuchen, diese Menge zu lösen.  Dazu bilden wir
      \\[0.2cm]
      \hspace*{1.3cm} 
      $K_6 = \textsl{reduce}\bigl(K_5, \neg p\bigr) =\Big\{ \{q, s\},\; \{r, s\},\;\{\neg s\},\; \{\neg q, s\},\; \{\neg r, \neg s\},\;\{\neg p\}\, \Big\}$.
      \\[0.2cm]
      Die Menge $K_6$ enthält die  Unit-Klausel $\{\neg s\}$.  Wir bilden daher \\[0.2cm]
      \hspace*{1.3cm} 
      $K_7 = \textsl{reduce}\bigl(K_6, \neg s\bigr) =\Big\{ \{q\},\; \{r\},\;\{\neg s\},\; \{\neg q\},\;\{\neg p\}\, \Big\}$.
      \\[0.2cm]
      Die Menge $K_7$ enthält die neue Unit-Klausel $\{q\}$, mit der wir als nächstes reduzieren:\\[0.2cm]
      \hspace*{1.3cm} 
      $K_8 = \textsl{reduce}\bigl(K_7, q \bigr) =\Big\{ \{q\},\; \{r\},\;\{\neg s\},\; \{\},\;\{\neg p\}\, \Big\}$.
      \\[0.2cm]
      Da $K_8$ die leere Klausel enthält, ist $K_8$ und damit auch die ursprünglich
      gegebene Menge $K$ unlösbar.
\end{enumerate}
Bei diesem Beispiel hatten wir Glück, denn wir mussten nur eine einzige Fallunterscheidung
durchführen. Bei komplexeren Beispielen ist es häufig so, dass wir innerhalb einer Fallunterscheidung eine
oder mehrere weitere Fallunterscheidungen durchführen müssen.

\subsection{Implementierung des Algorithmus von Davis und Putnam}
Wir zeigen jetzt die Implementierung der Funktion 
\href{https://github.com/karlstroetmann/Logic/blob/master/SetlX/davis-putnam.stlx}{\texttt{solve}}, 
mit der die Frage, ob eine Menge von Klauseln erfüllbar ist, beantwortet werden kann. Die
Implementierung ist in Abbildung \ref{fig:solve} auf Seite \pageref{fig:solve}
gezeigt.  Die Funktion erhält zwei Argumente: Die Mengen \texttt{Clauses} und \texttt{Variables}.
Hier ist \texttt{Clauses} eine Menge von Klauseln und \texttt{Variables} ist eine Menge von
Variablen.  Falls die Menge \texttt{Clauses} erfüllbar ist, so liefert
der Aufruf 
\\[0.2cm]
\hspace*{1.3cm}
\texttt{solve(Clauses, Variables)} \index{\texttt{solve}}
\\[0.2cm]
eine Menge von Unit-Klauseln \texttt{Result}, so
dass jede Belegung $\I$, die alle Unit-Klauseln aus \texttt{Result} erfüllt, auch alle Klauseln aus
der Menge  $\texttt{Clauses}$ erfüllt.  Falls die Menge $\texttt{Clauses}$ nicht erfüllbar ist, liefert der Aufruf
\\[0.2cm]
\hspace*{1.3cm}
\texttt{solve(Clauses, Variables)} 
\\[0.2cm]
als Ergebnis die Menge $\bigl\{ \{\} \bigr\}$ zurück, denn die leere Klausel repräsentiert die unerfüllbare Formel $\falsum$.

Sie fragen sich vielleicht, wozu wir in der Funktion \texttt{solve} die Menge
\texttt{Variables} brauchen.  Der Grund ist, dass wir uns bei den rekursiven Aufrufen
merken müssen, welche Variablen wir schon für Fallunterscheidungen benutzt haben.  Diese Variablen sammeln wir in der
Menge \texttt{Variables}.


\begin{figure}[!ht]
  \centering
\begin{minted}[ frame         = lines, 
                framesep      = 0.3cm, 
                numbers       = left,
                numbersep     = -0.2cm,
                bgcolor       = sepia,
                xleftmargin   = 0.2cm,
                xrightmargin  = 0.2cm
              ]{python3}
    def solve(Clauses, Variables):
        S      = saturate(Clauses)
        empty  = frozenset()
        Falsum = {empty}
        if empty in S:                  # S is inconsistent
            return Falsum           
        if all(len(C) == 1 for C in S): # S is trivial
            return S
        l       = selectLiteral(S, Variables)
        lBar    = complement(l)
        p       = extractVariable(l)
        newVars = Variables | { p }
        Result  = solve(S | { frozenset({l}) }, newVars)
        if not isFalsum(Result):
            return Result
        return solve(S | { frozenset({lBar}) }, newVars)
\end{minted}
\vspace*{-0.3cm}
  \caption{Die Funktion \texttt{solve}}
  \label{fig:solve}
\end{figure} 

Die in Abbildung \ref{fig:solve} gezeigte Implementierung funktioniert wie folgt:
\begin{enumerate}
\item In Zeile 2 reduzieren wir mit Hilfe der Methode \texttt{saturate} 
      solange wie möglich die gegebene Klausel-Menge \texttt{Clauses} mit Hilfe
      von Unit-Schnitten und entfernen alle Klauseln, die durch Unit-Klauseln
      subsumiert werden.
\item Anschließend testen wir in Zeile 5, ob die so vereinfachte Klausel-Menge \texttt{S}
      die leere Klausel enthält und geben in diesem Fall als Ergebnis die Menge 
      $\bigl\{\{\}\bigr\}$ zurück.
\item Dann testen wir in Zeile 7, ob bereits alle Klauseln $C$ aus der Menge
      \texttt{S} Unit-Klauseln sind.  Wenn dies so ist,
      dann ist die Menge \texttt{S} trivial und wir geben diese Menge als Ergebnis zurück.
\item Andernfalls wählen wir in Zeile 9 ein Literal $l$, das in einer Klausel aus der Menge \texttt{S} vorkommt, 
      das wir aber noch nicht benutzt haben.
      Wir untersuchen dann in Zeile 13 rekursiv, ob die Menge \\[0.2cm]
      \hspace*{1.3cm} 
      $\texttt{S} \cup \bigl\{\{\texttt{l}\}\bigr\}$ 
      \\[0.2cm]
      lösbar ist.  Dabei gibt es zwei Fälle:
      \begin{enumerate}
      \item Falls diese Menge lösbar ist, geben wir die Lösung dieser Menge als Ergebnis zurück.

      \item Sonst prüfen wir rekursiv, ob die Menge 
            \\[0.2cm]
            \hspace*{1.3cm}
            $\texttt{S} \cup \Bigl\{ \bigl\{ \komplement{\texttt{l}} \bigr\} \Bigr\}$ 
            \\[0.2cm]
            lösbar ist.  Ist diese Menge lösbar, so ist diese Lösung auch eine
            Lösung der Menge $\texttt{Clauses}$ und wir geben diese Lösung zurück.  Ist die
            Menge unlösbar, dann muss auch die Menge $\texttt{Clauses}$ unlösbar sein.
      \end{enumerate}
\end{enumerate}

Wir diskutieren nun die Hilfsprozeduren, die bei der Implementierung der Funktion
\texttt{solve} verwendet wurden.
Als erstes besprechen wir die Funktion \texttt{saturate}.  Diese Funktion erhält eine
Menge $S$ von Klauseln als Eingabe und führt alle möglichen Unit-Schnitte und
Unit-Subsumptionen durch.  
Die Funktion \texttt{saturate} ist in Abbildung \ref{fig:saturate} auf Seite \pageref{fig:saturate}
gezeigt.

\begin{figure}[!ht]
  \centering
\begin{minted}[ frame         = lines, 
                framesep      = 0.3cm, 
                numbers       = left,
                numbersep     = -0.2cm,
                bgcolor       = sepia,
                xleftmargin   = 0.8cm,
                xrightmargin  = 0.8cm
              ]{python3}
    def saturate(Clauses):
        S     = Clauses.copy()
        Units = { C for C in S if len(C) == 1 }
        Used  = set()
        while len(Units) > 0:
            unit  = Units.pop()
            Used |= { unit }
            l     = arb(unit)
            S     = reduce(S, l)
            Units = { C for C in S if len(C) == 1 } - Used        
        return S
\end{minted}
\vspace*{-0.3cm}
  \caption{Die Funktion \texttt{saturate}. \index{\texttt{saturate}}}
  \label{fig:saturate}
\end{figure} 
Die Implementierung von \texttt{saturate} funktioniert wie folgt: 
\begin{enumerate}
\item Zunächst kopieren wir die Menge \texttt{Clauses} in die Variable \texttt{S}.
      Dies ist notwendig, da wir die Menge \texttt{S} später verändern werden.  Die Funktion
      \texttt{saturate} soll das Argument \texttt{Clauses} aber nicht verändern und muss daher
      eine Kopie der Menge \texttt{S} anlegen.
\item Dann berechnen wir in Zeile 3 die Menge \texttt{Units} aller Unit-Klauseln.  
\item Anschließend initialisieren wir in Zeile 4 die Menge \texttt{Used} als die leere Menge.
      In dieser Menge merken wir uns, welche Unit-Klauseln wir schon für Unit-Schnitte und
      Subsumptionen benutzt haben.
\item Solange die Menge \texttt{Units} der Unit-Klauseln nicht leer ist, wählen wir in Zeile 6
      mit Hilfe der Funktion $\texttt{pop}$ eine beliebige Unit-Klausel \texttt{unit} aus der Menge
      \texttt{Units} aus und entfernen diese Unit-Klausel aus der Menge \texttt{Units}. 
\item In Zeile 7 fügen wir die Klausel \texttt{unit} zu der Menge
      \texttt{Used} der benutzten Klausel hinzu.  
\item In Zeile 8 extrahieren mit der Funktion \texttt{arb} das Literal \texttt{l} der Klausel
      \texttt{Unit}.  Die Funktion $\texttt{arb}$ liefert ein beliebiges Element der Menge zurück,
      das dieser Funktion als Argument übergeben wird.  Enthält diese Menge nur ein Element, so
      wird also dieses Element zurück gegeben.
\item In Zeile 9 wird  die eigentliche Arbeit durch einen Aufruf der Funktion
      \texttt{reduce} geleistet.  Diese Funktion berechnet alle Unit-Schnitte, die mit der
      Unit-Klausel $\{\texttt{l}\}$ möglich sind und entfernt darüber hinaus alle Klauseln, die
      durch die Unit-Klausel $\{\texttt{l}\}$ subsumiert werden.
\item Wenn die Unit-Schnitte mit der Unit-Klausel $\{\texttt{l}\}$ berechnet werden, können neue
      Unit-Klauseln entstehen, die wir in Zeile 10 aufsammeln.  Wir sammeln dort aber nur die Unit-Klauseln auf,
       die wir noch nicht benutzt haben. 
\item Die Schleife in den Zeilen 5 -- 10 wird nun solange durchlaufen, wie wir 
      Unit-Klauseln finden, die wir noch nicht benutzt haben.
\item Am Ende geben wir die verbliebene Klauselmenge als Ergebnis zurück.
\end{enumerate}
Die dabei verwendete Funktion $\texttt{reduce}$ ist in Abbildung \ref{fig:reduce} gezeigt.
Im vorigen Abschnitt hatten wir die Funktion $\textsl{reduce}(S, l)$, die eine
Klausel-Menge $\texttt{Cs}$ mit Hilfe des Literals $l$ reduziert, als
\\[0.2cm]
\hspace*{1.3cm}
$\textsl{reduce}(\texttt{Cs},l)  = 
 \Bigl\{\, C \backslash \bigl\{\komplement{l}\bigr\} \;|\; C \in \texttt{Cs} \wedge \komplement{l} \in C \,\Bigr\} 
       \,\cup\, \Bigl\{\, C \in \texttt{Cs} \mid \komplement{l} \not\in C \wedge l \not\in C \Bigr\} \cup \Bigl\{\{l\}\Bigr\}
$
\\[0.2cm]
definiert.
Die Implementierung setzt diese Definition unmittelbar um.  


\begin{figure}[!ht]
  \centering
\begin{minted}[ frame         = lines, 
                  framesep      = 0.3cm, 
                  numbers       = left,
                  numbersep     = -0.2cm,
                  bgcolor       = sepia,
                  xleftmargin   = 0.8cm,
                  xrightmargin  = 0.8cm
                ]{python3}
    def reduce(Clauses, l):
        lBar = complement(l)
        return   { C - { lBar } for C in Clauses if lBar in C }          \
               | { C for C in Clauses if lBar not in C and l not in C }  \
               | { frozenset({l}) }
\end{minted}
\vspace*{-0.3cm}
  \caption{Die Funktion \texttt{reduce}. \index{\texttt{reduce}}}
  \label{fig:reduce}
\end{figure} 

Die Implementierung des Algorithmus von Davis und Putnam benutzt außer den bisher diskutierten Funktionen
noch zwei weitere Hilfsprozeduren, deren Implementierung in 
Abbildung \ref{fig:solve-aux} auf Seite \pageref{fig:solve-aux} gezeigt wird.
\begin{enumerate}
\item Die Funktion \texttt{selectLiteral} wählt ein Literal aus 
      einer gegeben Menge $\texttt{Clauses}$ von Klauseln aus, dessen Variable außerdem nicht in der Menge
      \texttt{Forbidden} von den Variablen vorkommen darf, die bereits benutzt worden sind.
      Dazu iterieren wir zunächst über alle Klauseln $C$ aus der Menge $\texttt{Clauses}$ und dann über alle Literale $l$
      der Klausel $C$.  Aus diesen Literalen extrahieren wir die darin enthaltene Variable mit Hilfe der Funktion
      \texttt{extractVariable}.  Anschließend wird das Literal zurück gegeben, für welches der Wert der
      \blue{Jereslow-Wang-Heuristik} \cite{jereslow:1990} maximal ist.
\item Die Funktion \texttt{arb} gibt ein nicht näher spezifiziertes Element einer Menge zurück.
\end{enumerate}
\begin{figure}[!ht]
  \centering
\begin{minted}[ frame         = lines, 
                framesep      = 0.3cm, 
                numbers       = left,
                numbersep     = -0.2cm,
                bgcolor       = sepia,
                xleftmargin   = 0.3cm,
                xrightmargin  = 0.3cm
              ]{python3}
    def selectLiteral(Clauses, Forbidden):
        Variables = { extractVariable(l) for C in Clauses for l in C } - Forbidden
        Scores    = {} 
        for var in Variables:
            cmp = ('¬', var)
            Scores[var] = 0.0
            Scores[cmp] = 0.0
            for C in Clauses:
                if var in C:
                    Scores[var] += 2 ** -len(C)
                if cmp in C: 
                    Scores[cmp] += 2 ** -len(C)
        return max(Scores, key=Scores.get)
\end{minted}
\vspace*{-0.3cm}
  \caption{Die Funktionen \texttt{select} und \texttt{negateLiteral}}
  \label{fig:solve-aux}
\end{figure}

Die oben dargestellte Version des Verfahrens von Davis und Putnam lässt sich in vielerlei
Hinsicht verbessern.  Aus Zeitgründen können wir auf solche Verbesserungen nicht
weiter eingehen. Der interessierte Leser sei hier auf die folgende Arbeit von Moskewicz
et.al.~\cite{moskewicz:2001}  verwiesen: 
\\[0.2cm]
\hspace*{1.3cm} \textsl{Chaff: Engineering an Efficient SAT Solver} \\
\hspace*{1.3cm} von \blue{M. Moskewicz, C. Madigan, Y. Zhao, L. Zhang, S. Malik} 

\exercise
Die Klausel-Menge $M$ sei wie folgt gegeben: \\[0.2cm]
\hspace*{1.3cm} $M := \bigl\{ \; \{ r, p, s \},
                         \{r, s \}, \{ q, p, s \},
                         \{ \neg p,  \neg q\},
                         \{ \neg p, s,  \neg r\},
                         \{p,  \neg q, r \},$ \\[0.2cm]
\hspace*{2.6cm} $\{ \neg r,  \neg s, q\},
                         \{p, q, r, s \},
                         \{r,  \neg s, q\},
                         \{ \neg r,  s, \neg q\},
                         \{s, \neg r\} \bigr\}$ \\[0.2cm]
Überprüfen Sie mit dem Verfahren von Davis und Putnam, ob die Menge $M$ widersprüchlich ist.  
\eox


\section{Das 8-Damen-Problem}
In diesem Abschnitt zeigen wir, wie bestimmte kombinatorische Probleme als aussagenlogische Fragestellungen
formuliert werden können.  Diese können dann anschließend mit dem Algorithmus von Davis und Putnam gelöst werden.  Als
konkretes Beispiel betrachten wir das \href{https://en.wikipedia.org/wiki/Eight_queens_puzzle}{8-Damen-Problem}.  
Dabei geht es darum, 8 Damen so auf einem Schach-Brett aufzustellen, dass keine Dame eine andere Dame schlagen kann.
Beim \href{https://en.wikipedia.org/wiki/Chess}{Schach-Spiel} kann eine Dame dann eine andere Figur schlagen,
wenn diese Figur entweder 
\begin{itemize}
\item in derselben Reihe,
\item in derselben Spalte  oder
\item in derselben Diagonale
\end{itemize}
wie die Dame steht.  Abbildung \ref{fig:queens-problem} auf Seite \pageref{fig:queens-problem}
zeigt ein Schachbrett, in dem sich in der dritten Reihe in der vierten Spalte
eine Dame befindet.  Diese Dame kann auf alle die Felder ziehen, die mit Pfeilen markierte
sind, und kann damit Figuren, die sich auf diesen Feldern befinden, schlagen.

\begin{figure}[!ht]
  \centering
\setlength{\unitlength}{1.0cm}
\begin{picture}(10,9)
\thicklines
\put(1,1){\line(1,0){8}}
\put(1,1){\line(0,1){8}}
\put(1,9){\line(1,0){8}}
\put(9,1){\line(0,1){8}}
\put(0.9,0.9){\line(1,0){8.2}}
\put(0.9,9.1){\line(1,0){8.2}}
\put(0.9,0.9){\line(0,1){8.2}}
\put(9.1,0.9){\line(0,1){8.2}}
\thinlines
\multiput(1,2)(0,1){7}{\line(1,0){8}}
\multiput(2,1)(1,0){7}{\line(0,1){8}}
\put(4.15,6.15){{\chess Q}}
\multiput(5.25,6.5)(1,0){4}{\vector(1,0){0.5}}
\multiput(3.75,6.5)(-1,0){3}{\vector(-1,0){0.5}}
\multiput(5.25,7.25)(1,1){2}{\vector(1,1){0.5}}
\multiput(5.25,5.75)(1,-1){4}{\vector(1,-1){0.5}}
\multiput(3.75,5.75)(-1,-1){3}{\vector(-1,-1){0.5}}
\multiput(3.75,7.25)(-1,1){2}{\vector(-1,1){0.5}}
\multiput(4.5,7.25)(0,1){2}{\vector(0,1){0.5}}
\multiput(4.5,5.75)(0,-1){5}{\vector(0,-1){0.5}}
\end{picture}
\vspace*{-1.0cm}
  \caption{Das 8-Damen-Problem}
  \label{fig:queens-problem}
\end{figure}

Als erstes überlegen wir uns, wie wir ein Schach-Brett mit den darauf
positionierten Damen aussagenlogisch repräsentieren können.  Eine Möglichkeit besteht darin, 
für jedes Feld eine aussagenlogische Variable einzuführen.  Diese Variable drückt
aus, dass auf dem entsprechenden Feld eine Dame steht.  Wir ordnen diesen Variablen wie
folgt Namen zu:  Die Variable, die das $j$-te Feld in der $i$-ten
Reihe bezeichnet, stellen wir durch den String
\\[0.2cm]
\hspace*{1.3cm}
 $\texttt{'Q<}i\texttt{,}j\texttt{>'}$ \quad mit $i,j \in \{1, \cdots, 8\}$ 
\\[0.2cm]
dar.   Wir nummerieren die Reihen dabei von oben beginnend von 1 bis 8 durch, während die
Spalten von links nach rechts numeriert werden.  Abbildung \ref{fig:queens-assign} auf
Seite \pageref{fig:queens-assign} zeigt die Zuordnung der Variablen zu den Feldern.  Die in Abbildung
\ref{fig:var} gezeigte Funktion $\texttt{var}(r,c)$ berechnet die Variable, die ausdrückt, dass sich in Reihe
$r$ und Spalte $c$ eine Dame befindet.


\begin{figure}[!ht]
\centering
\begin{minted}[ frame         = lines, 
                framesep      = 0.3cm, 
                firstnumber   = 1,
                numbers       = left,
                numbersep     = -0.2cm,
                bgcolor       = sepia,
                xleftmargin   = 0.8cm,
                xrightmargin  = 0.8cm,
              ]{python3}
    def var(row, col):
        return 'Q<' + str(row) + ',' + str(col) + '>'
\end{minted}
\vspace*{-0.3cm}
\caption{Die Funktion \texttt{var} zur Berechnung der aussagenlogischen Variablen}
\label{fig:var}
\end{figure}



\begin{figure}[!ht]
  \centering
\setlength{\unitlength}{1.8cm}
\begin{picture}(10,9)
\thicklines
\put(0.9,0.9){\line(1,0){8.2}}
\put(0.9,9.1){\line(1,0){8.2}}
\put(0.9,0.9){\line(0,1){8.2}}
\put(9.1,0.9){\line(0,1){8.2}}
\put(1,1){\line(1,0){8}}
\put(1,1){\line(0,1){8}}
\put(1,9){\line(1,0){8}}
\put(9,1){\line(0,1){8}}
\thinlines
\multiput(1,2)(0,1){7}{\line(1,0){8}}
\multiput(2,1)(1,0){7}{\line(0,1){8}}

%%  for (i = 1; i <= 8; i = i + 1) {
%%for (j = 1; j <= 8; j = j + 1) \{
%%   \put(\$j.15,<9-$i>.35){{\Large p<$i>\$j}}
%%\}
%%  }

\put(1.20,8.45){{ Q<1,1> }}
\put(2.20,8.45){{ Q<1,2> }}
\put(3.20,8.45){{ Q<1,3> }}
\put(4.20,8.45){{ Q<1,4> }}
\put(5.20,8.45){{ Q<1,5> }}
\put(6.20,8.45){{ Q<1,6> }}
\put(7.20,8.45){{ Q<1,7> }}
\put(8.20,8.45){{ Q<1,8> }}
\put(1.20,7.45){{ Q<2,1> }}
\put(2.20,7.45){{ Q<2,2> }}
\put(3.20,7.45){{ Q<2,3> }}
\put(4.20,7.45){{ Q<2,4> }}
\put(5.20,7.45){{ Q<2,5> }}
\put(6.20,7.45){{ Q<2,6> }}
\put(7.20,7.45){{ Q<2,7> }}
\put(8.20,7.45){{ Q<2,8> }}
\put(1.20,6.45){{ Q<3,1> }}
\put(2.20,6.45){{ Q<3,2> }}
\put(3.20,6.45){{ Q<3,3> }}
\put(4.20,6.45){{ Q<3,4> }}
\put(5.20,6.45){{ Q<3,5> }}
\put(6.20,6.45){{ Q<3,6> }}
\put(7.20,6.45){{ Q<3,7> }}
\put(8.20,6.45){{ Q<3,8> }}
\put(1.20,5.45){{ Q<4,1> }}
\put(2.20,5.45){{ Q<4,2> }}
\put(3.20,5.45){{ Q<4,3> }}
\put(4.20,5.45){{ Q<4,4> }}
\put(5.20,5.45){{ Q<4,5> }}
\put(6.20,5.45){{ Q<4,6> }}
\put(7.20,5.45){{ Q<4,7> }}
\put(8.20,5.45){{ Q<4,8> }}
\put(1.20,4.45){{ Q<5,1> }}
\put(2.20,4.45){{ Q<5,2> }}
\put(3.20,4.45){{ Q<5,3> }}
\put(4.20,4.45){{ Q<5,4> }}
\put(5.20,4.45){{ Q<5,5> }}
\put(6.20,4.45){{ Q<5,6> }}
\put(7.20,4.45){{ Q<5,7> }}
\put(8.20,4.45){{ Q<5,8> }}
\put(1.20,3.45){{ Q<6,1> }}
\put(2.20,3.45){{ Q<6,2> }}
\put(3.20,3.45){{ Q<6,3> }}
\put(4.20,3.45){{ Q<6,4> }}
\put(5.20,3.45){{ Q<6,5> }}
\put(6.20,3.45){{ Q<6,6> }}
\put(7.20,3.45){{ Q<6,7> }}
\put(8.20,3.45){{ Q<6,8> }}
\put(1.20,2.45){{ Q<7,1> }}
\put(2.20,2.45){{ Q<7,2> }}
\put(3.20,2.45){{ Q<7,3> }}
\put(4.20,2.45){{ Q<7,4> }}
\put(5.20,2.45){{ Q<7,5> }}
\put(6.20,2.45){{ Q<7,6> }}
\put(7.20,2.45){{ Q<7,7> }}
\put(8.20,2.45){{ Q<7,8> }}
\put(1.20,1.45){{ Q<8,1> }}
\put(2.20,1.45){{ Q<8,2> }}
\put(3.20,1.45){{ Q<8,3> }}
\put(4.20,1.45){{ Q<8,4> }}
\put(5.20,1.45){{ Q<8,5> }}
\put(6.20,1.45){{ Q<8,6> }}
\put(7.20,1.45){{ Q<8,7> }}
\put(8.20,1.45){{ Q<8,8> }}

\end{picture}
\vspace*{-1.0cm}
  \caption{Zuordnung der Variablen}
  \label{fig:queens-assign}
\end{figure}

Als nächstes überlegen wir uns, wie wir die einzelnen Bedingungen des 8-Damen-Problems 
als aussagenlogische
Formeln kodieren können.  Letztlich lassen sich alle Aussagen der Form
\begin{itemize}
\item ``in einer Reihe steht höchstens eine Dame'', 
\item ``in einer Spalte steht höchstens eine Dame'', oder 
\item ``in einer Diagonale steht höchstens eine Dame'' 
\end{itemize}
auf dasselbe Grundmuster zurückführen:
Ist eine Menge von aussagenlogischen Variablen \\[0.2cm]
\hspace*{1.3cm} $V = \{ x_1, \cdots, x_n \}$ \\[0.2cm]
gegeben, so brauchen wir eine Formel die aussagt, dass \blue{höchstens} eine der Variablen aus
$V$ den Wert \texttt{True} hat.  Das ist aber gleichbedeutend damit, dass für jedes Paar
$x_i, x_j \in V$ mit $x_i \not= x_j$ die folgende Formel gilt: \\[0.2cm]
\hspace*{1.3cm} $\neg (x_i \wedge x_j)$. \\[0.2cm]
Diese Formel drückt aus, dass die Variablen $x_i$ und $x_j$ nicht gleichzeitig den Wert
\texttt{True} annehmen.  Nach den De\-Morgan'schen Gesetzen gilt
\\[0.2cm]
\hspace*{1.3cm}
$\neg (x_i \wedge x_j) \leftrightarrow \neg x_i \vee \neg x_j$
\\[0.2cm]
und die Klausel auf der rechten Seite dieser Äquivalenz schreibt sich in Mengen-Schreibweise als
\\[0.2cm]
\hspace*{1.3cm}  $\{\neg x_i, \neg x_j \}$. \\[0.2cm]
Die Formel, die für eine Variablen-Menge $V$ ausdrückt, dass keine zwei verschiedenen
Variablen gleichzeitig wahr sind, kann daher als Klausel-Menge in der Form
\\[0.2cm]
\hspace*{1.3cm}
$\bigl\{\, \{ \neg p, \neg q \} \;|\; p \in V \wedge\ q \in V \wedge p \not= q \bigr\}$
\\[0.2cm]
geschrieben werden.
Wir setzen diese Überlegungen in eine \textsl{Python}-Funktion um.  Die in Abbildung \ref{fig:atMostOne}
gezeigte Funktion \texttt{atMostOne}() bekommt als Eingabe eine Menge $S$ von
aussagenlogischen Variablen.  Der Aufruf $\texttt{atMostOne}(S)$ berechnet eine Menge von
Klauseln.  Diese Klauseln sind genau dann wahr, wenn höchstens eine der Variablen aus $S$
den Wert \texttt{True} hat.

\begin{figure}[!ht]
  \centering
\begin{minted}[ frame         = lines, 
                framesep      = 0.3cm, 
                numbers       = left,
                numbersep     = -0.2cm,
                bgcolor       = sepia,
                xleftmargin   = 0.8cm,
                xrightmargin  = 0.8cm
              ]{python3}
    def atMostOne(S): 
        return { frozenset({('¬',p), ('¬', q)}) for p in S
                                                 for q in S 
                                                 if  p != q 
               }
\end{minted}
\vspace*{-0.3cm}
  \caption{Die Funktion \texttt{atMostOne}}
  \label{fig:atMostOne}
\end{figure}

Mit Hilfe der Funktion \texttt{atMostOne} können wir nun die Funktion
\texttt{atMostOneInRow} implementieren.  Der Aufruf \\[0.2cm]
\hspace*{1.3cm} 
\texttt{atMostOneInRow(row, n)} \\[0.2cm]
berechnet für eine gegebene Reihe \texttt{row} bei einer Brettgröße von \texttt{n} eine Formel,
die ausdrückt, dass in der Reihe \texttt{row} höchstens eine Dame steht.
Abbildung \ref{fig:atMostOneInRow} zeigt die
Funktion \texttt{atMostOneInRow}: Wir sammeln alle Variablen der durch \texttt{row}
spezifizierten Reihe
in der Menge 
\\[0.2cm]
\hspace*{1.3cm}
$\bigl\{ \texttt{var}(\texttt{row},j) \mid j \in \{1, \cdots, n \} \bigr\}$
\\[0.2cm]
 auf und rufen mit dieser Menge die Funktion $\texttt{atMostOne}()$ auf, die das Ergebnis
als Menge von Klauseln liefert.

\begin{figure}[!ht]
  \centering
\begin{minted}[ frame         = lines, 
                framesep      = 0.3cm, 
                numbers       = left,
                numbersep     = -0.2cm,
                bgcolor       = sepia,
                xleftmargin   = 0.8cm,
                xrightmargin  = 0.8cm
              ]{python3}
    def atMostOneInRow(row, n):
        return atMostOne({ var(row, col) for col in range(1,n+1) })
\end{minted}
\vspace*{-0.3cm}
  \caption{Die Funktion \texttt{atMostOneInRow}}
  \label{fig:atMostOneInRow}
\end{figure}

Als nächstes berechnen wir eine Formel die aussagt, dass \blue{mindestens} eine Dame in einer gegebenen
Spalte steht.  Für die erste Spalte hätte diese Formel die Form 
\\[0.2cm]
\hspace*{1.3cm}
$\texttt{Q<1,1>} \vee \texttt{Q<2,1>} \vee \texttt{Q<3,1>} \vee \texttt{Q<4,1>} \vee \texttt{Q<5,1>} \vee
\texttt{Q<6,1>} \vee \texttt{Q<7,1>} \vee \texttt{Q<8,1>}$
\\[0.2cm]
und wenn allgemein eine Spalte $c$ mit $c \in \{1,\cdots,8\}$ gegeben ist, lautet die Formel
\\[0.2cm]
\hspace*{1.3cm}
$\texttt{Q<1,c>} \vee \texttt{Q<2,c>} \vee \texttt{Q<3,c>} \vee \texttt{Q<4,c>} \vee \texttt{Q<5,c>} \vee
\texttt{Q<6,c>} \vee \texttt{Q<7,c>} \vee \texttt{Q<8,c>}$.
\\[0.2cm]
Schreiben wir diese Formel in der Mengenschreibweise als Menge von Klauseln, so erhalten wir
\\[0.2cm]
\hspace*{1.3cm}
$\bigl\{ \{\texttt{Q<1,c>} , \texttt{Q<2,c>} , \texttt{Q<3,c>} , \texttt{Q<4,c>} , \texttt{Q<5,c>} ,
\texttt{Q<6,c>} , \texttt{Q<7,c>} , \texttt{Q<8,c>} \}\bigr\}$.
\\[0.2cm]
Abbildung \ref{fig:oneInColumn} zeigt eine \textsl{Python}-Funktion, die für eine gegebene Spalte
\texttt{col} und eine gegebene Brettgröße \texttt{n} die entsprechende Klausel-Menge berechnet.
Der Schritt, von einer einzelnen Klausel 
zu einer Menge von Klauseln überzugehen ist notwendig, denn unsere Implementierung des Algorithmus von
Davis und Putnam arbeitet mit einer Menge von Klauseln.

\begin{figure}[!ht]
  \centering
\begin{minted}[ frame         = lines, 
                framesep      = 0.3cm, 
                numbers       = left,
                numbersep     = -0.2cm,
                bgcolor       = sepia,
                xleftmargin   = 0.8cm,
                xrightmargin  = 0.8cm
              ]{python3}
    def oneInColumn(col, n):
        return { frozenset({ var(row, col) for row in range(1,n+1) }) }
\end{minted}
\vspace*{-0.3cm}
  \caption{Die Funktion \texttt{oneInColumn}}
  \label{fig:oneInColumn}
\end{figure}

An dieser Stelle erwarten Sie vielleicht, dass wir noch Formeln angeben die
ausdrücken, dass in einer gegebenen Spalte höchstens eine Dame steht und dass in jeder
Reihe mindestens eine Dame steht.
Solche Formeln sind aber unnötig, denn wenn wir wissen, dass in jeder Spalte mindestens
eine Dame steht, so wissen wir bereits, dass auf dem Brett mindestens 8 Damen stehen.
Wenn wir nun zusätzlich wissen, dass in jeder Reihe höchstens eine Dame steht, so ist
automatisch klar, dass höchstens 8 Damen auf dem Brett stehen.  Damit stehen also insgesamt genau
8 Damen auf dem Brett.  Dann kann aber in jeder Spalte nur höchstens eine Dame stehen, denn sonst hätten wir
mehr als 8 Damen auf dem Brett und genauso muss in jeder Reihe mindestens eine Dame stehen, denn sonst würden
wir in der Summe nicht auf 8 Damen kommen. 

Als nächstes überlegen wir uns, wie wir die Variablen, die auf derselben \blue{Diagonale}
stehen, charakterisieren können.  Es gibt grundsätzlich zwei verschiedene Arten von
Diagonalen: \blue{Absteigende} Diagonalen und \blue{aufsteigende} Diagonalen.  Wir be\-trach\-ten zunächst
die aufsteigenden Diagonalen.  Die längste aufsteigende Diagonale, wir sagen dazu auch
\blue{Hauptdiagonale}, besteht im Fall eines $8 \times 8$-Bretts aus den
Variablen \\[0.2cm]
\hspace*{1.3cm} 
$\texttt{Q<8,1>},\; \texttt{Q<7,2>},\; \texttt{Q<6,3>},\; \texttt{Q<5,4>},\; \texttt{Q<4,5>},\; \texttt{Q<3,6>},\; 
 \texttt{Q<2,7>},\; \texttt{Q<1,8>}$. 
\\[0.2cm]
Die Indizes $r$ und $c$ der Variablen $\texttt{Q}(r,c)$ erfüllen offenbar
die Gleichung \\[0.2cm]
\hspace*{1.3cm} $r + c = 9$. \\[0.2cm]
Allgemein erfüllen die Indizes der Variablen einer aufsteigenden Diagonale, die mehr als ein Feld enthält, die
Gleichung \\[0.2cm] 
\hspace*{1.3cm} $r + c = k$, \\[0.2cm]
wobei $k$ im Falle eines $8 \times 8$ Schach-Bretts einen Wert aus der Menge $\{3, \cdots, 15 \}$ annimmt.  Den Wert $k$ geben wir als Argument bei der
Funktion \texttt{atMostOneInRisingDiagonal} mit.  Diese Funktion ist in Abbildung
\ref{fig:atMostOneInUpperDiagonal} gezeigt. 

\begin{figure}[!ht]
  \centering
\begin{minted}[ frame         = lines, 
                framesep      = 0.3cm, 
                numbers       = left,
                numbersep     = -0.2cm,
                bgcolor       = sepia,
                xleftmargin   = 0.0cm,
                xrightmargin  = 0.0cm
              ]{python3}
    def atMostOneInRisingDiagonal(k, n):
        S = { var(row, col) for row in range(1, n+1)
                            for col in range(1, n+1) 
                            if  row + col == k 
            }
        return atMostOne(S)
\end{minted}
\vspace*{-0.3cm}
  \caption{Die Funktion \texttt{atMostOneInUpperDiagonal}}
  \label{fig:atMostOneInUpperDiagonal}
\end{figure}

Um zu sehen, wie die Variablen einer fallenden Diagonale
charakterisiert werden können, betrachten wir die fallende Hauptdiagonale, die aus den
Variablen \\[0.2cm]
\hspace*{1.3cm} 
$\texttt{Q<1,1>},\; \texttt{Q<2,2>},\; \texttt{Q<3,3>},\; \texttt{Q<4,4>},\; \texttt{Q<5,5>},\; 
 \texttt{Q<6,6>},\; \texttt{Q<7,7>},\; \texttt{Q<8,8>}$ 
\\[0.2cm]
besteht. Die Indizes  $r$ und $c$ dieser Variablen erfüllen offenbar
die Gleichung \\[0.2cm]
\hspace*{1.3cm} $r - c = 0$. \\[0.2cm]
Allgemein erfüllen die Indizes der Variablen einer absteigenden Diagonale die Gleichung \\[0.2cm]
\hspace*{1.3cm} $r - c = k$, \\[0.2cm]
wobei $k$ einen Wert aus der Menge $\{-6, \cdots, 6 \}$ annimmt.  Den Wert $k$ geben wir als Argument bei der
Funktion \texttt{atMostOneInLowerDiagonal} mit. Diese Funktion ist in Abbildung
\ref{fig:atMostOneInLowerDiagonal} gezeigt. 

\begin{figure}[!ht]
  \centering
\begin{minted}[ frame         = lines, 
                framesep      = 0.3cm, 
                numbers       = left,
                numbersep     = -0.2cm,
                bgcolor       = sepia,
                xleftmargin   = 0.0cm,
                xrightmargin  = 0.0cm
            ]{python3}
    def atMostOneInFallingDiagonal(k, n):
        S = { var(row, col) for row in range(1, n+1)
                            for col in range(1, n+1) 
                            if  row - col == k 
            }
        return atMostOne(S)
\end{minted}
\vspace*{-0.3cm}
  \caption{Die Funktion \texttt{atMostOneInLowerDiagonal}}
  \label{fig:atMostOneInLowerDiagonal}
\end{figure}

Jetzt sind wir in der Lage, unsere Ergebnisse zusammen zu fassen:  Wir können eine
Menge von Klauseln konstruieren, die das 8-Damen-Problem vollständig beschreiben.
Abbildung \ref{fig:allClauses} zeigt die Implementierung der Funktion \texttt{allClauses}.
Der Aufruf \\[0.2cm]
\hspace*{1.3cm} $\texttt{allClauses}(n)$ \\[0.2cm]
rechnet für ein Schach-Brett der Größe $n$ eine Menge von Klauseln aus, die
genau dann erfüllt sind, wenn auf dem Schach-Brett
\begin{enumerate}
\item in jeder Reihe höchstens eine Dame steht (Zeile 2),
\item in jeder absteigenden Diagonale höchstens eine Dame steht (Zeile 3),
\item in jeder aufsteigenden Diagonale höchstens eine Dame steht (Zeile 4) und
\item in jeder Spalte mindestens eine Dame steht (Zeile 5).
\end{enumerate}
Die Ausdrücke in den einzelnen Zeilen liefern Listen, deren Elemente
Klausel-Mengen sind.  Was wir als Ergebnis brauchen, ist aber eine Klausel-Menge
und keine Liste von Klausel-Mengen.  Daher wandeln wir in Zeile 6 die Liste \texttt{All} in eine Menge von
Klauseln um.


\begin{figure}[!ht]
  \centering
\begin{minted}[ frame         = lines, 
                framesep      = 0.3cm, 
                numbers       = left,
                numbersep     = -0.2cm,
                bgcolor       = sepia,
                xleftmargin   = 0.3cm,
                xrightmargin  = 0.3cm
              ]{python3}
    def allClauses(n):
        All = [ atMostOneInRow(row, n)           for row in range(1, n+1)        ] \
            + [ atMostOneInFallingDiagonal(k, n) for k in range(-(n-2), (n-2)+1) ] \
            + [ atMostOneInRisingDiagonal(k, n)  for k in range(3, (2*n-1)+1)    ] \
            + [ oneInColumn(col, n)              for col in range(1, n+1)        ]
        return { clause for S in All for clause in S }
\end{minted}
\vspace*{-0.3cm}
  \caption{Die Funktion \texttt{allClauses}}
  \label{fig:allClauses}
\end{figure}

Als letztes zeigen wir in Abbildung \ref{fig:solve:queens} die Funktion
\texttt{queens}, mit der wir das 8-Damen-Problem lösen können.
\begin{enumerate}
\item Zunächst kodieren wir das Problem als eine Menge von Klauseln, die genau dann lösbar ist,
      wenn das Problem eine Lösung hat.
\item Anschließend berechnen wir die Lösung mit Hilfe der Funktion \texttt{sovle} aus dem Modul
      \texttt{davisPutnam}, das wir als \texttt{dp} importiert haben.
\item Zum Schluss wird die berechnete Lösung mit Hilfe der Funktion \texttt{printBoard} ausgedruckt.

      Hierbei ist $\texttt{printBoard}$ eine Funktion, welche die Lösung in lesbarere Form 
      ausdruckt.  Das funktioniert allerdings nur, wenn ein Font verwendet wird, bei dem alle Zeichen die
      selbe Breite haben.  Diese Funktion ist der Vollständigkeit halber in Abbildung \ref{fig:printBoard}
      gezeigt, wir wollen die Implementierung aber nicht weiter diskutieren.
\end{enumerate}
Das vollständige Programm finden Sie als Jupyter Notebook auf meiner Webseite unter dem Namen
\href{https://github.com/karlstroetmann/Logic/blob/master/Python/N-Queens.ipynb}{\texttt{N-Queens.ipynb}}.


\begin{figure}[!ht]
\centering
\begin{minted}[ frame         = lines, 
                framesep      = 0.3cm, 
                firstnumber   = 1,
                numbers       = left,
                numbersep     = -0.2cm,
                bgcolor       = sepia,
                xleftmargin   = 0.8cm,
                xrightmargin  = 0.8cm,
              ]{python3}
    def queens(n):
        "Solve the n queens problem."
        Clauses  = allClauses(n)
        Solution = dp.solve(Clauses, set())
        if Solution != { frozenset() }:
            return Solution
        else:
            print(f'The problem is not solvable for {n} queens!')
\end{minted}
\vspace*{-0.3cm}
\caption{Die Funktion \texttt{queens} zur Lösung des $n$-Damen-Problems.}
\label{fig:solve:queens}
\end{figure}



\begin{figure}[!ht]
\centering
\begin{minted}[ frame         = lines, 
                framesep      = 0.3cm, 
                numbers       = left,
                numbersep     = -0.2cm,
                bgcolor       = sepia,
                xleftmargin   = 0.8cm,
                xrightmargin  = 0.8cm,
              ]{python3}
    import chess
                
    def show_solution(Solution, n):
        board = chess.Board(None)  # create empty chess board
        queen = chess.Piece(chess.QUEEN, True)
        for row in range(1, n+1):
            for col in range(1, n+1):
                field_number = (row - 1) * 8 + col - 1
                if frozenset({ var(row, col) }) in Solution:
                    board.set_piece_at(field_number, queen)
        display(board)        
\end{minted}
\vspace*{-0.3cm}
\caption{Die Funktion $\texttt{show\_solution}()$.}
\label{fig:printBoard}
\end{figure}



Die durch den Aufruf $\texttt{solve}(\textsl{Clauses}, \{\})$ 
berechnete Menge \texttt{solution} enthält für jede der Variablen $\texttt{'Q<}r\texttt{,}c\texttt{>'}$
entweder die Unit-Klausel $\{\texttt{'Q<}r\texttt{,}c\texttt{>'}\}$  (falls auf diesem Feld eine Dame steht) oder
aber die Unit-Klausel  $\{ \texttt{('¬', 'Q<}r\texttt{,}c\texttt{>')}\}$ (falls das Feld leer bleibt).
Eine graphische Darstellung einer berechneten Lösungen sehen Sie in Abbildung \ref{fig:8-queens.pdf}.
Diese graphische Darstellung habe ich mit Hilfe der Bibliothek
\href{https://python-chess.readthedocs.io/en/latest/}{\texttt{python-chess}} und der Funktion
\texttt{show\_solution}, die in Abbildung \ref{fig:printBoard} gezeigt ist, erzeugt.

Jessica Roth und Koen Loogman (das sind zwei ehemalige DHBW-Studenten) haben eine Animation des Verfahren von
Davis und Putnam implementiert.  Sie können diese Animation unter der Adresse
\\[0.2cm]
\hspace*{1.3cm}
\href{https://koenloogman.github.io/Animation-Of-N-Queens-Problem-In-JavaScript/}{https://koenloogman.github.io/Animation-Of-N-Queens-Problem-In-JavaScript/}
\\[0.2cm]
im Netz finden und ausprobieren.

Das 8-Damen-Problem ist natürlich nur eine spielerische Anwendung der Aussagen-Logik.
Trotzdem zeigt es die Leistungsfähigkeit des Algorithmus von Davis
und Putnam sehr gut, denn die Menge der Klauseln, die von der Funktion \texttt{allClauses}
berechnet wird, besteht aus 512 verschiedenen Klauseln.  In dieser Klausel-Menge kommen 64 verschiedene
Variablen vor. 

In der Praxis gibt es viele Probleme, die sich in ganz ähnlicher Weise auf die Lösung einer
Menge von Klauseln zurückführen lassen.  Dazu gehört zum Beispiel das Problem, einen
Stundenplan zu erstellen, der gewissen Nebenbedingungen genügt.  Verallgemeinerungen des
Stundenplan-Problems werden in der Literatur als \blue{Scheduling-Probleme} bezeichnet.
Die effiziente Lösung solcher Probleme ist Gegenstand der aktuellen Forschung.




\begin{figure}[!ht]
  \centering
  \framebox{\epsfig{file=Figures/8-queens.pdf, scale=0.8}} 
  \caption{Eine Lösung des 8-Damen-Problems.}
  \label{fig:8-queens.pdf}
\end{figure}
\FloatBarrier
\vspace*{3cm}

\pagebreak


\section{Reflexion}
\begin{enumerate}[(a)]
\item Wie haben wir die Menge der aussagenlogischen Formeln definiert?
\item Wie ist die Semantik der aussagenlogischen Formeln festgelegt worden?
\item Wie können wir aussagenlogische Formeln in \textsl{Python} darstellen?
\item Was ist eine Tautologie?
\item Was ist eine konjunktive Normalform?
\item Wie können Sie die konjunktive Normalform einer gegebenen aussagenlogischen Formel berechnen und wie lässt
      sich diese Berechnung in \textsl{Python} implementieren?
\item Wie haben wir den Beweis-Begriff $M \vdash C$ definiert?
\item Welche Eigenschaften hat der Beweis-Begriff $\vdash$?
\item Wann ist eine Menge von Klauseln lösbar?
\item Wie funktioniert das Verfahren von Davis und Putnam?
\item Wie können Sie das 8-Damen-Problem als aussagenlogisches Problem formulieren?
\end{enumerate}

%\input{compact-barwise}

%%% Local Variables: 
%%% mode: latex
%%% TeX-master: "logic"
%%% End: 
