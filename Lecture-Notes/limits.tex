\chapter{Limits of Computability}
Every discipline of the sciences has its limits: Students of the medical sciences soon realize that
it is difficult to \href{http://www.wowhead.com/spell=61999}{raise the dead} and even religious zealots
have trouble \href{http://www.youtube.com/watch?v=RUMX_b_m3Js}{to walk on water}.  Similarly,
computer science has its limits.  We will discuss 
these limits next.  First, we show that we cannot decide whether a computer program will eventually
terminate or whether it will run forever.  
Second, we prove that it is impossible to automatically check whether two functions are equivalent.


\section{The Halting Problem}
In this subsection we prove that it is not possible for a computer program to decide whether 
another computer program does terminate.  This problem is known as the 
\href{http://en.wikipedia.org/wiki/Halting_problem}{\emph{halting problem}}.
\index{halting problem}
Before we give a formal proof that the halting problem is undecidable, let us
discuss one example that shows why it is indeed difficult to decide whether a program does always
terminate.  Consider the program shown in Figure \ref{fig:legendre.stlx} on page
\pageref{fig:legendre.stlx}.  This program contains a \texttt{while}-loop in line 18.  
If there is a natural number $n \geq m$ such that the expression,
\\[0.2cm]
\hspace*{1.3cm}
$\mathtt{legendre}(n)$
\\[0.2cm]
in line 19 evaluates to \texttt{false}, then the program prints a message and terminates.   However, if 
$\mathtt{legendre}(n)$ is true for all $n \geq m$, then the \texttt{while}-loop does not terminate.

Given a natural number \texttt{n}, the expression
$\texttt{legendre}(n)$ tests whether there is a prime number between $\texttt{n}^2$ and $(\texttt{n}+1)^2$.  
If, however, the set
\\[0.2cm]
\hspace*{1.3cm}
$\{ k \in \mathbb{N} \mid n^2 \leq k \wedge k \leq (n+1)^2 \}$
\\[0.2cm]
does not contain a prime number, then $\texttt{legendre}(n)$ evaluates
to \texttt{False} for this value of $n$.  The function \texttt{legendre} is defined in line 7.  
Given a natural number $n$, it returns \texttt{True} if and only if the formula
\\[0.2cm]
\hspace*{1.3cm}
$\exists k \in \mathbb{N}:\bigl( n^2 < k \wedge k < (n+1)^2 \wedge \textsl{isPrime}(k)\bigr)$
\\[0.2cm]
holds true.  The French mathematican 
\href{http://en.wikipedia.org/wiki/Adrien-Marie_Legendre}{Adrien-Marie Legendre} (1752 -- 1833) conjectured that
for any natural number $n \in \mathbb{N}$ there is prime number $p$ such that
\\[0.2cm]
\hspace*{1.3cm}
$n^2 < p \wedge  p < (n+1)^2$
\\[0.2cm]
holds.  Although there are a number of arguments in support of Legendre's conjecture,  to this day
nobody has been able to prove it.  The answer to the question, whether the invocation of the function $f$ will
terminate for every user input is, therefore, unknown as it depends on the truth of 
\href{http://en.wikipedia.org/wiki/Legendre's_conjecture}{Legendre's conjecture}:  If we
had some procedure that could check whether the function call $\texttt{find\_counter\_example}(1)$ does terminate,
then this procedure would be able to decide whether Legendre's theorem is true.  Therefore, it
should come as no surprise that such a procedure does not exist.


\begin{figure}[!ht]
\centering
\begin{minted}[ frame         = lines, 
                framesep      = 0.3cm, 
                firstnumber   = 1,
                numbers       = left,
                numbersep     = -0.2cm,
                bgcolor       = sepia,
                xleftmargin   = 0.0cm,
                xrightmargin  = 0.0cm,
              ]{python3}
    def divisors(k):
        return { t for t in range(1, k+1) if k % t == 0 }

    def is_prime(k):
        return divisors(k) == {1, k}    
    
    def legendre(n):
        k = n * n + 1;
        while k < (n + 1) ** 2:
            if is_prime(k):
                print(f'{n}**2 < {k} < {n+1}**2')
                return True
            k += 1
        return False

    def find_counter_example(m):
        n = m
        while True:
           if legendre(n):
               n = n + 1
           else:
               print(f'Counter example found: No prime between {n}**2 and {n+1}**2!')
               return
\end{minted}
\vspace*{-0.3cm}
\caption{A program checking Legendre's conjecture.}
\label{fig:legendre.stlx}
\end{figure}

Let us proceed to prove formally that the halting problem is not solvable.  To this end, we need the
following definition.

\begin{Definition}[Test Function] 
A string $t$ is a \blue{test function with name $f$} 
\index{test function}
iff $t$ has the form {\em\\[0.2cm]
\hspace*{1.3cm} \texttt{\symbol{34}\symbol{34}\symbol{34}}         \\
\hspace*{1.3cm} \texttt{def $f$(x):} \\
\hspace*{1.8cm} \textsl{body}        \\
\hspace*{1.3cm} \texttt{\symbol{34}\symbol{34}\symbol{34}}}         \\[0.2cm]
and, furthermore, the string $t$ can be parsed as a \textsl{Python} function, that is the evaluation of
the expression
\\[0.2cm]
\hspace*{1.3cm}
$\texttt{exec}(t)$
\\[0.2cm]
does not yield an error.  
The set of all test functions is denoted as $T\!F$.  If $t \in T\!F$ and $t$ has the name $f$, then
this is written as 
\\[0.2cm]
\hspace*{1.3cm}
$\mathtt{name}(t) = f$. \hspace*{\fill} $\Box$
\end{Definition}

\examplesEng
\begin{enumerate}
\item We define the string $s_1$ as follows:
      \begin{verbatim}
      """
      def simple(x): 
          return 0
      """
      \end{verbatim}
      \vspace*{-0.8cm}

      Then $s_1$ is a test function with the name \texttt{simple}.
\item We define the string $s_2$ as
      \begin{verbatim}
      """
      def loop(x): 
          while True: 
              x = x + 1
      """
      \end{verbatim}
      \vspace*{-0.8cm}

      Then $s_2$ is a test function with the name \texttt{loop}. 
\item We define the string $s_3$ as
      \begin{verbatim}
      """
      def hugo(x):
          return ++x
      """
      \end{verbatim}
      \vspace*{-0.8cm}

      Then $s_3$ is not a test function.  The reason is that \textsl{Python} does not support the operator
      ``\texttt{++}''.  Therefore, 
      \\[0.2cm]
      \hspace*{1.3cm}
      \texttt{exec(s3)}
      \\[0.2cm]
      yields an error message complaining about the two ``\texttt{+}'' characters.
\end{enumerate}
In order to be able to formalize the halting problem succinctly, we introduce three additional
notations.

\begin{Notation}[$\leadsto$, $\downarrow$, $\uparrow$]
If $n$ is the name of a \textsl{Python} function that takes $k$ arguments $a_1$, $\cdots$, $a_k$,
then we write 
\\[0.2cm]
\hspace*{1.3cm}
 $n(a_1, \cdots, a_k) \leadsto r$ 
\\[0.2cm]
iff the evaluation of the expression $n(a_1, \cdots, a_k)$ yields the result $r$.  If we are not
concerned with the result $r$ but only want to state that the evaluation \blue{terminates} eventually,
then we will write
\\[0.2cm]
\hspace*{1.3cm} $n(a_1, \cdots, a_k) \,\downarrow$ \\[0.3cm]
and read this notation as ``\emph{evaluation of $n(a_1, \cdots, a_k)$ terminates}''.
If the evaluation of the expression $n(a_1, \cdots, a_k)$ does \underline{not} \blue{terminate}, this is
written as \\[0.2cm]
\hspace*{1.3cm}
 $n(a_1, \cdots, a_k) \,\uparrow$. 
\\[0.2cm]
This notation is read as ``\emph{evaluation of $n(a_1, \cdots, a_k)$ \blue{diverges}}''.
\hspace*{\fill} $\Box$
\end{Notation}

\examplesEng  Using the test functions defined earlier, we have:
\begin{enumerate}
\item {\tt simple(\symbol{34}emil\symbol{34}) $\leadsto 0$},
\item {\tt simple(\symbol{34}emil\symbol{34}) $\downarrow$},
\item {\tt loop(2) $\uparrow$}.
\end{enumerate}

\noindent
The \blue{halting problem} \index{halting problem} for \textsl{Python} functions is the question whether there is a
\textsl{Python} function \\[0.2cm]
\hspace*{1.3cm} \texttt{def stops($t$,$\;a$): } \\
\hspace*{2.3cm} $\vdots$
\\[0.2cm] 
that takes as input a test function $t$ and a string $a$ and that satisfies the following specification:
\begin{enumerate}
\item $t \not\in T\!F \quad\Leftrightarrow\quad \mathtt{stops}(t, a) \leadsto 2$.

      If the first argument of \texttt{stops} is not a test function, then 
      \texttt{stops($t$, $a$)} returns the number $2$.

\item $t \in T\!F \,\wedge\, \mathtt{name}(t) = n \,\wedge\, n(a)\downarrow \quad\Leftrightarrow\quad
       \mathtt{stops}(t, a) \leadsto 1$.

      If the first argument of \texttt{stops} is a test function with name $n$ and, furthermore,
      the evaluation of $n(a)$ terminates, then \texttt{stops($t$, $a$)} returns the number $1$.

\item $t \in T\!F \,\wedge\, \mathtt{name}(t) = n \,\wedge\, n(a)\uparrow \quad\Leftrightarrow\quad
       \mathtt{stops}(t, a) \leadsto 0$.

      If the first argument of \texttt{stops} is a test function with name $n$ but the evaluation of $n(a)$ 
      diverges, then 
      \texttt{stops($t$, $a$)} returns the number $0$.
\end{enumerate}
If there was a \textsl{Python} function \texttt{stops} that did satisfy the specification given above,
then the halting problem for \textsl{Python} would be \blue{decidable}.
\index{decidable}

\begin{Theorem}[\href{http://en.wikipedia.org/wiki/Alan_Turing}{Alan Turing}, 1936]
  The halting problem is undecidable.
\end{Theorem}
\index{Turing, Alan}

\noindent
\textbf{Proof}:  In order to prove the undecidabilty of the halting problem we have to show that
there can be no function \texttt{stops} satisfying the specification given above.  This calls for an
indirect proof also known as a \href{http://en.wikipedia.org/wiki/Indirect_proof}{\emph{proof by contradiction}}.
We will therefore assume that a function \texttt{stops} solving the halting problem does
exist and we will then show that this assumption leads to a contradiction.  This contradiction will
leave us with the conclusion that there can be no function \texttt{stops} that satisfies
the specification given above and that, therefore, the halting problem is undecidable.

In order to proceed, let us assume that a \textsl{Python} function \texttt{stops}
satisfying the specification given above exists and let us define the string
\textsl{turing} as shown in Figure \ref{fig:turing-string} below.

\begin{figure}[!h]
  \centering
\begin{minted}[ frame         = lines, 
                framesep      = 0.3cm, 
                numbers       = left,
                numbersep     = -0.2cm,
                bgcolor       = sepia,
                xleftmargin   = 0.8cm,
                xrightmargin  = 0.8cm,
              ]{python3}  
    turing = """
             def alan(x):
                 result = stops(x, x)
                 if result == 1:
                     while True:
                         print("... looping ...")
                 return result
             """ 
\end{minted}
  \vspace*{-0.3cm}
  \caption{Definition of the string \textsl{turing}.}
  \label{fig:turing-string}
\end{figure}

Given this definition it is easy to check that \textsl{turing} is, indeed, a test function with the name
``\texttt{alan}'', that is we have 
\\[0.3cm]
\hspace*{1.3cm} 
$\textsl{turing} \in T\!F \;\wedge\; \mathtt{name}(\textsl{turing}) = \mathtt{alan}$. 
\\[0.2cm]
Therefore, we can use the string \textsl{turing} as the first argument of the function
\texttt{stops}.  Let us determine the value of the following expression:
\\[0.2cm]
\hspace*{1.3cm} 
\texttt{stops(\textsl{turing}, \textsl{turing})} 
\\[0.2cm]
Since we have already noted that \textsl{turing} is test function, according to the specification of
the function \texttt{stops} there are only two cases left:
\\[0.2cm]
\hspace*{1.3cm} 
$\mathtt{stops}(\textsl{turing}, \textsl{turing}) \leadsto 0 \quad \vee\quad
 \mathtt{stops}(\textsl{turing}, \textsl{turing}) \leadsto 1$. 
\\[0.2cm]
Let us consider these cases in turn.
\begin{enumerate}
\item $\mathtt{stops}(\textsl{turing}, \textsl{turing}) \leadsto 0$. 

      According to the specification of \texttt{stops} we should then have
      \\[0.2cm]
      \hspace*{1.3cm}
      $\mathtt{alan}(\textsl{turing}) \uparrow$.
      \\[0.2cm]
      Let us check whether this is true.  In order to do this, we have to check what happens when
      the expression
      \\[0.2cm]
      \hspace*{1.3cm}
      \texttt{alan(\textsl{turing})} 
      \\[0.2cm]
      is evaluated:
      \begin{enumerate}
      \item Since we have assumed for this case that the expression 
            $\mathtt{stops}(\textsl{turing}, \textsl{turing})$ yields $0$, 
            in line 2, the variable \texttt{result} is assigned the value 0. 
      \item Line 3 now tests whether \texttt{result} is $1$.  Of course,
            this test fails.  Therefore, the block of the \texttt{if}-statement is not executed.
      \item Finally, in line 8 the value of the variable \texttt{result} is returned. 
      \end{enumerate}
      All in all we see that the call of the function \texttt{alan} does terminate when given the argument
      \textsl{turing}.  However, this is the opposite of what the function \texttt{stops} has claimed.
      
      Therefore, this case has lead us to a contradiction.
\item  $\mathtt{stops}(\textsl{turing}, \textsl{turing}) \leadsto 1$. 

      According to the specification of \texttt{stops} we should then have
      \\[0.2cm]
      \hspace*{1.3cm}
      $\mathtt{alan}(\textsl{turing}) \downarrow$, 
      \\[0.2cm]
      i.e.~the evaluation of $\mathtt{alan}(\textsl{turing})$ should terminate.
      
      Again, let us check in detail whether this is true.  
      \begin{enumerate}
      \item Since we have assumed for this case that the expression 
            $\mathtt{stops}(\textsl{turing}, \textsl{turing})$ yields $1$, 
            in line 2, the variable \texttt{result} is assigned the value $1$. 
      \item Line 3 now tests whether \texttt{result} is $1$.  Of course,
            this time the test succeeds.  
            Therefore, the block of the \texttt{if}-statement \underline{is} executed.
      \item However, this block contains an infinite loop.  Therefore, the
            evaluation of $\mathtt{alan}(\textsl{turing})$ \underline{diver}g\underline{es}.
            But this contradicts the specification of \texttt{stops}!
      \end{enumerate}   
      Therefore, the second case also leads to a contradiction.
\end{enumerate}
As we have obtained contradictions in both cases, the assumption that there is a function
\texttt{stops} that solves the halting problem is refuted.
\hspace*{\fill} $\Box$
\vspace*{0.3cm}

\noindent
\textbf{Remark}:
The proof of the fact that the halting problem is undecidable was given 1936 by Alan Turing (1912 -- 1954)
\cite{turing:36}.  Of course, Turing did not solve the problem for \textsl{Python} but rather
for the so called 
\href{http://en.wikipedia.org/wiki/Indirect_proof}{\emph{Turing machines}}.  
A \blue{Turing machine} \index{turing machine} 
can be interpreted as a formal description of an algorithm.  
Therefore, Turing has shown that there is no algorithm that is able to decide whether some given
algorithm will always terminate.
\vspace*{0.3cm}

\noindent
\textbf{Remark}:
At this point you might wonder whether there might be another programming language
that is more powerful so that programming in this more powerful language it would be possible to
solve the halting problem.  However, if you check the proof given for \textsl{Python} you will easily
see that this proof can be adapted to any other programming language that is as least as powerful as
\textsl{Python}. \eox

Of course, if a programming language is very restricted, then it might be possible to check the
halting problem for this weak programming language.  But for any programming language that supports
at least \texttt{while}-loops, \texttt{if}-statements, and the definition of procedures the argument
given above shows that the halting problem is not solvable.

\exerciseEng
Show that if the halting problem would be solvable, then it would be possible to write a program that checks
whether there are infinitely many \blue{twin primes}.  
\index{twin prime}
A \blue{twin prime} is pair of natural numbers
$\langle p, p + 2 \rangle$ such that both $p$ and $p+2$ are prime numbers.  
The \href{http://en.wikipedia.org/wiki/Twin_prime_conjecture}{\emph{twin prime conjecture}} is one
of the oldest unsolved mathematical problems.  \eox

\exerciseEng
A set $X$ is \blue{countably infinite}\index{countably infinite} iff $X$ is infinite and there is a function 
\\[0.2cm]
\hspace*{1.3cm}
 $f: \mathbb{N} \rightarrow X$ 
\\[0.2cm]
such that for all $x\in X$ there is a $n \in \mathbb{N}$ such that $x$ is the image of
$n$ under $f$: 
\\[0.2cm]
\hspace*{1.3cm} $\forall x \in X: \exists n \in \mathbb{N}: x = f(n)$.
\\[0.2cm]
(A function of this kind is called \blue{surjective}. \index{surjective}
Some authors define a set to be countably infinite iff
there is an \blue{injective} \index{injective} function $f:\mathbb{N} \rightarrow X$.  It can be shown that if there is a
surjective function $f:\mathbb{N} \rightarrow X$ and $X$ is infinite, then there also is an injective function
$f:\mathbb{N} \rightarrow X$.  Therefore, these definitions are equivalent.) 
If a set is infinite, but not countably infinite, we call it \blue{uncountable}.
Prove that the set $2^\mathbb{N}$, which is the set of all subsets of $\mathbb{N}$ is \underline{not} countably
infinite. 

\vspace*{0.2cm}

\noindent
\textbf{Hint}:  Your proof should be similar to the proof that the halting problem is undecidable. 
Proceed as follows:
Assume that there is a function $f$ enumerating the subsets of $\mathbb{N}$, that is assume that 
\\[0.2cm]
\hspace*{1.3cm}
$\forall x \in 2^\mathbb{N}: \exists n \in \mathbb{N}: x = f(n)$
\\[0.2cm]
holds.  Next, and this is the crucial step, define a set \texttt{Cantor} as follows:
\\[0.2cm]
\hspace*{1.3cm} $\mathtt{Cantor} := \bigl\{ n \in \mathbb{N} \mid n \notin f(n) \bigr\}$.
\\[0.2cm]
Now try to derive a contradiction.  \eox



\section[The Equivalence Problem]{Undecidability of the Equivalence Problem}
Unfortunately, the halting problem is not the only undecidable problem in computer science.  Another
important problem that is undecidable is the question whether two given functions always compute the
same result.  To state this more formally, we need the following definition.


\begin{Definition}[$\simeq$] 
Assume $n_1$ and $n_2$ are the names of two \textsl{Python} functions that take arguments
  $a_1$, $\cdots$, $a_k$.  Let us define \\[0.2cm]
\hspace*{1.3cm} 
$n_1(a_1,\cdots,a_k) \simeq n_2(a_1,\cdots,a_k)$ 
\\[0.2cm]
if and only if either of the following cases is true:
\begin{enumerate}
\item $n_1(a_1,\cdots,a_k)\uparrow \quad\wedge\quad n_2(a_1,\cdots,a_k)\uparrow$,

      that is both function calls diverge.
\item $\exists r: \Bigl(n_1(a_1,\cdots,a_k) \leadsto r \quad\wedge\quad n_2(a_1,\cdots,a_k) \leadsto
  r\Bigr)$

      that is both function calls terminate and compute the same result.
\end{enumerate}
If $n_1(a_1,\cdots,a_k) \simeq n_2(a_1,\cdots,a_k)$ holds, then the expressions $n_1(a_1,\cdots,a_k)$ and $n_2(a_1,\cdots,a_k)$ are 
\blue{partially equivalent}. 
\index{partially equivalent}
\hspace*{\fill} $\Box$
\end{Definition}

\noindent
We are now ready to state the \blue{equivalence problem}.  A \textsl{Python} function \texttt{equal} solves the
\emph{equivalence problem} \index{equivalence problem} if it is defined as
\\[0.2cm]
\hspace*{1.3cm} \texttt{def equal(p1, p2, a):}               \\
\hspace*{2.1cm} \textsl{body}                                \\[0.2cm]
and, furthermore, it satisfies the following specification:
\begin{enumerate}
\item $p_1 \not\in T\!F \;\vee\; p_2 \not\in T\!F \quad\Leftrightarrow\quad \mathtt{equal}(p_1, p_2, a) \leadsto 2$.
\item If 
      \begin{enumerate}
      \item $p_1 \in T\!F \;\wedge\; \mathtt{name}(p_1) = n_1$,
      \item $p_2 \in T\!F \;\wedge\; \mathtt{name}(p_2) = n_2$ \quad and
      \item $n_1(a) \simeq n_2(a)$
      \end{enumerate}
      holds, then we must have: 
      \\[0.2cm]
      \hspace*{1.3cm} 
      $\mathtt{equal}(p_1, p_2, a) \leadsto 1$.
\item Otherwise we must have \\[0.2cm]
      \hspace*{1.3cm} 
      $\mathtt{equal}(p_1, p_2, a) \leadsto 0$.
\end{enumerate}


\begin{Theorem}
The equivalence problem is undecidable.  
\end{Theorem}

\noindent
\textbf{Proof}:
The proof is by contradiction.  Therefore, assume that there is a function \texttt{equal}
such that \texttt{equal} solves the equivalence problem.  Assuming \texttt{equal} exists, we will
then proceed to define a function \texttt{stops} that solves the halting problem.
Figure \ref{fig:stops} shows this construction of the function \texttt{stops}.


\begin{figure}[!h]
  \centering
\begin{minted}[ frame         = lines, 
                framesep      = 0.3cm, 
                numbers       = left,
                numbersep     = -0.2cm,
                bgcolor       = sepia,
                xleftmargin   = 0.3cm,
                xrightmargin  = 0.3cm
              ]{python3}
     def stops(t, a):
         l = """def loop(x): 
                    while True:
                        x = 1
             """ 
         e = equal(l, t, a);
         if e == 2:
             return 2
         else:
             return 1 - e
\end{minted}
  \vspace*{-0.3cm}
  \caption{An implementation of the function \texttt{stops}.}
  \label{fig:stops}
\end{figure}

Notice that in line 6 the function \texttt{equal} is called with a string that is test function with
name \texttt{loop}.  This test function has the following form:
\begin{verbatim}
        def loop(x): 
             while True:
                 x = 1
\end{verbatim}
Independent from the argument $x$, the function \texttt{loop} does not terminate.
Therefore, if the first argument $t$ of \texttt{stops} is a test function with name $n$, 
the function \texttt{equal} will return $1$ if $n(a)$ diverges, and will return $0$ otherwise.
But this implementation of \texttt{stops} would then solve the halting problem as
for a given test function $t$ with name $n$ and argument $a$ the function \texttt{stops} would
return 1 if and only the evaluation of $n(a)$ terminates.  As we have already proven that the
halting problem is undecidable, there can be no function \texttt{equal} that solves the equivalence
problem either.
\qed

\remarkEng
The unsolvability of the equivalence problem has been proven by \href{http://en.wikipedia.org/wiki/Henry_Gordon_Rice}{Henry Gordon Rice} \cite{rice:1953} in 1953.
\eox
\pagebreak

\section{Concluding Remarks}
Although, in general, we cannot decide whether a program terminates for a given input, this does not mean
that we should not attempt to do so.  After all, we only have proven that there is no procedure that
can \underline{alwa}y\underline{s} check whether a given program will terminate.  There might well exist a
procedure for termination checking that works most of the time.  Indeed, there are a number of
systems that try to check whether a program will terminate for every input.  For example, for
\href{https://en.wikipedia.org/wiki/Prolog}{Prolog}
programs, the paper
``\href{http://link.springer.com/chapter/10.1007%2F3-540-61739-6_44}{\emph{Automated Modular Termination Proofs for Real Prolog Programs}}''
\cite{mueller:1996} describes a successful approach.  The recent years have seen a lot of progress in
this area.  The article 
``\href{http://dl.acm.org/citation.cfm?id=1941509}{{Proving Program Termination}}''
\cite{cook:2011} reviews these developments.  However, as the recently developed systems rely on both
\href{http://en.wikipedia.org/wiki/Automated_theorem_proving}{\emph{automatic theorem proving}} and
\href{http://en.wikipedia.org/wiki/Ramsey_theory}{\emph{Ramsey theory}} they are quite out of the
scope of this lecture.

\section{Chapter Review}
You should be able to solve the following exercises.
\begin{enumerate}[(a)]
\item Define the halting problem.
\item Prove that the halting problem is not decidable.
\item Define the equivalence problem.
\item Prove that the equivalence problem is not decidable.
\item Define the notion of a countable set.
\item Prove that the set $2^{\mathbb{N}}$ is not countable.
\end{enumerate}


\section{Further Reading}
The book ``\emph{Introduction to the Theory of Computation}'' by Michael Sipser \cite{sipser:1996}
discusses the undecidability of the halting problem in section 4.2.  It also covers many related
undecidable problems.

Another good book discussing undecidability is the book 
``\emph{Introduction to Automata Theory, Languages, and Computation}'' written by John E.~Hopcroft,
Rajeev Motwani and Jeffrey D.~Ullman \cite{hopcroft:06}.  This book is the third edition of a
classic text.  In this book, the topic of undecidability is discussed in chapter 9.

The exposition in these books is based on
\href{https://en.wikipedia.org/wiki/Turing_machine}{Turing machines} and is therefore more formal than the
exposition given here.  This increased formality is necessary to prove that, for example, it is undecidable
whether two \href{https://en.wikipedia.org/wiki/Context-free_grammar}{context free grammars} are equivalent.

A word of warning: The two books mentioned above are not intended to be read by undergraduates in their first
year.  If you want to dive deeper into the concept of undecidability, you should do so only after you have
finished your second year.



%%% Local Variables: 
%%% mode: latex
%%% TeX-master: "logic.tex"
%%% End: 
